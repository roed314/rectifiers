\documentclass[11pt]{amsart}
\usepackage{amsmath,amscd,amssymb,latexsym, amsfonts}
\usepackage{mathtools}

\theoremstyle{plain}
\newtheorem{iassumption}{Assumption}
\newtheorem{theorem}{Theorem}[section]
\newtheorem{conjecture}[theorem]{Conjecture}
\newtheorem{proposition}[theorem]{Proposition}
\newtheorem{corollary}[theorem]{Corollary}
\newtheorem{hypothesis}[theorem]{Hypothesis}
\newtheorem{assumption}[theorem]{Assumption}
\newtheorem{lemma}[theorem]{Lemma}
\newtheorem{question}[theorem]{Question}
\newtheorem{exercise}[theorem]{Exercise}
\newtheorem{statement}[theorem]{Statement}
\newtheorem{example}[theorem]{Example}

\newcommand{\MAxxx}[1]{$\clubsuit$\footnote{#1}}
\newcommand{\DRxxx}[1]{$\spadesuit$\footnote{#1}}
\newcommand{\HT}[1]{\hat{\HH}{}^{#1}}

\theoremstyle{definition}
\newtheorem{definition}[theorem]{Definition}
\newtheorem{remark}[theorem]{Remark}

\setlength{\oddsidemargin}{0.2in}
\setlength{\evensidemargin}{0.2in}
\setlength{\textwidth}{6.1in}

\DeclareMathOperator{\Gal}{Gal}
\DeclareMathOperator{\val}{val}
\DeclareMathOperator{\HH}{H}
\DeclareMathOperator{\Ad}{Ad}
\DeclareMathOperator{\Nm}{Nm}
\DeclareMathOperator{\Hom}{Hom}
\DeclareMathOperator{\Spec}{Spec}
\DeclareMathOperator{\Res}{Res}
\DeclareMathOperator{\Fr}{Fr}
\DeclareMathOperator{\Ind}{Ind}
\DeclareMathOperator{\gen}{gen}
\DeclareMathOperator{\Z}{Z}

\DeclareMathOperator{\GL}{GL}
\DeclareMathOperator{\PGL}{PGL}
\DeclareMathOperator{\SL}{SL}

\newcommand{\mat}[4]{\left( \begin{array}{cc} {#1} & {#2} \\ {#3} & {#4}
\end{array} \right)}
\newcommand{\TT}{\mathcal{T}}
\newcommand{\C}{\mathcal{C}}
\newcommand{\CC}{\mathbb{C}}
\newcommand{\CCx}{\mathbb{C}^\times}
\newcommand{\OK}{\mathcal{O}_K}
\newcommand{\OKn}{\mathcal{O}_{K_n}}
\newcommand{\PK}{\mathcal{P}_K}
\newcommand{\PL}{\mathcal{P}_L}
\newcommand{\OL}{\mathcal{O}_L}
\newcommand{\ZZ}{\mathbb{Z}}
\newcommand{\QQ}{\mathbb{Q}}
\newcommand{\Gm}{\mathbb{G}_m}
\newcommand{\Lx}{L^\times}
\newcommand{\Fq}{\mathbb{F}_q}
\newcommand{\Fqb}{\bar{\mathbb{F}}_q}

\newcommand{\Weil}{\mathcal{W}}
\newcommand{\WD}{\mathcal{W}'}
\newcommand{\Lpack}{\mathcal{L}}
\newcommand{\Pmin}{P_G^{\min}}
\newcommand{\bmu}{\boldsymbol\mu}
\newcommand{\mumin}{\bmu^{\min}}

\newcommand{\st}{\ensuremath{\ \ \ \vert\ }}
\newcommand{\la}{\langle}
\newcommand{\ra}{\rangle}

\newcommand{\invlim}[1]{\varprojlim_{#1}}
\newcommand{\Normalizer}[2]{\operatorname{N}_{#2}(#1)}

\newcommand{\Thadm}{T^*_{\operatorname{adm}}}
\newcommand{\Thinadm}{T^*_{\operatorname{inadm}}}
\newcommand{\hatT}{T^*}

\begin{document}
\title{Rectifiers and the local Langlands Correspondence: the unramified case}
\author{Moshe Adrian and David Roe}
%\address{Department of Mathematics, University of Utah, Salt Lake City, UT 84112, U.S.A.}

%\email{savin@math.utah.edu}


\begin{abstract}

In this paper we initiate a program to generalize the
rectifier of Bushnell and Henniart, which occurs in the local Langlands correspondence for
$\GL_{n}(K)$, to groups outside of $\GL_{n}(K)$.  Our generalization begins with unramified
connected reductive
groups, and certain classes of Langlands parameters.

\end{abstract}

\maketitle

\section{Introduction} \label{section:intro}

Let $G$ be a connected reductive group, defined over a $p$-adic field $K$.
The local Langlands conjecture predicts the existence of a finite to one map,
from the set of isomorphism classes of irreducible admissible representations
of $G(K)$, to the set of equivalence classes of admissible homomorphisms
from the Weil-Deligne group $\WD_K$, to the $L$-group ${}^L G$.

In the case of supercuspidal representations of $G(K)$, there has been a
significant amount of progress in the local Langlands correspondence in
recent years.  In the recent work of \cite{bushnellhenniart}, \cite{debackerreeder},
\cite{kaletha}, and \cite{reeder}, the strategy of constructing a local Langlands
correspondence is to first attach a character of an elliptic torus to a Langlands
parameter $\Weil_K \rightarrow {}^L G$, where $\Weil_K$ is the Weil group of
$K$, and then to construct a putative supercuspidal $L$-packet associated to
this character.

Some of the difficulty arises in the construction of the character of an elliptic
torus.  The difficulty arises from the fact that a Langlands parameter
$\Weil_K \rightarrow {}^L G$ does not necessarily have image inside the
$L$-group of a maximal torus that is dual to an elliptic torus.
If it did, then the local Langlands correspondence
for tori could give us a desired character of an elliptic torus.
Instead, one has to rectify the
situation in some way.

The simplest example of this difficulty arising is in the case of supercuspidal
representations of $\PGL_{2}(K)$.  Suppose that $p \neq 2$, and that
$\varphi : \Weil_K \rightarrow \SL_{2}(\CC)$ is a homomorphism that is
irreducible as a representation.  Then $\varphi = \Ind_{\Weil_L}^{\Weil_K}(\chi)$,
where $L/K$ is a tamely ramified quadratic extension, for some character
$\chi$ of $\Lx$ that is trivial on $\Nm_{L/K}(\Lx)$, where $\Nm_{L/K}$
denotes the norm map from $L$ to $K$.  The image of $\varphi$
is contained in the normalizer of the dual split torus $\hat{T}$.  This normalizer
is a non-split extension of $\Gal(L/K)$ by $\hat{T}$.
In particular, the image of $\varphi$ is not contained in the $L$-group of a torus.

Moreover, we have the exact sequence
$$1 \rightarrow \mathbb{Z} / 2 \mathbb{Z} \rightarrow \Lx / \Nm_{L/K}(\Lx) \rightarrow L^1 \rightarrow 1$$
$$\ \ \ \ \ \ \ \ \ \ \ \ \ \ \ \ \ \ \ \ \ \ w \Nm_{L/K}(L^{\times}) \mapsto w / \sigma(w)$$
where $L^1$ denotes the norm one elements in $L$, and where $\sigma$ is
the nontrivial element of $\Gal(L/K)$.  We note that $L^1$
is an elliptic torus in $\PGL_{2}(K)$.  In particular, the Langlands parameter
$\varphi$ naturally provides a character $\chi$, not of the elliptic torus
$L^1 \subset \PGL_2(K)$,
but of the two-fold cover $\Lx / \Nm_{L/K}(\Lx)$.  To construct a supercuspidal
representation of $\PGL_{2}(K)$, we need a character of $L^1$.
To rectify the situation at hand, we can twist $\chi$ by a genuine character of
$\Lx / \Nm_{L/K}(\Lx)$ in order to obtain a character of $L^1$.  The twist that gives
the correct supercuspidal representation of $\PGL_2(K)$ is exactly what
appears in the rectifier of Bushnell and Henniart (see \cite{bushnellhenniart2} and \cite{bushnellhenniart}).

In this paper, we initiate a program to generalize Bushnell and Henniart's
notion of a
rectifier to groups outside of $\GL_n(K)$.  In particular, we define the notion of a rectifier for
unramified minisotropic tori $T$ in an arbitrary unramified connected reductive group $G$.
We use a recent construction of Benedict
Gross, called ``groups of type L'', in order to give us a framework that
will allow us to define a rectifier.  This construction also allows us to define
a new notion of admissible pair.   We will show there exists a unique rectifier for $T$, up
to equivalence. We will then show that our rectifier is compatible
with the rectifier of Bushnell and Henniart in the setting of depth zero
supercuspidal representations of $\GL_{n}(K)$.  In this setting, there is
a canonical construction of supercuspidal representations of $\GL_{n}(K)$, and in
particular, there is a canonical ``naive'' construction of the local Langlands correspondence.
This means, in particular, that in this depth zero case, the rectifier is canonical.
In the setting of positive depth supercuspidal representations of $\GL_{n}(K)$,
there is no canonical construction of supercuspidal representations of $\GL_{n}(K)$,
and therefore there is no canonical ``naive'' construction of the local Langlands correspondence.
In particular, in the positive depth setting, a rectifier depends on the
methods used to construct supercuspidal
representations, and so there is no canonical rectifier.
Since we are using Adler's construction of supercuspidal representations
(see \cite{adler}),
which may be different than Bushnell and Henniart's construction of supercuspidal
representations in \cite{bushnellhenniart}, it does not make sense to compare our rectifier
to that of Bushnell and Henniart in this setting.  Indeed, our rectifiers differ in the
positive depth setting due to the different constructions of supercuspidal representations.

Bushnell and Henniart motivate their rectifier as follows.
Suppose that $\varphi$ is an \emph{essentially tame} supercuspidal Langlands parameter for
$\GL_n(K)$.  The local Langlands correspondence for tori then yields a canonical character $\xi$
of $L^{\times}$, where $L/K$ is some degree $n$ separable extension.  The rectifier of $\xi$ is a character
$\mu_{\xi}$ of $L^{\times}$ such that $\varphi \mapsto \pi_{\xi \cdot \mu_{\xi}}$ is
the local Langlands correspondence for $\GL_n(K)$.  Here the association
$\chi \mapsto \pi_{\chi}$ is some fixed construction of supercuspidal representations of $\GL_n(K)$
from \emph{admissible} characters of $L^{\times}$.

We generalize their notion of rectifier to other reductive groups $G$.
Suppose that $\varphi : \Weil_K \rightarrow {}^L G$ is a Langlands parameter
that factors through the normalizer of a maximal torus.
The local Langlands for tori again provides a canonical way to proceed.  Assuming a mild
cohomological condition, one
obtains from $\varphi$ a character $\xi$ of a cover of an elliptic torus.
The rectifier of $\xi$ is then a character $\mu_{\xi}$ of this cover such that
$\varphi \mapsto \Lpack(\xi \cdot \mu_{\xi})$
 is the local Langlands correspondence for $G(K)$.  Again, we must fix an association
$\chi \mapsto \Lpack(\xi \cdot \mu_{\xi})$
of supercuspidal $L$-packets of $G$
to \emph{admissible} characters of this cover.

We now present an outline of the paper.  In \S\ref{section:BH_recall} we recall
the notion of rectifier due to Bushnell and Henniart, and we describe
the rectifier in the setting that we will need.  In \S\ref{section:padic_tori}
we present some results about Tate cohomology of $p$-adic tori that will be used
in the rest of the paper.  In \S\ref{section:groups_of_type_L} we introduce
the theory of ``groups of type L'', due to Benedict Gross.  In
\S\ref{section:gross_debacker_reeder} we describe the relationship between the
construction of Gross, via groups of type L, and the constructions of
DeBacker-Reeder and Reeder.  In \S\ref{Q_T} we study how translation by a character affects
the association $\chi \mapsto \Lpack(\xi \cdot \mu_\xi)$.
In \S\ref{section:general_rectifiers} we
introduce our notion of rectifier, we state a conjecture, and then we
state and prove the main theorem of this paper (see Theorem \ref{thm:unique_semisimple}).
Finally, in \S\ref{section:BH_compat} we show that our rectifier is compatible with
the rectifier of Bushnell and Henniart in the setting of depth zero
supercuspidal representations of $\GL_n(K)$.

\subsection*{Acknowledgements}

This paper has benefited from conversations with Gordan Savin, Jeffrey Adams,
Geo Kam Fai, Andrew Fiori, Jeffrey Adler, Guy Henniart, and Colin Bushnell.  We thank them all.

\section{Notation and Preliminaries} \label{section:notation}

Throughout, $k$ will denote a finite field, and
$k_n$ will denote the degree $n$ extension of
$k$.  $K$ will denote a nonarchimedean local field of
charcteristic zero, $\OK$ its ring of integers,
$\PK$ the maximal ideal in $\OK$,
and $k$ will also denote the residue field of $K$.  Fix
a uniformizer $\varpi$ of $K$.  Let $K^u$ denote the maximal
unramified extension of $K$, and fix a valuation $\val : (K^u)^\times
\rightarrow \mathbb{Z}$ normalized so that $\val(\varpi) = 1$.
Write $K_n$ for the unramified extension of $K$ of degree $n$, $k_n$ for
the degree $n$ extension of $k$,
and set $\Gamma_n = \Gal(K_n/K) = \Gal(k_n/k)$.

A geometric Frobenius is an element of $\Gal(\bar{K}/K)$
inducing the automorphism $x \mapsto x^{1/p}$ of $\bar{k}$.  Under the
Artin reciprocity map of local class field theory the choice of uniformizer $\varpi$
determines a geometric Frobenius $\Fr$ \cite[\S 2]{serre1}.

If $\chi : K^{\times} \rightarrow \mathbb{C}^{\times}$ is a character, we define
the \emph{level} of $\chi$ to be the smallest integer $r$ such that
$\chi|_{1 + \PK^{r+1}} \equiv 1$, with
$\chi|_{1 + \PK^{r}} \not\equiv 1$.

If $T$ is a torus defined over $K$ we write $X^*(T)$
for the character lattice $\Hom_{\bar{K}}(T, \Gm)$ and $X_*(T)$ for the
cocharacter lattice $\Hom_{\bar{K}}(\Gm, T)$ \cite[\S 16.2]{humphreys}.
$T$ will split over an extension
$L$ of $K$ if and only if $\Gal(\bar{K}/L)$ acts trivially on $X^*(T)$.
We may thus define \emph{the} splitting field $L$ of $T$ as the
minimal extension of $K$ splitting $T$; note that $L$ is necessarily
Galois over $K$.  Write $\Gamma$ for $\Gal(L/K)$; $X_*(T)$, $X^*(T)$ and $T(L)$
are all $\Gamma$-modules.

Let $\hat{T}$ be the dual torus over $\CC$ with $X_*(\hat{T}) = X^*(T)$ and
$X^*(\hat{T}) = X_*(T)$.  By transport of structure $\hat{T}$ is also a $\Gamma$-module.
For $\lambda \in X_*(T)$ and $\mu \in X^*(T)$ we write $\la \lambda, \mu \ra$
for the canonical $\Gamma$-equivariant pairing
$$\Hom_{\bar{K}}(\Gm, T) \times \Hom_{\bar{K}}(T, \Gm) \rightarrow \Hom_{\bar{K}}(\Gm, \Gm) \cong \ZZ.$$

Suppose now that $T \subset G$ for a connected reductive group $G$ over $K$.  Let $\hat{G}$ denote
the complex dual group of $G$.
We write $N$ for the normalizer $\Normalizer{T}{G}$ of $T$ in $G$ and define $W = N/T$;
set $\hat{N} = \Normalizer{\hat{T}}{\hat{G}}$ and
$\hat{W} = \hat{N}/\hat{T}$.  The identification of $X^*(T)$ and $X_*(\hat{T})$
yields a canonical anti-isomorphism between $W$ and $\hat{W}$.
Note that $W$ is a scheme over $K$; in general $W(K) \ne N(K) / T(K)$.

Write $\Nm$ for the norm map
\begin{align*}
T(L) &\rightarrow T(K) \\
t &\mapsto \prod_{\sigma \in \Gamma} \sigma(t)
\end{align*}
and for its restriction to $X_*(T)$.

The following theorem, due to Lang \cite{lang}, underpins the facts in
\S\ref{section:padic_tori} on tori over $p$-adic fields.
Let $H$ be a commutative connected algebraic group over a
finite field $k$, and suppose $H$ splits over $k_n$.

\begin{theorem} \label{thm:lang}
$\HT{i}(\Gamma_n, H(k_n)) = 0$ for all $i$.
\end{theorem}
\begin{proof}
Since $\Gamma_n$ is cyclic,
$\HT{i}(\Gamma_n, H(k_n)) \cong \HT{i+2}(\Gamma_n, H(k_n))$ \cite[Thm. 5]{atiyah-wall},
so it suffices to prove the result for $i=1$ and $i=2$, which is done
by Serre \cite[\S VI.6]{serre2}.
\end{proof}

\section{Rectifier for $\GL_{n}(K)$} \label{section:BH_recall}

In this section, we recall the rectifier of Bushnell and Henniart,
and we describe the rectifier in the setting
that we will need.

We first recall the construction of the
essentially tame local Langlands correspondence for $\GL_{n}(K)$, due to Bushnell
and Henniart.  An irreducible smooth representation of the Weil group $\Weil_K$ of $K$ is
called \emph{essentially tame} if its restriction to wild inertia is a
sum of characters.  In the
essentially tame setting, there is a natural bijection
$\varphi_{\xi} \leftrightarrow (L/K, \xi)$, between irreducible representations
$\varphi_{\xi} : \Weil_K \rightarrow \GL_{n}(\mathbb{C})$, and
\emph{admissible pairs} $(L/K, \xi)$.
Here, $L/K$ is a degree $n$ separable extension
and $\xi$ is a character (satisfying certain conditions) of $\Lx$.
Note that one can view
$L^{\times}$ as a maximal elliptic torus in $\GL_{n}(K)$.  Bushnell and Henniart
construct a map (see \cite{bushnellhenniart})
\begin{equation*}
\left\{
\begin{array}{cc}
\mathrm{isomorphism \ classes \ of} \\
\mathrm{admissible \ pairs}
\end{array}
\right\} \rightarrow \left\{
\begin{array}{cc}
\mathrm{supercuspidal \ representations} \\
\mathrm{of} \ \GL_{n}(K)
\end{array} \right\}
\end{equation*}
$$\hspace{-.5in} (L/K, \xi) \mapsto \pi_{\xi}$$
The problem is that the map $$\varphi_{\xi} \mapsto \pi_{\xi}$$
is not the local Langlands
correspondence because $\pi_{\xi}$ has the wrong central character.
Instead, the local Langlands correspondence is given by

\begin{equation}\label{llcgln}
\varphi_{\xi} \mapsto \pi_{\xi \cdot {}_K \mu_{\xi}} \tag{$\star$}
\end{equation}

for some subtle finite order
character ${}_K \mu_{\xi}$ of $\Lx$.  Since we will not be changing $K$
in this paper we will write $\mu_\xi$ for ${}_K \mu_{\xi}$.

The relation $\star$ does not determine $\mu_{\xi}$ uniquely.  As pointed out
in \cite{bushnellhenniart}, the obstruction to uniqueness revolves around the
group $\GL_2(\mathbb{F}_3)$.  Bushnell and Henniart therefore make the following definition.

\begin{definition}\label{rectifierbushnellhenniart}
Let $L/K$ be a finite, tamely ramified field extension, of degree $n$.  A \emph{rectifier}
for $L/K$ is a function
$$\bmu : (L/K, \xi) \mapsto \mu_{\xi}$$
which attaches to each admissible pair $(L/K, \xi)$, a character $\mu_{\xi}$ of $L^{\times}$
satisfying the following conditions.
\begin{enumerate}
\item The character $\mu_{\xi}$ is tamely ramified.
\item Writing $\xi' = \xi \cdot \mu_{\xi}$, the pair $(L/K, \xi')$ is admissible and
$\varphi_{\xi} \mapsto \pi_{\xi \cdot \mu_{\xi}}$ is the local Langlands correspondence
for $\GL_n(K)$.
\item If $(L/K, \xi_i), i = 1,2$, are admissible pairs such that $\xi_1^{-1} \xi_2$ is
tamely ramified, then $ \mu_{\xi_1} =  \mu_{\xi_2}$.
\end{enumerate}
\end{definition}

Bushnell and Henniart then prove (see \cite{bushnellhenniart}):

\begin{theorem}
Any finite, tamely ramified, field extension $L/K$ admits a unique rectifier
$\bmu : (L/K, \xi) \mapsto \mu_{\xi}$.
\end{theorem}

Both the description of and the
intuition behind the rectifiers $\bmu$ have been
studied (see \cite{bushnellhenniart}, \cite{geo}, \cite{adrian}).  As stated in
the introduction, the goal of this paper is to initiate a program to
generalize this rectifier to groups outside of $\GL_n(K)$.

We will need a description of the characters $\mu_{\xi}$ in certain cases.
Let us recall some notions from \cite[\S8]{bushnellhenniart}.

Let $(L/K,\xi)$ be an admissible pair, and let $i \in \mathbb{Z}, i \geq 0$.
Then there is a unique sub-extension $L_i/K$ of $L/K$ such that
$\xi|_{1 + \PL^{i+1}}$ factors through the norm $\Nm_{L/L_i}$,
and which is minimal for this property.  We say that $i \in \mathbb{Z}$
is a \emph{jump of} $\xi$ \emph{over} $K$ if $i \geq 1$ and $L_{i-1} \neq L_i$.

\begin{proposition}\label{prop:BH_result1}
  Suppose that $(L/K, \xi)$ is an admissible pair, where $L/K$
  is unramified and $\xi$ has level $0$.
  Then $\mu_{\xi}$ is unramified and
  $\mu_{\xi}(\varpi) = (-1)^{n-1}$.
\end{proposition}

\proof
It is clear that the set of jumps of $\xi$ over $F$ is empty.
Therefore, by \cite[Proposition 21]{bushnellhenniart}, we have the result.
\qed

\section{Tori over $p$-adic fields} \label{section:padic_tori}

Let $T$ be a torus defined over $K$.    Let $n$ be such that $K_n$ is the maximal
unramified subextension of the splitting field $L$, and set $I = \Gal(L/K_n)$.

Let $\TT$ be the N\'eron model of $T$, a canonical model of $T$
over $\OK$ \cite[Ch. 10]{blr}.  As a consequence of the N\'eron mapping
property, we may identify $\TT(\OK)$ with $T(K)$.  The connected
component of the identity, $\TT^\circ$, cuts out a subgroup
$T(K)_0 = \TT^\circ(\OK)$ of $T(K)$; we also write $T(K_n)_0$ for
$\TT^\circ(\OKn)$.

In fact, this subgroup of $T(K)$ is the first in a decreasing filtration.
Moy and Prasad \cite{moyprasad}
define one such filtration by
embedding $T$ into an induced torus and defining the filtration of
$\Res_{L/K} \Gm$ in terms of the valuation on $L$.  Yu \cite[\S 5]{yu}
describes a different filtration, agreeing with that of Moy and Prasad
in the case of tame tori but with nicer features in the presence of wild
ramification.  Let $\TT_r$ be the integral models of $T$ defined in Yu's
minimal congruent filtration, and let $\{T(K)_r\}_{r \ge 0}$ and
$\{T(K_n)_r\}_{r \ge 0}$ be the corresponding filtrations of $T(K)$ and
$T(K_n)$.

Let $\C$ be the scheme of
connected components of $\TT$,
which we may identify with the
components of $\TT \times \Spec(k)$ since $T = \TT \times \Spec(K)$
is connected.  The structure of $\C$ is described by Xarles:

\begin{proposition}[{\cite[Cor. 2.12]{xarles}}]
There is an exact sequence of $\Gamma_n$-modules
$$0 \rightarrow \Hom_{\ZZ}(\HH^1(I, X^*(T)), \QQ/\ZZ) \rightarrow
\C \rightarrow \Hom_{\ZZ}(X^*(T)^I, \ZZ) \rightarrow 0.$$
\end{proposition}

\begin{corollary}[{\cite[Thm. 1.1]{xarles}}] \label{cor:unram_components}
If $T$ is unramified, then $\C \cong X_*(T)$.
\end{corollary}

Using our filtration of $T(K_n)$, we may relate the cohomology of $T(K_n)$
with that of $\C$.

\begin{proposition}\label{prop:T0_cohom_triv}
$\HT{i}(\Gamma_n, T(K_n)_0) = 0$ for all $i$.
\end{proposition}
\begin{proof}
Note that
$$T(K_n)_0 = \invlim{r} T(K_n)_0 / T(K_n)_r.$$
So by a result of Serre \cite[Lem. 3]{serre1}, it suffices to prove that
$\HT{i}(\Gamma_n, T(K_n)_r / T(K_n)_{r+}) = 0$ for all $i$.  But $T(K_n)_r / T(K_n)_{r+}$
is connected \cite[Prop. 5.2]{yu} and thus has trivial cohomology by
Theorem \ref{thm:lang}.
\end{proof}

\begin{corollary}
$\HT{i}(\Gamma_n, T(K_n)) \cong \HT{i}(\Gamma_n, \C)$.
\end{corollary}

\begin{proof}
This follows from the long exact sequence in cohomology associated to the sequence
$$0 \rightarrow \TT^0 \rightarrow \TT \rightarrow \C \rightarrow 0.$$
\end{proof}

Suppose now that $T$ is unramified with splitting field $L = K_n$.

\begin{corollary} \label{cor:cohom_tori}
If $T$ is unramified, then $\HT{i}(\Gamma_n, T(L)) \cong \HT{i}(\Gamma_n, X_*(T))$
for all $i$.
\end{corollary}

\begin{proof}
This follows from the previous corollary together with Corollary \ref{cor:unram_components}.
\end{proof}

\begin{corollary}\label{cor:vanishing_H0}
If $T$ is unramified and anisotropic, then $\HT{0}(\Gamma_n, T(L)) = 0$.
\end{corollary}

\begin{proof}
Since $T$ is anisotropic $\HH^0(\Gamma_n, T(L)) \cong X_*(T)^{\Gamma_n} = 0$.
\end{proof}

For unramified $T$ the jumps in the filtration on $T(K)$ and $T(L)$ occur at integers, and we write
\begin{align*}
T(\OK) &= T(K)_0, \\
T(\OL) &= T(L)_0, \\
T(\PK^r) &= T(K)_r\qquad \mbox{for $r > 0$}, \\
T(\PL^r) &= T(L)_r\qquad\,\,\mbox{for $r > 0$}.
\end{align*}

\section{Groups of type L} \label{section:groups_of_type_L}

We now review the theory of ``groups of type L'' due to Benedict
Gross.  For a torus $T$ over $K$, recall that the dual torus $\hat{T}$ is equipped with
an action of $\Gamma$.

\begin{definition}
A \emph{group of type L} is a group extension of $\Gamma$ by $\hat{T}$.
\end{definition}

For such a group $D$ we have by definition an exact sequence
$$1 \rightarrow \hat{T} \rightarrow D \rightarrow \Gamma \rightarrow 1.$$

We now describe how, given a Langlands parameter
$$\varphi : \Weil_K \rightarrow D,$$
where $D$ is a group of type L, we can naturally attach a character of the coinvariants
$T(L)_{\Gamma}.$
Restricting $\varphi$ to $\Weil_L$ we get a homomorphism
$$\varphi|_{\Weil_L} : \Weil_L \rightarrow \hat{T}.$$
By the Langlands correspondence for tori, this gives us a character
$\xi_{\varphi} : T(L) \rightarrow \CCx$.  Since $\varphi|_{\Weil_L}$ extends
to $\varphi$, one can see that
$$\xi_{\varphi}(t^{\sigma}) = \xi_{\varphi}(t)\ \mbox{for all $\sigma \in \Gamma$.}$$
Therefore, $\xi_{\varphi}(t^{\sigma - 1}) = 1$ for all $\sigma \in \Gamma$.
Thus, $\xi_{\varphi}$ is trivial on the augmentation ideal $I_{\Gamma}(T(L))$
and gives $$\xi_{\varphi} : T(L)_\Gamma \rightarrow \CCx$$ Invariants
and coinvariants are related by the norm map
$$\Nm : T(L) \rightarrow T(K)$$ $$t \mapsto \displaystyle\prod_{\sigma \in \Gamma} \sigma(t)$$
in the Tate cohomology sequence
$$1 \rightarrow \HT{-1}(\Gamma,T(L)) \rightarrow T(L)_{\Gamma} \xrightarrow{\Nm} T(K)
  = T(L)^{\Gamma} \rightarrow \HT{0}(\Gamma,T(L)) \rightarrow 1$$
(note that the norm map $\Nm$ factors to $T(L)_{\Gamma}$).
We note that $T(L)_{\Gamma}$ is a cover of
$\Nm(T(L)_{\Gamma})$, which is a subgroup of $T(K)$.  It is sometimes
the case that $\Nm$ is surjective, in which case $\xi_{\varphi}$ is then a
character of $T(L)_{\Gamma}$, which is a cover of $T(K)$.

We will need the following structural result about Langlands
parameters mapping to groups of type L later.  Suppose that $L/K$ is now unramified.
Suppose $\varphi$ and $\varphi'$ are two Langlands parameters,
with $\varphi'(\Fr) \varphi(\Fr)^{-1} \in \hat{T}$.
Let $\xi_{\varphi}$ and $\xi_{\varphi'}$ be the associated characters of $T(L)_{\Gamma}$.

\begin{lemma} \label{lem:toral_modification}
$\xi_{\varphi}$ and $\xi_{\varphi'}$ have the same restriction to $\HT{-1}(\Gamma, T(L))$.
\end{lemma}

\begin{proof}
It suffices to prove that $\xi_{\varphi'} \cdot \xi_{\varphi}^{-1}$ vanishes on
$\ker(\Nm : T(L) \rightarrow T(K))$.  Define $g \in D$ and $t \in \hat{T}$ by
$\varphi(\Fr) = g$, $\varphi'(\Fr) = tg$.  Then
\begin{align*}
\varphi'(\Fr^n) \varphi(\Fr^n)^{-1} &= (tg)^n g^{-n} \\
&= \prod_{i=0}^{n-1} g^i t g^{-i} \\
&= \prod_{i=0}^{n-1} \Fr^i(t)
\end{align*}
since $g$ projects to $\Fr \in \Gamma$.  Define $\varphi_i \colon \Weil_L \rightarrow \hat{T}$
by $\varphi_i(z) = 1$ for $z \in I_L$ and
$\varphi_i(\Fr^n) = \Fr^i(t)$; let $\xi_i$ be the associated character
of $T(L)$.  By \cite[Lem. 4.3.1]{debackerreeder}, $\xi_i = \xi_0 \circ \Fr^i.$
Suppose that $x \in T(L)$ with $\Nm(x) = 1$.  Then
\begin{align*}
\xi_{\varphi'}(x) \xi_{\varphi}(x)^{-1} &= \prod_{i=0}^{n-1} \xi_i(x) \\
&= \xi_0 \left(\prod_{i=0}^{n-1}\Fr^i(x)\right) \\
&= 1.\\
\end{align*}
\end{proof}

We will also need the following lemma, in order to define our notion of admissible pair
in \S\ref{section:general_rectifiers}.

\begin{lemma} \label{lem:weyl_groups}
Let $G$ be a connected reductive $K$-group and let $T$ be a maximal
$K$-torus of $G$.
\begin{enumerate}
\item $\Normalizer{T(K)}{G(L)} / T(L) \cong W(K)$.
\item The standard action of $\Normalizer{T(L)}{G(L)} / T(L)$ on $T(L)$ determines
well-defined actions of $\Normalizer{T(L)}{G(L)}^\Gamma / T(K)$ and $W(K)$
on $T(L)$, which factor naturally to actions on $T(L)_\Gamma$.
\end{enumerate}
\end{lemma}

\begin{proof}
See \cite[Lem. 9.1]{adrianlansky}.
\end{proof}

\section{The relationship between the Gross construction and the DeBacker--Reeder and Reeder construction}
\label{section:gross_debacker_reeder}

Let $\varphi : \Weil_K \rightarrow {}^L G$ be a regular semisimple elliptic Langlands
parameter for an unramified connected reductive group $G$
(see \cite{debackerreeder} and \cite{reeder}).
Here, ${}^L G = \langle \hat{\theta} \rangle \ltimes \hat{G}$,
where $\hat{\theta}$ is the dual Frobenius automorphism on $\hat{G}$
(see \cite[\S 3]{debackerreeder}).
Note that $\varphi$ has image in a group of
type L.  Let $L,K,T,\hat{T}, \Gamma, \xi_{\varphi}$ be as in
\S\ref{section:groups_of_type_L}.  Note that $L/K$ is unramified.
Then $\varphi(I_K) \subset \hat{T}$ and
$\varphi(\Fr) = \hat{\theta} f$ for some $f \in \hat{N}$.  Let $\hat{w}$
be the image of $f$ in $\hat{W}$.
DeBacker--Reeder and Reeder (in \cite{debackerreeder}
and \cite{reeder}) associate a character $\chi_{\varphi}$ of $T(K)$ to $\varphi$.

Let us fix a splitting $(\hat{T}, \hat{B}, \{X_{\alpha} \})$
for the dual group $\hat{G}$.  Here $\{ X_{\alpha} \}$ is a set of root vectors
indexed by the set of simple roots of $\hat{T}$ in $\hat{B}$.
For each simple root $\alpha$, let $\phi_{\alpha} : \SL_2 \rightarrow \hat{G}$
be defined by $\phi_{\alpha}(\mathrm{diag}(z,1/z)) = \alpha^{\vee}(z)$
and $d \phi_{\alpha}\mat{0}{1}{0}{0} = X_{\alpha}$. Let
$\sigma_{\alpha} = \phi_{\alpha}\mat{0}{1}{-1}{0}$.

\begin{definition}
  The Tits group $\widetilde{W}$ is the subgroup of $\hat{N}$
  generated by $\{\sigma_{\alpha} \}$ for $\alpha$ simple.
\end{definition}

For each simple root $\alpha$, let $m_{\alpha} = \sigma_{\alpha}^2 = \alpha^{\vee}(-1)$.
Let $\hat{T}_2$ be the subgroup of $\hat{T}$ generated by the $m_{\alpha}$.

\begin{theorem}{(\cite{tits})}
\begin{enumerate}

\item The kernel of the natural map $\widetilde{W} \rightarrow \hat{W}$
  is $\hat{T}_2$,
\item The elements $\sigma_{\alpha}$ satisfy the braid relations,
\item There is a canonical lifting of $\hat{W}$ to a subset of
  $\widetilde{W}$: take a reduced expression $w = s_{\alpha_1} \cdots s_{\alpha_n}$,
  and let $\tilde{w} = \sigma_{\alpha_1} ... \sigma_{\alpha_n}$.
\end{enumerate}
\end{theorem}

\begin{definition}
Given $\hat{u} \in \hat{W}$, let $\tilde{u}$ be its canonical lift to $\widetilde{W}$.
We define a homomorphism $\varphi_{\hat{u}} : \Weil_K \rightarrow {}^L G$ by
\begin{enumerate}
\item $\varphi_{\hat{u}}|_{I_K} \equiv 1$
\item $\varphi_{\hat{u}}(\Fr) = \hat{\theta} \tilde{u}$
\end{enumerate}
\end{definition}

By
\S\ref{section:groups_of_type_L}, $\varphi$ and $\varphi_{\hat{w}}$ gives rise to characters
$\xi_{\varphi}$ and $\xi_{\varphi_{\hat{w}}}$ of $T(L)_{\Gamma}$, respectively.

\begin{lemma} \label{lem:GDR_compat}
$\xi_{\varphi}$ and $\chi_{\varphi} \circ \Nm$ have the same restriction to $T(\OL)_{\Gamma}$.
\end{lemma}

\proof
We have the exact sequence

$$1 \rightarrow \HT{-1}(\Gamma, T(L)) \rightarrow T(L)_{\Gamma} \rightarrow T(K)
  \rightarrow \HT{0}(\Gamma, T(L)) \rightarrow 1.$$

Recall that the character $\xi_{\varphi}$ is associated to $\varphi$ by
the local Langlands correspondence for tori (see \S\ref{section:groups_of_type_L}).
Note that the above exact sequence restricts to an exact sequence

$$1 \rightarrow \HT{-1}(\Gamma, T(\OL)) \rightarrow T(\OL)_{\Gamma}
  \rightarrow T(\OK) \rightarrow \HT{0}(\Gamma, T(\OL)) \rightarrow 1$$

Moreover, by Proposition \ref{prop:T0_cohom_triv}, we have
$\HT{-1}(\Gamma, T(\OL)) = \HT{0}(\Gamma, T(\OL)) = 1$.
Therefore, the map
$$T(\OL)_{\Gamma} \xrightarrow{\Nm} T(\OK)$$
is an isomorphism, so
$\xi_{\varphi}|_{T(\OL)_{\Gamma}}$ factors to a character of
$T(\OK)$ via this isomorphism.  But this is exactly how the character
$\chi_{\varphi}|_{T(\OK)}$ is constructed in \cite{debackerreeder} and \cite{reeder}.
\qed

\begin{proposition}\label{existenceofrectifier}
If $G$ is semisimple, then $\xi_{\varphi} \otimes \xi_{\varphi_{\hat{w}}}^{-1} = \chi_{\varphi} \circ \Nm$.
\end{proposition}

\begin{proof}
Since $G$ is semisimple, $T(K)$ is compact.  In particular,
$\HT{0}(\Gamma, T(L)) = 0$ by Corollary \ref{cor:vanishing_H0},
so we have the following exact sequence:
$$1 \rightarrow \HT{-1}(\Gamma, T(L)) \rightarrow T(L)_{\Gamma} \rightarrow T(K) \rightarrow 1$$
Note that $T(K) = T(\OK)$, and so in particular
$T(\OL)_{\Gamma}$ surjects onto $T(K)$ via the norm map
$\Nm$.  Therefore, $\HT{-1}(\Gamma,T(L))$ and
$T(\OL)_{\Gamma}$ together generate $T(L)_{\Gamma}$.  It thus suffices to check that
$\xi_{\varphi} \otimes \xi_{\varphi_{\hat{w}}}^{-1} = \chi_{\varphi} \circ \Nm$
on each of these two subgroups.

Since $\varphi_{\hat{w}}|_{I_K} \equiv 1$, $\xi_{\varphi_{\hat{w}}}$ is trivial on
$T(\OL)_{\Gamma}$ so Lemma
\ref{lem:GDR_compat} implies equality on $T(\OL)_{\Gamma}$.
Equality on $\HT{-1}(\Gamma,T(L))$ is Lemma \ref{lem:toral_modification}.
\end{proof}

\section{L-packets fixed under translation by a character}\label{Q_T}

The general definition of rectifier is complicated by the fact that different
characters of a torus can yield the same L-packet.  Consider the following archetypical example.
Let $K = \QQ_3$, $G = \SL_2$ and $T$ be an unramified anisotropic torus.  There are four depth zero
characters: two admissible and two inadmissible.  Since the two admissible characters are interchanged
by the action of the Weyl group, the corresponding L-packets are isomorphic \cite[\S10]{murnaghan}.
In this section we investigate characters of $T(K)$ that leave the association $\chi \mapsto \Lpack(\chi)$ invariant upon translation:
$$\Lpack(\chi) = \Lpack(\alpha\cdot\chi) \mbox{ for all regular $\chi$}.$$

\begin{definition} \label{def:admissible}
Let $T$ be a $K$-minisotropic torus, that splits over an unramified
extension $L$ (see \cite[\S3]{reeder}).  Suppose $\xi$ is a character of $T(L)_{\Gamma}$.
The pair $(T, \xi)$ is called \emph{admissible} if $\xi$ is not fixed
by any element of $W(K)$ (see Lemma \ref{lem:weyl_groups}). We say that
two admissible pairs $(T, \xi)$ and $(T', \xi')$ are \emph{isomorphic} if there
exists a $g \in G(K)$ such that $gT(K)g^{-1} = T'(K)$ and $\xi(t) = \xi'(gtg^{-1})$
for all $t \in T(K)$.  We denote by $P_G(K)$ the set of admissible pairs in $G$.
\end{definition}

We note that this definition of admissible pair generalizes
the notion of admissible pair of \cite{bushnellhenniart} in
the case of unramified elliptic tori.  Indeed,
if $G = \GL_n$, and $T$ is an elliptic torus in $G$ splitting over
an unramified extension $L/K$, then one can show that
$W(K) = \Gamma$.  Thus, for $\xi$ to not be fixed by any
element of $W(K)$ is equivalent to $\xi$ not being fixed by
any subgroup of $\Gamma$.  This can be seen to be equivalent to $\xi$
not coming from the norms $\Nm_{L/M}$ for any proper subfield $M$ contained
in $L$, containing $K$.

As for characters of $T(L)_\Gamma$ we will call a character of $T(K)$ \emph{admissible}
if it is not fixed by any nontrivial element
of the rational Weyl group $W(K)$ (see \cite[p. 802]{debackerreeder} and \cite[\S3]{reeder}).
For example, the depth zero character of the split torus in $\SL_3(\QQ_7)$ inflated from
$$\begin{pmatrix} 3^x & & \\ & 3^y & \\ & & 3^{-x-y} \end{pmatrix} \mapsto \zeta_6^{2x + 4y}$$
is fixed by a 3-cycle and thus not admissible.

In the next section we will be particularly interested in depth zero characters; write $\hatT$ for the set of
depth zero characters of $T(\OK)$, $\Thadm$ for the admissible
characters and $\Thinadm$ for the inadmissible ones.  Each of these
sets is finite since they may be identified with characters of $T(k)$.

\begin{definition}
Write $Q_T$ for the set of $\alpha \in \hatT$ with the following property:
\begin{itemize}
\item For every $\chi \in \Thadm$ there is a $w \in W(K)$ with $\frac{\chi}{w(\chi)} = \alpha$.
\end{itemize}
\end{definition}

The $\SL_2(\QQ_3)$ example above has $Q_T$ of order two, but $Q_T$ is trivial for most tori.
We spend the rest of this section giving criteria constraining $Q_T$.

\begin{proposition} \label{irr-sub}
The set $Q_T$ is a subgroup of $\hatT$, contained within $\Thinadm$ and stable under the action of $W(K)$.
\end{proposition}
\begin{proof}
If $\alpha \in \Thadm \cap Q_T$ then there is some $w \in W(K)$ with
$\frac{\alpha}{w(\alpha)} = \alpha$, so $\alpha = 1$ which is not regular.

We now show that $Q_T$ is a group.  Certainly $1 \in Q_T$.  Suppose
$\alpha, \alpha' \in Q_T$ and $\chi \in \Thadm$.  Then there are $w, w' \in W(K)$ with
\begin{align*}
\frac{\chi}{w(\chi)} &= \alpha, \\
\frac{w(\chi)}{w'(w(\chi))} &= \alpha'.
\end{align*}
Multiplying the two relations yields $\frac{\chi}{w'w(\chi)} = \alpha\alpha'$, so
$\alpha\alpha' \in Q_T$.  We finish by noting that $Q_T$ is finite and thus closed under inversion.

Finally, suppose $\tau \in W(K)$.  Given $\chi \in \Thadm$ with $\alpha = \frac{\chi}{w(\chi)}$ we have
$$\tau(\alpha) = \frac{\tau(\chi)}{\tau w(\chi)} = \frac{\tau(\chi)}{w' \tau(\chi)}$$
for some $w' \in W(K)$.  Since $\tau$ permutes the regular characters we get that $\tau(\alpha) \in Q_T$.
\end{proof}

The condition on $\alpha \in Q_T$ is an extremely stringent one, and an abundance of regular
characters will preclude a nontrivial $\alpha$.  We can make this statement precise:

\begin{proposition} \label{pigeonhole}
Suppose $\#\Thadm > (\# W(K) - 1) \cdot \# \Thinadm$.  Then $Q_T = \{ 1 \}$.
\end{proposition}
\begin{proof}
For $w \in W(K)$, set
$$S_w = \{\chi \in \Thadm \st  \frac{\chi}{w(\chi)} = \alpha\}.$$
Note that if $S_1$ is nonempty then we get $\alpha = 1$ immediately, so we may assume the contrary.  Then by the pigeonhole principle, there is a $w \in W(K)$ with $\# S_w > \# \Thinadm$.  Pick $\chi \in S_w$; since $\# S_w > \#\Thinadm$ there is some $\chi' \in S_w$ with $\frac{\chi}{\chi'}$ regular.  We now have
$$\frac{\chi}{w(\chi)} = \alpha = \frac{\chi'}{w(\chi')}$$
and therefore $\frac{\chi}{\chi'}$ is fixed by $w$.  Since $\frac{\chi}{\chi'}$ is regular, we must have $w = 1$ and thus
$$\alpha = \frac{\chi}{\chi} = 1.$$
\end{proof}

Recall that Frobenius acts on $X^*(T)$ via an endomorphism $F = qF_0$, where $F_0$ is an automorphism of finite order \cite[p. 82]{carter}.  So it makes sense to vary $q$: we fix $F_0$ and consider the tori dual to the $\Gal(\Fqb/\Fq)$-modules with Frobenius acting through $qF_0$.

\begin{corollary}[{c.f. \cite[Lemma 8.4.2]{carter}}]
Consider the family of tori $T_q$ with the same $F_0$.  Then for sufficiently large $q$, $Q_{T_q} = \{ 1 \}$ (regardless of the $G$ in which $T_q$ is embedded).
\end{corollary}
\begin{proof}
We will write $T$ for a general torus in the family and $r$ for the common dimension.  Note that $\hatT$ is the set of $\Fq$ points of a dual torus, also of rank $r$ over $\Fq$.  For $w \in W(K)$ with $w \ne 1$ the centralizer $\Z_{\hatT}(w)$ is a proper $F$-stable subgroup of $\hatT$, and thus $\dim(\Z_{\hatT}(w)) \le r - 1$.  By \cite[3.3.5]{carter} $\# \hatT$ is a polynomial in $q$ of degree $r$ and $\# \Z_{\hatT}(w)$ is a polynomial in $q$ of degree at most $r-1$.  Thus the ratio
$$\frac{\# \Thadm}{\# \Thinadm} = \frac{\# \hatT - \sum_{1 \ne w \in W} \# \Z_{\hatT}(w)}{\sum_{1 \ne w \in W} \# \Z_{\hatT}(w)}$$
grows without bound as $q$ does.  There are finitely many possibilities for the absolute Weyl group of $T$, so Proposition \ref{pigeonhole} gives the desired result.
\end{proof}

In computing $Q_T$ for small $q$ the following result is useful

\begin{proposition} \label{orderdiv}
If $\alpha \in Q_T$ has order $d$ and $\chi \in \Thadm$ has order $m$ then $d$ divides $m$.
\end{proposition}
\begin{proof}
There is a $w \in W(K)$ with
$$\frac{\chi}{w(\chi)} = \alpha.$$
Since $w(\chi)$ also has order $m$, raising both sides to the $m$th power  yields $\alpha^m = 1$.
\end{proof}

Finally, we note that Lemma 8 of Bushnell-Henniart \cite[p. 511]{bushnellhenniart} is equivalent to
the statement that $Q_T$ is trivial when $T$ is a $K$-minisotropic torus in $\GL_n$.

\section{Rectifiers for general reductive groups} \label{section:general_rectifiers}

Suppose that $G$ is a connected reductive group defined over a
$p$-adic field $K$.  Fix a $K$-torus $T$, contained in $G$, and suppose
that the splitting field of $T$ is $L$.
Let $\varphi : \Weil_K \rightarrow {}^L G$ be a
Langlands parameter for $G(K)$, and suppose that $\varphi$ factors
through a group of type L for $T$.  We note that every Langlands
parameter that factors through the normalizer of a maximal torus will
factor through a group of type L for some torus.
It is more
convenient however, for our purposes, to begin by fixing the torus beforehand.

As in \S\ref{section:groups_of_type_L}, one can canonically
associate to $\varphi$ a character $\xi_{\varphi}$ of $T(L)_{\Gamma}$,
the group of coinvariants of $T(L)$ with respect
to $\Gamma = \Gal(L/K)$.
Recall again the Tate cohomology sequence
$$1 \rightarrow \HT{-1}(\Gamma,T(L)) \rightarrow T(L)_{\Gamma} \xrightarrow{\Nm} T(K)
= T(L)^{\Gamma} \rightarrow \HT{0}(\Gamma,T(L)) \rightarrow 1.$$
Suppose that $\HT{0}(\Gamma, T(L)) = 0$, in which case
$T(L)_{\Gamma}$ surjects onto $T(K)$.  Let us also suppose that
$\varphi$ does not factor through a proper Levi subgroup, so that the
representations in the $L$-packet associated to $\varphi$ are
conjecturally all supercuspidal (see \cite[\S 3.5]{debackerreeder}).
If $G$ happens to be $\GL_n$, one can compute that
$\HT{0}(\Gamma, T(L)) = \HT{-1}(\Gamma, T(L)) = 0$ (see \S\ref{section:BH_compat}),
so that
$T(L)_{\Gamma} \cong T(K)$ can be identified with $\Lx$, and
$(L/K, \xi_{\varphi})$ is an admissible pair.  To construct the local Langlands
correspondence, one would then proceed (as in \S\ref{section:BH_recall}) to
attach the supercuspidal representation $\pi_{\xi_{\varphi} \cdot
  \mu_{\xi_{\varphi}}}$ to $\xi_{\varphi}$, via the construction of Bushnell and Henniart.

If $G$ were arbitrary, then in analogy to the case of $\GL_{n}(K)$, there
exist (in certain cases) constructions of supercuspidal $L$-packets
$\Lpack(\chi)$ associated to characters $\chi$ of $T(K)$ (see
\cite{debackerreeder}, \cite{kaletha}, \cite{reeder}).  However, as we
have seen, a Langlands parameter $\varphi$ does not naturally
provide a character of $T(K)$, but rather a character of
$T(L)_{\Gamma}$.

\begin{definition} \label{def:rectifier}
  Let $T$ be a $K$-minisotropic torus in $G$, that splits over an unramified
  extension $L$.  A \emph{rectifier} for $T$ is a function
  $$\bmu : (T, \xi) \mapsto \mu_{\xi}$$
  which attaches to each $(T, \xi) \in P_G(K)$ a character
  $\mu_{\xi}$ of $T(L)_{\Gamma}$ satisfying the following conditions:

\begin{enumerate}
\item The character $\mu_{\xi}$ is tamely ramified (i.e. trivial on
  $T(\PL)_{\Gamma}$)

\item The character $\xi \cdot \mu_{\xi}$ descends to $T(K)$, is regular,
and $\varphi \mapsto \Lpack(\xi_{\varphi} \cdot \mu_{\xi_{\varphi}})$
  is the local Langlands correspondence.

\item If $(T, \xi_i), i = 1,2$ are admissible pairs  such that
$\xi_1^{-1} \xi_2$ is tamely ramified, then
$\mu_{\xi_1} = \mu_{\xi_2}$.
\end{enumerate}
We say that two rectifiers $\bmu$ and $\bmu'$ for $T$ are \emph{equivalent}
if there is some $\alpha \in Q_T$ with $\mu_\xi' = \alpha \mu_\xi$ for depth zero $\xi$
and $\mu_\xi' = \mu_\xi$ for positive depth $\xi$.

\end{definition}

Since we have assumed $\HT{0}(\Gamma,T(L)) = 0$, the condition that $\xi \cdot \mu_\xi$
descends to $T(K)$ is equivalent to $\xi \cdot \mu_\xi$ vanishing on $\HT{-1}(\Gamma, T(L))$.  The
notion of equivalence is tailored for Theorem \ref{thm:unique_semisimple}; for some tori (such as the
$\SL_2(\mathbb{Q}_3)$ example at the beginning of \S \ref{Q_T}) there are
multiple equivalent rectifiers.

\begin{conjecture} \label{conj:unique_rectifier}
For $T$ as in Definition \ref{def:rectifier}, $T$ admits a unique rectifier up to equivalence.
\end{conjecture}

We note that as the local Langlands correspondence is not known in general, we must restrict
ourselves to cases where supercuspidal $L$-packets have been constructed.
Since we are in the present paper considering the situation when $T$ is unramified,
we consider those $L$-packets constructed in \cite{debackerreeder} and \cite{reeder}.
We note that not all possible admissible pairs arise in the settings
of \cite{debackerreeder} and \cite{reeder}, as can be seen in the case of $\GL_n$.
Since supercuspidal $L$-packets have not been constructed in the generality that
we need in order to prove this conjecture, we determine
the rectifier in the situation where supercuspidal $L$-packets have been
constructed.

\begin{definition}\label{def:general_pair}
Suppose $(T, \xi) \in P_G(K)$.
\begin{enumerate}
\item The \emph{depth} of $(T, \xi)$ is the integer $r$ so that $\xi$
is trivial on $T(\PL^{r+1})_{\Gamma}$ but nontrivial on
$T(\PL^{r})_{\Gamma}$
\item An admissible pair of depth $r$ is \emph{minimal}
if $\xi|_{T(\PL^{r})_{\Gamma}}$
is not fixed by any element of $W(K)$.
We denote by $\Pmin(K)$ the set
of minimal admissible pairs in $G$.
\item A \emph{weak rectifier} for $T \subset G$ is a function
\begin{align*}
\mumin : (T, \xi) \mapsto \mu_{\xi}
\end{align*}
which attaches to each $(T, \xi) \in \Pmin(K)$ a character
  $\mu_{\xi}$ of $T(L)_{\Gamma}$, satisfying conditions (1)-(3)
  of Definition \ref{def:rectifier}.
\end{enumerate}
We define equivalence of weak rectifiers as in Definition \ref{def:rectifier}.
\end{definition}

We note that this definition of minimal admissible pair generalizes
the definition of admissible pair of Bushnell and Henniart in
the case of unramified elliptic tori (see \cite[\S2.2]{bushnellhenniart1}).

\begin{theorem} \label{thm:unique_semisimple}
For $G$ semisimple and $T$ as in Definition \ref{def:rectifier}, $T$ admits a unique weak rectifier up to equivalence.
\end{theorem}

\begin{proof}
We first prove existence.
First recall that $T$ can be defined
via Galois twisting by a Weyl group element $w$.  We defined in
\S\ref{section:gross_debacker_reeder}
a Langlands parameter $\varphi_{\hat{w}} :\Weil _K \rightarrow {}^L G$ by
sending Frobenius to the canonical lift
$\tilde{w} \in \widetilde{W}$ of $\hat{w} \in \hat{W}$, and by setting
$\varphi_{\hat{w}}$ to be trivial on $I_K$.
If $G$ is semisimple,
we proved in Proposition \ref{existenceofrectifier}
that the function $$(T, \xi) \mapsto \xi_{\varphi_{\hat{w}}}^{-1}$$
satisfies condition (2) of Definition \ref{def:rectifier}..  Moreover, since $\varphi_{\hat{w}}|_{I_K} \equiv 1$,
$\xi_{\varphi_{\hat{w}}}^{-1}$ is unramified, so the function also
satisfies condition (1).  Finally, $\xi_{\varphi_{\hat{w}}}$ only depends on $T$,
and not on on $\xi$.  Therefore, condition (3) is automatically satisfied.
We may therefore set
$\mumin(T,\xi) = \xi_{\varphi_{\hat{w}}}^{-1}$.

We now prove uniqueness.
Let $\xi$ range over the set of characters of $T(L)_{\Gamma}$
such that $(T, \xi) \in \Pmin(K)$, and let
$\bmu$ and $\bmu'$ be weak rectifiers for
$T \subset G$.  By hypothesis, we have
$$\Lpack(\mu_{\xi} \cdot \xi) = \Lpack(\mu'_{\xi} \cdot \xi).$$
By \cite[\S10]{murnaghan}, there exists $w_{\xi} \in W(K)$,
depending on $\xi$, such that
$${}^{w_{\xi}} (\mu_\xi \cdot \xi) = \mu'_\xi \cdot \xi.$$
Suppose that $\xi$ has positive depth.
Restricting the equation
${}^{w_\xi} (\mu_\xi \cdot \xi) = \mu'_\xi \cdot \xi$
to $T(\PL)_{\Gamma}$, we get that ${}^{w_{\xi}} (\xi) = \xi$,
by condition (1) of Definition \ref{def:rectifier}.
Since $\xi$ is minimal, we get that $w_{\xi} = 1$, which implies
that $\mu_{\xi} = \mu'_{\xi}$.

Now suppose that $\xi$ has depth zero.
Define $\lambda$ on $T(\OL)_{\Gamma} \cong T(\OK)$ by $\lambda = ({}^{w_\xi} (\mu_\xi))^{-1} \cdot \mu'_\xi$,
which is independent of $\xi$ by condition (3). The equation ${}^{w_\xi} (\mu_\xi \cdot \xi) = \mu'_\xi \cdot \xi$
implies that $\lambda \in Q_T$.  Since $\mu_\xi \cdot \xi$ and $\mu_\xi' \cdot \xi$ descend to
$T(K)$ by condition (2) of Definition \ref{def:rectifier}, $\mu_\xi$ and $\mu_\xi'$ have the
same restriction to $\hat{H}^{-1}(\Gamma, T(L))$.  Since $G$ is semisimple we may pull $\lambda$ back to
a character on $T(L)_\Gamma$, vanishing on $\HT{-1}(\Gamma, T(L))$.  We get that $\mu'_\xi = \lambda\mu_\xi$
and thus $\bmu$ is equivalent to $\bmu'$.
\end{proof}

We note that in the
case that $G$ is semisimple, there was nothing special about
the canonical Tits group lift $\tilde{w}$ in the definition of $\varphi_{\hat{w}}$.
In fact, by the same arguments, $\xi_{\varphi_{\hat{w}}} = \xi_{\varphi'}$
for any Langlands parameter $\varphi'$ such that $\varphi'|_{I_K} \equiv 1$
and $\varphi'(\Fr) = w'$, where $w'$ is any lift of $\hat{w}$ to $\hat{N}$.
However, we will see that the canonical Tits group lift $\tilde{w}$ is forced upon us when we
consider groups that are not necessarily semisimple, such as $\GL_{n}(K)$.

Note that the condition $\HT{0}(\Gamma, T(L)) = 0$ was necessary in order to obtain a character on $T(K)$ rather
than the image of the norm map $T(L) \mapsto T(K)$.  For non-semisimple groups where $\HT{0}(\Gamma, T(L))$
is nontrivial we hope that the recipe for the central character in \cite{grossreeder} will provide an extension to all of $T(K)$.

\begin{remark}
The behavior of rectifiers under change of group is not yet clear to us.  Perhaps two embeddings of a torus into
different reductive groups with isomorphic Weyl groups can yield the same rectifier.  Moreover, given an embedding
$G \subset \GL_n$, a way of relating rectifiers for $G$ to those of $\GL_n$ would allow us to apply the results of
\cite{bushnellhenniart} in this setting.
\end{remark}

\section{Compatibility with Bushnell-Henniart} \label{section:BH_compat}

In this section, we show that our function $\mumin$
agrees with the rectifier of Bushnell/Henniart in the depth
zero setting.
Let $L/K$ be the degree $n$ unramified extension of $K$.
Let $T = \Res_{L/K}(\Gm)$.

\begin{proposition}
$\HT{0}(\Gamma, X_*(T)) = 0$.
\end{proposition}

\begin{proof}
$\Gamma$ acts on $X_*(T)$ by cyclic shift.
Therefore, $X_*(T)^{\Gamma} = \mathbb{Z}$, embedded diagonally in
$\mathbb{Z}^n = X_*(T)$.  Note that $\Nm(1,0,0,\cdots,0) = (1,1,\cdots,1)$, so
$X_*(T)^{\Gamma} \subset \Nm(X_*(T))$.
\end{proof}

\begin{proposition}
$\HT{-1}(\Gamma, X_*(T)) = 0$.
\end{proposition}

\begin{proof}
$(a_1, a_2, \cdots, a_n) \in \ker(\Nm)$ if and only if $\displaystyle\sum_{i=1}^n a_i = 0$.
It is then easy to see that $\ker(\Nm)$ is generated by $e_i - e_j$ for $i < j$, where
$e_i$ are the standard basis of $\mathbb{Z}^n$.  But $e_i - e_j = (1 - \tau)e_i$ for some
$\tau \in \Gamma$, since $\Gamma$ acts by cyclic shift.  Thus $\ker(\Nm) \subset I_{\Gamma}(X_*(T))$.
\end{proof}

The Tate cohomology exact sequence for $T$ therefore reduces to
$$1 \rightarrow T(L)_{\Gamma} \xrightarrow{\sim} T(K) \rightarrow 1$$ by
Corollary \ref{cor:cohom_tori}.

\begin{proposition}\label{prop:powers_of_lifts}
Let $\hat{w}$ be a Coxeter element of $\GL_{n}(\CC)$.  Let $\tilde{w}$ be the
canonical lift of $\hat{w}$ to $\widetilde{W}$. Then $\tilde{w}^n = (-Id)^{n-1}$, where $Id$
denotes the identity element in $\GL_{n}(\CC)$.
\end{proposition}

\begin{proof}
See \cite[\S3.1]{zaremsky}.
\end{proof}

By \cite[\S 2.4]{serre1}, $\varpi$ corresponds to $\Fr^n$ under the Artin
reciprocity map for $L$, so we obtain

\begin{corollary} \label{cor:rectifier_agreement}
$\xi_{\varphi_{\hat{w}}}^{-1}$ is unramified and
$\xi_{\varphi_{\hat{w}}}^{-1}(\varpi) = (-1)^{n-1}$.
\end{corollary}

\begin{proof}
Note that $T(L) \cong \Lx \times \Lx \times \cdots \times \Lx$, and
that $T(K) = \{(w, \sigma(w), \sigma^2(w), \cdots, \sigma^{n-1}(w)) : w \in \Lx \} \cong \Lx$, where
$\sigma$ is a generator of $\Gal(L/K)$.  A uniformizer $\varpi$ in $\Lx$
therefore corresponds to $(\varpi, \varpi, \cdots, \varpi) \in T(K)$, whose
preimage under $\Nm$ is the class of $(\varpi, 1, 1, \cdots, 1)$ in $T(L)_{\Gamma}$.
The local Langlands correspondence for tori says now that
$\xi_{\varphi_{\hat{w}}}(\varpi) = (-1)^{n-1}$, because of Proposition \ref{prop:powers_of_lifts}.
Finally, since $\varphi_{\hat{w}}|_{I_F} \equiv 1$, we get that $\xi_{\varphi_{\hat{w}}}$ is
unramified.
\end{proof}

\begin{theorem}
  If $G = \GL_{n}(K)$, the function $(T,\xi) \mapsto \xi_{\varphi_{\hat{w}}}^{-1}$ agrees with
  the rectifier of Bushnell/Henniart, when $(T,\xi) \in \Pmin(K)$ such that $\xi$ is trivial
  on $T(\PL)_{\Gamma}$.
\end{theorem}

We end this section by explaining why the Tits group lift $\tilde{w}$ is forced upon us.
Suppose we define
$\varphi' : \Weil_K \rightarrow \GL_{n}(\CC)$ by $\varphi'|_{I_K} \equiv 1$ and
$\varphi'(\Fr)$ to be a representative of an elliptic element $\hat{w}$ in $\hat{W}$.
Then \cite[p. 824]{debackerreeder} and \cite[\S6]{reeder} imply that the characteristic
polynomial of $\varphi'(\Fr)$ is $X^n - a$, for some $a \in \CCx$.  One can see that,
by analogous arguments as in Corollary \ref{cor:rectifier_agreement},
$\xi_{\varphi'}(\varpi) = a$.  By Proposition \ref{prop:BH_result1}, we are
forced to set $a = (-1)^{n-1}$.  Finally, one can show by an inductive argument that the
canonical lift $\tilde{w}$ to $\widetilde{W}$ has characteristic polynomial $X^n - (-1)^{n-1}$,
so that $\varphi'(\Fr)$ is indeed the canonical lift of $\hat{w}$ to $\widetilde{W}$.

\begin{thebibliography}{99}

\bibitem{adler}
  Jeffrey Adler,
  \emph{Refined anisotropic K-types and supercuspidal representations}, Pacific J. Math., 185 (1998), pp. 1-32.

\bibitem{adrian}
  Moshe Adrian,
  \emph{A new realization of the Langlands correspondence for $PGL(2,F)$}, Journal of Number Theory 133 (2013) 446-–474.

\bibitem{adrian1}
  Moshe Adrian
  \emph{On the Local Langlands Correspondences of DeBacker/Reeder and Reeder for $GL(\ell,F)$, where $\ell$ is prime}, Pacific Journal of Mathematics 255-2 (2012), 257--280.

\bibitem{adrianlansky}
  Moshe Adrian and Joshua Lansky,
  \emph{A real groups construction of the tame local Langlands correspondence for $PGSp(4,F)$}, preprint, arXiv:1209.6045.

\bibitem{amano}
  Kazuo Amano,
  \emph{A note on the Galois cohomology groups of algebraic tori}, Nagoya Math. J. Volume 34 (1969), 121-127.

\bibitem{atiyah-wall}
  Michael Atiyah and Charles Wall,
  \emph{Cohomology of Groups}, 84-115, in Algebraic Number Theory, John W. S. Cassels and Albrecht Frohlich (eds.).  Academic Press, London, 1967.

\bibitem{blr}
  Siegfried Bosch, Werner L\"utkebohmert, and Michel Reynaud.
  \emph{N\'eron Models}. Springer-Verlag, Berlin, 1980.

\bibitem{bushnellhenniart2}
  C. Bushnell and G. Henniart,
  \emph{The Local Langlands Conjecture for GL(2)}, A Series of Comprehensive Studies in Mathematics, Volume 335, Springer Berlin Heidelberg, 2006.

\bibitem{bushnellhenniart1}
  C. Bushnell and G. Henniart,
  \emph{The essentially tame local Langlands correspondence, II: totally ramified representations},
   Compositio Math. 141 (2005) 979-1011.

\bibitem{bushnellhenniart}
  Colin Bushnell, Guy Henniart,
  \emph{The essentially tame local Langlands correspondence, III: the general case}, Proc. Lond. Math. Soc. (3) 101 (2010), no. 2, 497–553.

\bibitem{carter}
  Roger Carter,
  \emph{Finite Groups of Lie Type: Conjugacy Classes and Complex Characters}.  Wiley \& Sons, Chichester, 1985.

\bibitem{debackerreeder}
  Stephen DeBacker and Mark Reeder,
  \emph{Depth-zero supercuspidal $L$-packets and their stability.}
  Ann. of Math. (2) 169 (2009), no. 3, 795--901.

\bibitem{geo}
  Geo Kam-Fai Tam,
  \emph{Transfer relations in essentially tame local Langlands correspondence}, Ph.D. thesis, University of Toronto, 2012.

\bibitem{grossreeder}
  B. Gross and M. Reeder,
  \emph{Arithmetic invariants of discrete Langlands parameters.}  Duke Math. Journal, 154, (2010), 431-508.

\bibitem{howe}
  Roger Howe,
  \emph{Tamely ramified supercuspidal representations of $GL_n(F)$},
   Pacific Journal of Math.  73  (1977),  437--460.

\bibitem{humphreys}
  James Humphreys,
  \emph{Linear Algebraic Groups}, Graduate Texts in Mathematics, 21.  Springer-Verlag, New York, 1975.

\bibitem{kaletha}
  Tasho Kaletha, \emph{Simple Wild L-packets}, J. Inst. Math. Jussieu (2013) 12(1), 43-75.

\bibitem{lang}
  Serge Lang, \emph{Algebraic groups over finite fields},  Amer. J. Math. (1956) 78, 555�563.

\bibitem{moy}
  Allen Moy,
  \emph{Local Constants and the Tame Langlands Correspondence},
   American Journal of Math.  108  (1986),  no. 4, 863--929.

\bibitem{moyprasad}
  \emph{A. Moy and G. Prasad},
  Unrefined minimal $K$-types for $p$-adic groups,
   Invent. Math. 116, no. 1-3, 393-408 (1994).

%\bibitem{moyprasad1}
%  Allen Moy, Gopal Prasad,
%  \emph{Jacquet functors and unrefined minimal $K$-types},
%   Comment. Math. Helv. 71 (1996), no. 1, 98--121.

\bibitem{murnaghan}
  Fiona Murnaghan,
  \emph{Parametrization of tame supercuspidal representations}, pp.439-470 in On Certain L-functions: A volume in honour of Freydoon Shahidi on the occasion of his 60th birthday , Clay Math. Proc. 13 (2011), edited by J. Arthur, J.W. Cogdell, S. Gelbart, D. Goldberg, D. Ramakrishnan.

\bibitem{reeder}
  Mark Reeder,
  \emph{Supercuspidal $L$-packets of positive depth and twisted Coxeter elements},
  J. Reine Angew. Math. 620 (2008), 1-33.

\bibitem{serre2}
  Jean-Pierre Serre,
  \emph{Algebraic Groups and Class Fields}, Graduate Texts in Mathematics, 117. Springer-Verlag, New York, 1988.

\bibitem{serre}
  Jean-Pierre Serre,
  \emph{Local Fields}, Graduate Texts in Mathematics, 67. Springer-Verlag, New York-Berlin, 1979.

\bibitem{serre1}
  Jean-Pierre Serre,
  \emph{Local Class Field Theory}, 129-162, in Algebraic Number Theory, John W. S. Cassels and Albrecht Frohlich (eds.).  Academic Press, London, 1967.

\bibitem{tits}
  Jacques Tits,
  \emph{Normalisateurs de tores. I. Groupes de Coxeter etendus}, J. Algebra, 4:96-116,1966.

\bibitem{xarles}
  Xavier Xarles, \emph{The scheme of connected components of the N�ron model of an algebraic torus},
  Journal f�r die reine und angewandte Mathematik 437 (1993): 167-180.

\bibitem{yu}
  Jiu-Kang Yu, \emph{Smooth models associated to concave functions in Bruhat-Tits theory}. Preprint, 2003.

\bibitem{zaremsky}
  Matthew Zaremsky,
  \emph{Representatives of elliptic Weyl group elements in algebraic groups}, preprint, arxiv:1109.5487.

\end{thebibliography}

\end{document}
