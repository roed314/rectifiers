\documentclass[11pt]{amsart}
\usepackage{amsmath,amscd,amssymb,latexsym, amsfonts}
\theoremstyle{plain}
\newtheorem{iassumption}{Assumption}
\newtheorem{theorem}{Theorem}[section]
\newtheorem{conjecture}[theorem]{Conjecture}
\newtheorem{proposition}[theorem]{Proposition}
\newtheorem{corollary}[theorem]{Corollary}
\newtheorem{hypothesis}[theorem]{Hypothesis}
\newtheorem{assumption}[theorem]{Assumption}
\newtheorem{lemma}[theorem]{Lemma}
\newtheorem{question}[theorem]{Question}
\newtheorem{exercise}[theorem]{Exercise}
\newtheorem{statement}[theorem]{Statement}
\newtheorem{example}[theorem]{Example}

\newcommand{\MAxxx}[1]{$\clubsuit$\footnote{#1}}
\newcommand{\DRxxx}[1]{$\spadesuit$\footnote{#1}}
\newcommand{\HT}[1]{\hat{\HH}{}^{#1}}

\theoremstyle{definition}
\newtheorem{definition}[theorem]{Definition}
\newtheorem{remark}[theorem]{Remark}

\setlength{\oddsidemargin}{0.2in}
\setlength{\evensidemargin}{0.2in}
\setlength{\textwidth}{6.1in}

\DeclareMathOperator{\Gal}{Gal}
\DeclareMathOperator{\val}{val}
\DeclareMathOperator{\HH}{H}
\DeclareMathOperator{\Ad}{Ad}
\DeclareMathOperator{\Nm}{Nm}
\DeclareMathOperator{\Hom}{Hom}
\DeclareMathOperator{\Spec}{Spec}
\DeclareMathOperator{\Res}{Res}

\newcommand{\mat}[4]{\left( \begin{array}{cc} {#1} & {#2} \\ {#3} & {#4}
\end{array} \right)}
\newcommand{\TT}{\mathcal{T}}
\newcommand{\OK}{\mathcal{O}_K}
\newcommand{\OKn}{\mathcal{O}_{K_n}}
\newcommand{\ZZ}{\mathbb{Z}}
\newcommand{\QQ}{\mathbb{Q}}
\newcommand{\Gm}{\mathbb{G}_m}

\begin{document}
\title{Rectifiers and the local Langlands Correspondence, I : the unramified case}
\author{Moshe Adrian and David Roe}
%\address{Department of Mathematics, University of Utah, Salt Lake City, UT 84112, U.S.A.}

%\email{savin@math.utah.edu}


\begin{abstract}

In this paper we propose a natural generalization of rectifiers, the
character twists that appear in the local Langlands correspondence for
$GL(n,F)$.  Our generalization is to arbitrary connected reductive
groups, with certain classes of Langlands parameters.

\end{abstract}

\maketitle

\section{Introduction}
In this paper, we propose a natural generalization of the notion of a
rectifier.  In particular, we define rectifiers for Langlands
parameters of arbitrary connected reductive groups that map into a
``group of type L''.  We then describe these rectifiers via their
associated Langlands parameters, and show that they are compatible
with the rectifiers of $GL(n,F)$ as in \cite{bushnellhenniart}.
\MAxxx{In the introduction, I should outline the case of $PGL(2,F)$
and therefore the notion of rectifier (i.e. first take Langlands
parameter, then do the natural thing to it, then ``rectify''
the situation) can be seen very clearly for $PGL(2,F)$.}

An irreducible smooth representation of the Weil group $W_F$ of $F$ is
called \emph{essentially tame} if its restriction to wild inertia is a
sum of characters.  Let us recall the classical construction of the
essentially tame local Langlands correspondence for $GL(n,F)$.  In the
essentially tame case, an irreducible representation $\varphi : W_F
\rightarrow GL(n,\mathbb{C})$ naturally provides an \emph{admissible
  pair} $(E/F, \xi)$.  Here, $E/F$ is a degree $n$ separable extension
and $\xi$ is a character (with certain conditions) of $E^{\times}$,
which one can view as a maximal elliptic torus in $GL(n,F)$.  Howe
constructs a map (see \cite{howe})
\begin{equation*}
\left\{
\begin{array}{ll}
isomorphism \ classes \ of \\
admissible \ pairs
\end{array}
\right\} \rightarrow \left\{
\begin{array}{ll}
supercuspidal \ representations \\
of \ GL(n,F)
\end{array} \right\}
\end{equation*}
$$\hspace{-.5in} (E/F, \xi) \mapsto \pi_{\xi}$$
The problem is that the obvious map, $$\varphi \mapsto \pi_{\xi},$$
the so-called ``naive correspondence'', is not the local Langlands
correspondence because $\pi_{\xi}$ has the wrong central character.
Instead, the local Langlands correspondence is given by $$\varphi
\mapsto \pi_{\xi \cdot {}_F \mu_{\xi}}$$ for some subtle finite order
character ${}_F \mu_{\xi}$ of $E^{\times}$.  The function $${}_F
\boldsymbol\mu : (E/F, \xi) \mapsto {}_F \mu_{\xi}$$ is called the
\emph{rectifier} associated to $E/F$.  Both the description of and the
intuition behind the rectifiers ${}_F \boldsymbol\mu$ have been
studied (see \cite{bushnellhenniart}, \cite{geo}, \cite{adrian}).

One may ask whether rectifiers exist for groups apart from $GL(n,F)$.
Even moreso, how does one define the notion of rectifier for
groups apart from $GL(n,F)$?  The answer can be gleamed from a recent
construction of Benedict Gross, which we briefly describe now (for
more details, see \S\ref{groupsoftypeL}).

Suppose that $G$ is a connected reductive group defined over a
$p$-adic field $F$.  Let $\phi : W_F \rightarrow {}^L G$ be a
Langlands parameter for $G(F)$, and suppose that $\phi$ factors
through the normalizer of a maximal torus. To $\phi$, one can
associate a maximal $F$-torus $T$ in $G$, the twisted $F$-torus
associated to $\phi(\Phi)$, where $\Phi$ is the geometric Frobenius
\MAxxx{geometric or arithmetic?  does it even make sense to say ``the
  geometric Frobenius of $W_F$''?} of $W_F$.  Suppose that $T$ splits
over the Galois extension $E$ of $F$, and set $\Gamma = \Gal(E/F)$.
Restricting $\phi$ to $W_E$, by the local Langlands correspondence for
tori, one can canonically associate to $\phi$ a character $\xi_{\phi}$
of $T(E)_{\Gamma}$, the group of coinvariants of $T(E)$ with respect
to $\Gamma$ (see~\S\ref{groupsoftypeL}).  Invariants and coinvariants
are related by the norm map
$$\Nm : T(E) \rightarrow T(F)$$ $$t \mapsto \displaystyle\prod_{\sigma \in \Gamma} \sigma(t)$$
in the cohomology sequence
$$1 \rightarrow \HT{-1}(\Gamma,T(E)) \rightarrow T(E)_{\Gamma} \xrightarrow{\Nm} T(F)
= T(E)^{\Gamma} \rightarrow \HT{0}(\Gamma,T(E)) \rightarrow 1,$$
where $\HT{i}$ denotes $\mathrm{i}^{\mathrm{th}}$ Tate cohomology.

Suppose that $\HT{0}(\Gamma, T(E)) = 0$, in which case
$T(E)_{\Gamma}$ surjects onto $T(F)$\MAxxx{Do we need to suppose this?
It is going to be true once we assume $G$ is semisimple and $T$
unramified, but we don't really need to suppose this now, do we?  I guess
it makes things easier to understand for the reader if we do suppose this
now, because then one can actually twist a character of the coinvariants
to get a character of $T(F)$, whereas one wouldn't be able to do this
if $\HT{0}$ wasn't trivial....hm, ask David if we should delete this
sentence or not.}.  Let us also suppose that
$\phi$ does not factor through a proper Levi subgroup, so that the
representations in the $L$-packet associated to $\phi$ are
conjecturally all supercuspidal (see \cite[\S 3.5]{debackerreeder}).
If $G$ happens to be $GL(n)$, one can compute that
$\HT{0}(\Gamma, T(E)) = \HT{-1}(\Gamma, T(E)) = 0$, so that
$T(E)_{\Gamma} \cong T(F)$ can be identified with $E^{\times}$, and
$(E/F, \xi)$ is an admissible pair.  To construct the local Langlands
correspondence, one would then proceed (as explained earlier) to
attach the supercuspidal representation $\pi_{\xi \cdot {}_F
  \mu_{\xi}}$ to $\xi$, via the construction of Howe.

If $G$ were arbitrary, then in analogy to the case of $GL(n,F)$, there
exist in certain cases constructions of supercuspidal $L$-packets
$L(\chi)$ associated to characters $\chi$ of tori $T(F)$ in $G(F)$ (see
\cite{debackerreeder}, \cite{kaletha}, \cite{reeder}).  However, as we
have seen, a Langlands parameter $\phi$ as above does not naturally
provide a character of $T(F)$, but rather a character of
$T(E)_{\Gamma}$.

\begin{definition}\label{admissibledefinition}
Let $T$ be a torus, defined over $F$, that splits over an unramified
  extension $E$ \MAxxx{Should we include the condition $F$-minisotropic
  in this definition and remove $F$-minisotropic from Definition \ref{regularpairs}?
  The reason I ask is that Bushnell/Henniart define admissible pairs as characters
  of \emph{elliptic} tori.  They don't define admissible pairs for \emph{arbitrary}
  tori}.  Suppose $\xi$ is a character of $T(E)_{\Gamma}$.
The pair $(T, \xi)$ is called \emph{admissible} if $\xi$ is not fixed
by any element of $(N(G,T)/T)(F)$. (see Lemma \ref{weylgroups}). \MAxxx{Do we need
to define a notion of isomorphic admissible pairs?  Also, should we
denote by $P_G(F)$ the set of isomorphism classes of admissible pairs?  This
would be in analogy with Bushnell/Henniart's notation $P_n(F)$ for admissible
pairs for elliptic tori in $GL(n,F)$.}
\end{definition}

\begin{definition}\label{rectifierdefinition}
  Let $T$ be a torus, defined over $F$, that splits over an unramified
  extension $E$.  A \emph{rectifier} for $T$ is a function $${}_F
  \boldsymbol\mu : (T, \xi) \mapsto {}_F \mu_{\xi}$$ which attaches,
  to each admissible pair $(T, \xi)$, a character ${}_F
  \mu_{\xi}$ of $T(E)_{\Gamma}$ satisfying the following conditions.

\begin{enumerate}
\item The character ${}_F \mu_{\xi}$ is tamely ramified (i.e. trivial on
  $T(1 + \mathfrak{p}_E)_{\Gamma}$)

\item $\xi \cdot {}_F \mu_{\xi} \in \widehat{T(F)}$, the pair $(T, \xi \cdot {}_F \mu_{\xi})$
 is regular \MAxxx{Need to define notion of regularity for characters
    of tori}, and $\phi \mapsto L(\xi_{\phi} \cdot {}_F \mu_{\xi_{\phi}})$
  is the local Langlands correspondence.

\item If $(T, \xi_i), i = 1,2$ are admissible pairs  such that
$\xi_1^{-1} \xi_2$ is tamely ramified, then
${}_F \mu_{\xi_1} = {}_F \mu_{\xi_2}$.
\end{enumerate}
\MAxxx{Gordan, during my seminar that I gave about this stuff, asked me:
What if $T$ occurs as a torus of two different groups $G$?  Do you
get the same rectifying character attached to $(T, \xi)$?  That is,
is our function ${}_F \boldsymbol\mu$ well-defined?  Because $(T,\xi)$
could occur for different groups $G_1, G_2$.  If they do, is the associated
rectifier correct in LLC for $G_1$ and $G_2$??? I'm not even sure if
I formulated this question correctly, and to be honest, I don't know
if Gordan did either, since he didn't completely understand my talk, but
we should probably think about this type of thing...}

\end{definition}

\begin{conjecture}
  Let $T$ be as in Definition \ref{rectifierdefinition}.  Then $T$
  admits a unique rectifier ${}_F \boldsymbol\mu : (T, \xi) \mapsto
  {}_F \mu_{\xi}$.
\end{conjecture}

We first note that the definition of admissible pair above generalizes
the notion of admissible pair of \cite{bushnellhenniart}.  Indeed,
if $G = GL(n)$, and $T$ is an elliptic torus in $G$ splitting over
an unramified extension $E/F$, then one can show that
$(N(G,T)/T)(F) = Gal(E/F)$.  Thus, for $\xi$ to not be fixed by any
element of $(N(G,T)/T)(F)$ is equivalent to $\xi$ not being fixed by
any subgroup of $Gal(E/F)$.  This can be seen to be equivalent to $\xi$
not coming from the norms $N_{E/L}$ for any proper subfield $L$ contained
in $E$, containing $F$.

We next note that as the local Langlands correspondence is not known in general, we must restrict
ourselves to cases where supercuspidal $L$-packets have been constructed.
Since we are in the present paper considering the situation when $T$ is unramified,
we consider those $L$-packets constructed in \cite{debackerreeder} and \cite{reeder}.
We will show that the characters ${}_F \mu_{\xi}$ that arise in this setting
happen to be unramified (i.e. trivial on $T(\mathcal{O}_E)_{\Gamma}$).  This implies,
in particular, that not all possible admissible pairs arise in the settings
of \cite{debackerreeder} and \cite{reeder}.
For example, if $G = GL(n)$, ${}_F \mu_{\xi}$ can be ramified,
even if $T$ is unramified.  We will return to this case when supercuspidal $L$-packets
are more generally constructed to allow for ramified ${}_F \mu_{\xi}$.

Since $L$-packets have not been constructed in the generality that
we need in order to prove this conjecture, we prove a theorem that determines
the rectifier in the situation where supercuspidal $L$-packets have been
constructed.\MAxxx{I really like this sentence, as it explains what we will do
in Theorem \ref{regularrectifierdefinition} before we state the theorem. I think
that is important, since it might not be easy to understand what the main
point of the theorem is.  On the other hand, this sentence sounds a bit
redundant when reading it after the previous paragraph.  Ask David what
he thinks.}

\begin{definition}\label{regularpairs}
Let $Reg(F)$ \MAxxx{If we have used the notation $P_G(F)$ to denote
the set of isomorphism classes of admissible pairs earlier, then we can maybe
write $P_G^{Reg}(F)$ instead of $Reg(F)$ here.  I only wrote $Reg(F)$ because
I wanted to change $Reg(T,\xi)$ since it didn't make sense to write
$(T,\xi) \in Reg(T,\xi)$.} denote the set of admissible pairs $(T,\xi)$ such that

(i) $T$ is $F$-minisotropic.\MAxxx{Shouldn't we say ``$T(F)$ is compact''?  Or something
similar?  The reason is that all the theorems we prove about cohomology of unramified tori
give the results that we need only when $T(F)$ is compact.  But maybe $G$ semisimple plus
$T$ being $F$-minisotropic implies that $T(F)$ is compact?  Well, actually, this seems
to be true. Because if $G$ is semisimple, then the center is finite.  But $T$ being
$F$-minisotropic is equivalent to $T(F) / Z(F)$ being compact.  But $Z(F)$ is finite, so
since $T(F) / Z(F)$ is compact, so is $T(F)$  (right???)}

(ii) $\xi$ is trivial on $T(1+\mathfrak{p}_E^{r+1})_{\Gamma}$ but nontrivial on
$T(1+\mathfrak{p}_E^{r})_{\Gamma}$ for some $r \geq 0$.

(iii) $\xi|_{T(1+\mathfrak{p}_E^{r})_{\Gamma}}$ is not fixed by any element of
$(N(G,T)/T)(F)$ (see Definition \ref{admissibledefinition})
\end{definition}
\MAxxx{IMPORTANT: These 3 conditions certainly fall into the set of admissible pairs considered in DeBacker/Reeder and Reeder.  We don't need to find conditions on admissible pairs $(T,\xi)$ that are equivallent to the 3 conditions on Langlands parameters that are considered on page 31 of Reeder's positive depth paper.  We just need our definition of $Reg(F)$ to be a subset of their pairs $(T,\xi)$ so that condition (3) of Theorem \ref{regularrectifierdefinition} makes sense. Note that David has an argument that condition (iii) is equivalent (or implied by?) condition (2) on page 31 of Reeder's positive depth paper.}

We can prove:

\begin{theorem}\label{regularrectifierdefinition}\MAxxx{David seemed a little bit
uncomfortable restating these four conditions, even though two of them are not
really restated, since they are different.  Let's think about how we can
 state this theorem better.}
Let $G$ be semisimple, and $T$ as in Definition \ref{rectifierdefinition}.
There exists a unique function $${}_F
  \boldsymbol\mu^{\mathrm{Reg}} : Reg(F) \mapsto {}_F \mu_{\xi}$$ which attaches,
  to each admissible pair $(T, \xi) \in Reg(F)$, a character ${}_F
  \mu_{\xi}$ of $T(E)_{\Gamma}$ satisfying the following conditions.

\begin{enumerate}
\item The character ${}_F \mu_{\xi}$ is unramified (i.e. trivial on
  $T(\mathcal{O}_E)_{\Gamma}$)

\item $\xi \cdot {}_F \mu_{\xi} \in \widehat{T(F)}$, the pair
$(T, \xi \cdot {}_F \mu_{\xi})$ is regular, and
$\phi \mapsto L(\xi_{\phi} \cdot {}_F \mu_{\xi_{\phi}})$
  agrees with \cite{debackerreeder} and \cite{reeder}

\item If $(T, \xi_i), i = 1,2$ are admissible pairs  such that
$\xi_1^{-1} \xi_2$ is unramified, then
${}_F \mu_{\xi_1} = {}_F \mu_{\xi_2}$.
\end{enumerate}
\end{theorem}

Once phrased in this manner, uniqueness is a straightforward matter
(see \S\ref{}) \MAxxx{Fix when we can.}. \MAxxx{Wait.  Can we really prove uniqueness?
I don't know if we have an argument.  For example, there might be different
characters of a torus $T(F)$ that give rise to the same supercuspidal $L$-packet.
Is this true?}

The existence of ${}_F  \boldsymbol\mu^{\mathrm{Reg}}$
will be proven by interpreting the characters ${}_F \mu_{\xi}$
in terms of a canonical Langlands parameter, as follows.  Let
$G$ be an arbitrary unramified connected reductive group.  If
$T$ is an unramified torus in $G$, then $T$ can be defined
via Galois twisting by a Weyl group element $w$.  We will define
a Langlands parameter $\phi_o :W _F \rightarrow {}^L G$ by
sending Frobenius to a canonical \MAxxx{Canonical?  Is the Tits group
lift canonical?} lift of $w$, using the Tits group, and by setting
$\phi_o$ to be trivial on $I_F$.
If $G$ is semisimple,
we will prove that the function $$(T, \xi) \mapsto \xi_{\phi_o}$$
\MAxxx{Should this be $(T, \xi) \mapsto \xi_{\phi_o}^{-1}$????  Since the inverse is what we twist by to get a character of $T(F)$ (see Section \ref{mainresults})}
satisfies conditions (1)-(4), and we may therefore set
${}_F \boldsymbol\mu^{\mathrm{Reg}}(T,\xi) = \xi_{\phi_o}$.
Moreover, we prove compatibility with Bushnell/Henniart:

\begin{theorem}
  If $G = GL(n,F)$, the function $(T,\xi) \mapsto \xi_{\phi_o}$ agrees with
  the rectifier of Bushnell/Henniart, when $(T,\xi) \in Reg(F)$.  Moreover, $\phi_o$ is the
  unique \MAxxx{up to a sign, or something like that?} Langlands
  parameter such that $\xi_{\phi_o} = {}_F \mu_{\xi}$.
\end{theorem}

We would like to note that in the case that $\HT{0}(\Gamma, T(E))
\neq 0$, the situation seems more difficult because twisting $\xi$
by a character of $T(E)_{\Gamma}$
can only result in a character of the image
of $T(E)_{\Gamma}$ under the norm map.  However, one might be
able to remedy this with a prediction of central character, as in
\cite{grossreeder}, for example.

Acknowledgements: Part of this paper was heavily influenced by
conversations with Gordan Savin.  We wish to thank him for these
conversations.  We would also like to thank Jeffrey Adams and Geo Kam
Fai for helpful conversations as well.

\section{Notation and Definitions}

Throughout, $k$ will denote a finite field, and
$k_n$ will denote the degree $n$ extension of
$k$.  $K$ will denote a nonarchimedean local field of
charcteristic zero, and $k$ will be its residue field.  Fix
a uniformizer $\varpi$ of $K$.  Let $K^u$ denote the maximal
unramified extension of $K$, and fix a valuation $\val : (K^u)^\times
\rightarrow \mathbb{Z}$ normalized so that $\val(\varpi) = 1$.
Write $K_n$ for the unramified extension of $K$ of degree $n$, $k_n$ for
the degree $n$ extension of $k$,
and set $\Gamma_n = \Gal(K_n/K) = \Gal(k_n/k)$.

\section{Preliminaries}

If $S$ is a torus that is defined over an arbitrary field $F$ and
$E/F$ is a Galois extension, then we have the norm map
$$\Nm_{E/F} : S(E) \rightarrow S(F)$$ $$\ \ \ \ \ \ \ \ \ \ \ \ \ \ \ \ \ s \mapsto
\prod_{\sigma \in \Gal(E/F)} \sigma(s)$$\MAxxx{Sometimes we write $\Nm_{E/F}$, sometimes we
write $\Nm$.  Be consistent.}

\section{Galois cohomology over finite fields}

The following theorem, due to Lang \cite{lang}, underpins the facts in
Section \ref{padic tori} on tori over $p$-adic fields.
Let $\mathbb{G}$ be a commutative connected algebraic group over a
finite field $k$, and suppose $\mathbb{G}$ splits over $k_n$.

\begin{theorem} \label{langs thm}
$\HT{i}(\Gamma_n, G(k_n)) = 0$ for all $i$.
\end{theorem}
\begin{proof}
Since $\Gamma_n$ is cyclic,
$\HT{i}(\Gamma_n, G(k_n)) \cong \HT{i+2}(\Gamma_n, G(k_n))$,
so it suffices to prove the result for $i=1$ and $i=2$, which is done
in \cite[\SS VI.6]{serre2}.
\end{proof}

\section{Tori over $p$-adic fields} \label{padic tori}

Let $T$ be a torus defined over $K$.  $T$ will split over an extension
$L$ of $K$ if and only if $\Gal(\bar{K}/L)$ acts trivially on $X^*(T)$.
We may thus define \emph{the} splitting field $L$ of $T$ as the
minimal extension of $K$ splitting $T$; note that $L$ is necessarily
Galois over $K$.  Let $K_n$ be the maximal unramified subextension
of $L/K$, and set $I = \Gal(L/K_n)$.

Let $\TT$ be the N\'eron model of $T$, a canonical model of $T$
over $\OK$ \cite[Ch. 10]{blr}.  As a consequence of the N\'eron mapping
property, we may identify $\TT(\OK)$ with $T(K)$.  The connected
component of the identity, $\TT^\circ$, cuts out a subgroup
$T(K)_0 = \TT^\circ(\OK)$ of $T(K)$; we also write $T(K_n)_0$ for
$\TT^\circ(\OKn)$.

In fact, this subgroup of $T(K)$ is the first in a decreasing filtration.
Moy and Prasad \cite{moyprasad} \MAxxx{Is this the correct reference?}
define one such filtration by
embedding $T$ into an induced torus and defining the filtration of
$\Res_{L/K} \Gm$ in terms of the valuation on $L$.  Yu \cite[\S 5]{yu}
describes a different filtration, agreeing with that of Moy and Prasad
in the case of tame tori but with nicer features in the presence of wild
ramification.  Let $\TT_r$ be the integral model of $T$ defined in Yu's
minimal congruent filtration, and let $\{T(K)_r\}_{r \ge 0}$ and
$\{T(K_n)_{r \ge 0}$ be the corresponding filtrations of $T(K)$ and
$T(K_n)$.

Let $C$ be the scheme of
connected components of $\TT$, which we may identify with the
components of $\TT \times \Spec(k)$ since $T = \TT \times \Spec(K)$
is connected.  The structure of $C$ is described by Xarles:

\begin{proposition}[{\cite[Cor. 2.12]{xarles}}]
There is an exact sequence of $\Gamma_n$-modules
$$0 \rightarrow \Hom_{\ZZ}(\HH^1(I, X^*(T)), \QQ/\ZZ) \rightarrow
C \rightarrow \Hom_{\ZZ}(X^*(T)^I, \ZZ) \rightarrow 0.$$
\end{proposition}

\begin{corollary}[{\cite[Thm. 1.1]{xarles}}] \label{unram comps}
If $T$ is unramified, then $C \cong X_*(T)$.
\end{corollary}

Using our filtration of $T(K_n)$, we may relate the cohomology of $T(K_n)$
with that of $C$.

\begin{proposition}\label{vanishingtateonintegralpoints}
$\HT{i}(\Gamma_n, T(K_n)_0) = 0$ for all $i$.
\end{proposition}
\begin{proof}
Note that
$$T(K_n)_0 = \lim_{\leftarrow r} T(K_n)_0 / T(K_n)_r.$$
So by a result of Serre \cite[Lem. 3]{serre1}, it suffices to prove that
$\HT{i}(\Gamma_n, T(K_n)_r / T(K_n)_{r+}) = 0$ for all $i$.  But $T(K_n)_r / T(K_n)_{r+}$
is connected \cite[Prop. 5.2]{yu} and thus has trivial cohomology by
Theorem \ref{langs thm}.
\end{proof}

\begin{corollary}
$\HT{i}(\Gamma_n, T(K_n)) \cong \HT{i}(\Gamma_n, C)$.
\end{corollary}

\begin{proof}
This follows from the long exact sequence in cohomology associated to the sequence
$$0 \rightarrow \TT^0 \rightarrow \TT \rightarrow C \rightarrow 0.$$
\end{proof}

\begin{corollary}\label{reductiontori}
If $T$ is unramified, then $\HT{i}(\Gamma_n, T(K_n)) \cong \HT{i}(\Gamma_n, X_*(T))$
for all $i$.
\end{corollary}

\begin{proof}
This follows from the previous corollary together with Corollary \ref{unram comps}.
\end{proof}

\begin{corollary}\label{vanishingH0}
If $T$ is unramified and anisotropic, then $\HT{0}(\Gamma_n, T(K_n)) = 0$.
\end{corollary}

\begin{proof}
Since $T$ is anisotropic $\HH^0(\Gamma_n, T(K_n)) \cong X_*(T)^{\Gamma_n} = 0$.
\end{proof}

\section{Groups of type L}\label{groupsoftypeL}
We now review the theory of ``groups of type L'' due to Benedict
Gross.  Let $F$ be a field, $F^{\mathrm s}$ a separable closure, and
$T$ a torus defined over $F$ that splits over an extension $E \subset
F^s$.  Let $\Gamma = \Gal(E/F)$.  Let $X^*(T)$ be the character module
of $T$ and $X_*(T)$ the cocharacter module of $T$.  Define
$\hat{T} = X^*(T) \otimes \mathbb{C}^\times$.  The group $\Gamma$ acts on
$\hat{T}$ via its action on $X^*(T)$.

\begin{definition}
A \emph{group of type L} is a group extension of $\Gamma$ by $\hat{T}$.
\end{definition}

Let $D$ be such a group.  Then we have an exact sequence
$$1 \rightarrow \hat{T} \rightarrow D \rightarrow \Gamma \rightarrow 1$$

We now describe how, given a Langlands parameter
$$\phi : W_F \rightarrow D,$$
where $D$ is a group of type L, we can naturally attach a character of
$T(E)_{\Gamma} := T(E) / I_{\Gamma}(T(E))$, where
$I_{\Gamma}(T(E)) = \{(1 - \sigma)t \ : t \in T(E), \sigma \in \Gamma \}$.
Restricting $\phi$ to $W_E$ we get a homomorphism
$$\phi|_{W_E} : W_E \rightarrow \hat{T}.$$
By the Langlands correspondence for tori, this gives us a character
$\chi : T(E) \rightarrow \mathbb{C}^\times$.  Since $\phi|_{W_E}$ extends
to $\phi$, one can see that
$$\chi(t^{\sigma}) = \chi(t)\ \mbox{for all $\sigma \in \Gamma$.}$$
Therefore, $\chi(t^{\sigma - 1}) = 1$ for all $\sigma \in \Gamma$.
Thus, $\chi$ is trivial on the augmentation ideal $I_{\Gamma}(T(E))$
and gives $$\chi : T(E)_\Gamma \rightarrow \mathbb{C}^\times$$ Invariants
and coinvariants are related by the norm map
$$\Nm : T(E) \rightarrow T(F)$$ $$t \mapsto \displaystyle\prod_{\sigma \in \Gamma} \sigma(t)$$
in the Tate cohomology sequence
$$1 \rightarrow \HT{-1}(\Gamma,T(E)) \rightarrow T(E)_{\Gamma} \xrightarrow{\Nm} T(F)
  = T(E)^{\Gamma} \rightarrow \HT{0}(\Gamma,T(E)) \rightarrow 1$$
(note that the norm map $\Nm$ factors to $T(E)_{\Gamma}$).
We have thus constructed a character $\chi$ of $T(E)_{\Gamma}$ from a
Langlands parameter $\phi$. We note that $T(E)_{\Gamma}$ is a cover of
$\Nm(T(E)_{\Gamma})$, which is a subgroup of $T(F)$.  It is sometimes
the case that $\Nm$ is surjective, in which case $\chi$ is then a
character of $T(E)_{\Gamma}$, which is a cover of $T(F)$.

We now give meaning to the definition of \emph{admissible pair} in the introduction.

\begin{lemma}\label{weylgroups}
Let $G$ be a connected reductive $F$-group and let $T$ be a maximal
$F$-torus of $G$.  Let $E$ be the splitting field of $T$, and set
$\Gamma = \Gal(E/F)$.\MAxxx{Umm.....does $E/F$ need to be Galois?  I asked 
the same question when introducing groups of type $L$ in the introduction of this paper...
the torus $T$ in the introduction (when introducing groups of type $L$) could
split over a non-Galois extension.}
\begin{enumerate}
\item $N(G(E), T(F)) / T(E) \cong (N(G,T)/T)(F)$.
\item The standard action of $N(G(E),T(E)) / T(E)$ on $T(E)$ determines
well-defined actions of $N(G(E), T(E))^{\Gamma} / T(F)$ and $(N(G,T)/T)(F)$
on $T(E)$, which factor naturally to actions on $T(E)_{\Gamma}$.
\end{enumerate}
\end{lemma}

\proof
See \cite[Lemma 9.1]{adrianlansky}.
\qed

We will need the following structural result about Langlands
parameters mapping to groups of type $L$ later.  First let $E,F,T,\hat{T}, \Gamma$
be as above, with $E/F$ unramified.  Let $\varphi_1 : W_F \rightarrow D$
be a Langlands parameter mapping to a group of type $L$.  By the theory
of groups of type $L$, $\varphi_1$
gives rise to a character $\chi_1$ of $T(E)_{\Gamma}$.  Let $\varphi_2
: W_F \rightarrow D$ be defined by $\varphi_2|_{I_F} \equiv \varphi_1|_{I_F}$ and
$\varphi_2(\Phi_F) = t \varphi_1(\Phi_F)$ for some $t \in \hat{T}$.  Then likewise
$\varphi_2$ gives rise to a character $\chi_2$ of $T(E)_{\Gamma}$.

\begin{lemma}\label{toralmodification}
$\chi_1$ and $\chi_2$ have the same restriction to $\HT{-1}(\Gamma, T(E))$.
\end{lemma}

\proof
Let $\sigma$ be a generator of $\Gamma$.  First note that
$\varphi_1(\Phi_E) = \varphi_1(\Phi_F)^r$, where $r$ is the
order of $\Gamma$.
Set $n = \varphi_1(\Phi_F)$.
Note that $n^r \in \hat{T}$.

By Corollary \ref{reductiontori}, we know that
$\HT{-1}(\Gamma, T(E)) = \HT{-1}(\Gamma, X_*(T))$,
since $E/F$ is unramified.  Recall that
$\HT{-1}(\Gamma, X_*(T)) = \ker(\Nm : X_*(T) \rightarrow X_*(T)) / I_{\Gamma}(X_*(T))$.

Let $\lambda \in \ker(\Nm : X_*(T) \rightarrow X_*(T))$.  Note that
$\lambda \in X^*(\hat{T})$.
Write $n^r = \mu \otimes z$, for some $\mu \in X_*(\hat{T})$,
$z \in \mathbb{C}^\times$.  Then $\chi_1(\lambda) = z^{<\mu, \lambda>}$
(where $< , >$ denotes the canonical pairing
on $X_*(\hat{T}) \times X^*(\hat{T})$ ).

We now note that
$$(tn)^r = t ntn^{-1} n^2 t n^{-2} \cdots n^{r-1} t n^{1-r} n^r = t w(t) w^2(t) \cdots w^{r-1}(t),$$
where $w$ is the class of $n$ in the Weyl group of $\hat{T}$.  If we write
$t = \mu' \otimes z'$ for $\mu' \in X_*(\hat{T}), z' \in \mathbb{C}^{\times}$, 
then $$t w(t) w^2(t) \cdots w^{r-1}(t) = \Nm(\mu' ) \otimes z'.$$
But now one see that $\chi_2(\lambda) = z^{<\mu, \lambda>} (z' )^{<\Nm(\mu'), \lambda>}$, 
and
$$<\Nm(\mu'), \lambda> = \displaystyle\sum_{i=0}^{r-1} < \sigma^i(\mu'), \lambda >.$$
Since $< , >$ is Galois equivariant, 
$$\displaystyle\sum_{i=0}^{r-1} < \sigma^i(\mu'), \lambda > = 
\displaystyle\sum_{i=0}^{r-1} <\mu', \sigma^{-i}(\lambda)> = <\mu', \Nm(\lambda)>.$$
But $\lambda \in \ker(\Nm : X^*(\hat{T}) \rightarrow X^*(\hat{T}))$, so
$\chi_2(\lambda) = z^{<\mu, \lambda>} = \chi_1(\lambda)$.
\qed

\section{The relationship between the Gross construction and the DeBacker--Reeder construction}\label{grossdebackerreeder}

We consider here an RSELP $\phi$ for any unramified connected
reductive group G.  Let $T$ be as in \S\ref{groupsoftypeL}.  Let $E$,
$\chi$, etc.~be as in~\S\ref{groupsoftypeL}.  Let $w$ be the Weyl
group element associated to $\phi$, and set $\sigma = w \theta\in {\rm Aut}(T)$.
Let $\chi_{\phi}$ be the character of $T(F) = T^{\Phi_{\sigma}}$ that
DeBacker and Reeder attach to $\phi$.

We have the exact sequence

$$1 \rightarrow \HT{-1}(\Gamma, T(E)) \rightarrow T(E)_{\Gamma} \rightarrow T(F)
  \rightarrow \HT{0}(\Gamma, T(E)) \rightarrow 1$$

where $\Gamma = \Gal(E/F)$.  Recall that $\phi$ canonically gives
rise to a character $\chi$ of $T(E)_{\Gamma}$, by local Langlands
for tori (see section \ref{groupsoftypeL}).  Note that the above
exact sequence restricts to an exact sequence

$$1 \rightarrow \HT{-1}(\Gamma, T(\mathcal{O}_E)) \rightarrow T(\mathcal{O}_E)_{\Gamma}
  \rightarrow T(\mathcal{O}_F) \rightarrow \HT{0}(\Gamma, T(\mathcal{O}_E)) \rightarrow 1$$

Moreover, by Proposition \ref{vanishingtateonintegralpoints}, we have
$\HT{-1}(\Gamma, T(\mathcal{O}_E)) = \HT{0}(\Gamma, T(\mathcal{O}_E)) = 1$.
Therefore, the map
$$T(\mathcal{O}_E)_{\Gamma} \xrightarrow{\Nm} T(\mathcal{O}_F)$$
is an isomorphism, and we may view
$\chi|_{T(\mathcal{O}_E)_{\Gamma}}$ as a character of
$T(\mathcal{O}_F)$ via this isomorphism.  Now, since $\chi$ was
obtained from $\phi|_{W_E}$ via the local Langlands correspondence, one can see
from the constructions in \cite{debackerreeder} and \cite{reeder} that
$\chi_{\phi} \circ \Nm = \chi$ on $T(\mathcal{O}_E)_{\Gamma}$ (one can see
\cite{adrianlansky} for more details on the constructions of
\cite{debackerreeder} and \cite{reeder} and their relationship to groups of
type $L$).
We have therefore proven the following lemma.

\begin{lemma}\label{grossanddebackerreedercompatibility}
  The restriction to $T(\mathcal{O}_E)_{\Gamma} \xrightarrow{\sim} T(\mathcal{O}_F)$
  of the genuine character arising from the Gross construction
  coincides with the character of $T(\mathcal{O}_F)$ that is
  constructed from $\phi$ via the construction of DeBacker--Reeder
  construction.
\end{lemma}

\section{Rectifiers and DeBacker/Reeder}\label{mainresults}

Recall that we have fixed a splitting $(\hat{T}, \hat{B}, \{x_{\alpha} \})$
for the dual group $\hat{G}$ and that $\hat{N} = N_{\hat{G}}(\hat{T})$.
For each simple root $\alpha$, let $\phi_{\alpha} : SL(2) \rightarrow \hat{G}$
be defined by $\phi_{\alpha}(\mathrm{diag}(z,1/z)) = \alpha^{\vee}(z)$
and $\phi_{\alpha}\mat{1}{1}{0}{1} = x_{\alpha}$. Let
$\sigma_{\alpha} = \phi_{\alpha}\mat{0}{1}{-1}{0}$.

\begin{definition}
  The Tits group $\widetilde{W}$ is the subgroup of $\hat{N}$
  generated by $\{\sigma_{\alpha} \}$ for $\alpha$ simple.
\end{definition}

For each simple root $\alpha$, let $m_{\alpha} = \sigma_{\alpha}^2 = \alpha^{\vee}(-1)$.
Let $\hat{T}_2$ be the subgroup of $\hat{T}$ generated by the $m_{\alpha}$.

\begin{theorem}{(Tits \cite{tits})}
\begin{enumerate}

\item The kernel of the natural map $\widetilde{W_o} \rightarrow \hat{W}_o$
  is $\hat{T}_2$,
\item The elements $\sigma_{\alpha}$ satisfy the braid relations,
\item There is a canonical lifting of $\hat{W}_o$ to a subset of
  $\widetilde{W}$: take a reduced expression $w = s_{\alpha_1} \cdots s_{\alpha_n}$,
  and let $\tilde{w} = \sigma_{\alpha_1} ... \sigma_{\alpha_n}$.
\end{enumerate}
\end{theorem}

Let $\phi$ be a RSELP.  Then $\phi(I_F) \subset \hat{T}$ and
$\phi(\Phi) = \hat{\theta} f$ for some $f \in \hat{N}$.  Let $\hat{w}$
be the image of $f$ in $\hat{W}_o$, and let $w$ be the element of
$W_o$ corresponding to $\hat{w}$.  Write $\hat{w} = s_{\alpha_1} \cdots s_{\alpha_n}$
as a product of simple reflections, and let
$\tilde{w} = \sigma_{\alpha_1} \cdots \sigma_{\alpha_n}$ be the \MAxxx{the?  Is there
a canonical lift?}
canonical lift of $\hat{w}$ to $\widetilde{W}$.

\begin{definition}
Given $\phi$, we define a homomorphism $\phi_o : W_F \rightarrow {}^L G$ by
\begin{enumerate}
\item $\phi_o|_{I_F} \equiv 1$
\item $\phi_o(\Phi) = \hat{\theta} \tilde{w}$
\end{enumerate}
\end{definition}

By the theory in \cite{debackerreeder}, $\phi$ and $\phi_o$ give
rise to a character $\chi_{\phi}$ and $\chi_{\phi_o}$ of
$T(F) = T^{\Phi_{\sigma}}$, respectively.  By the theory in section
\ref{groupsoftypeL}, $\phi$ and $\phi_o$ gives rise to characters
$\xi_{\phi}$ and $\xi_{\phi_o}$ of $T(E)_{\Gamma}$, respectively.

\begin{proposition}
If $G$ is semisimple, then $\xi_{\phi} \otimes \xi_{\phi_o}^{-1} = \chi_{\phi}$.
\end{proposition}

\proof
Since $G$ is semisimple, $T(F)$ is compact.  In particular,
$\HT{0}(\Gamma, T(E)) = 0$ by Corollary \ref{vanishingH0},
so we have the following exact sequence:
$$1 \rightarrow \HT{-1}(\Gamma, T(E)) \rightarrow T(E)_{\Gamma} \rightarrow T(F) \rightarrow 1.$$
Note that $T(F) = T(\mathcal{O}_F)$, and so in particular
$T(\mathcal{O}_E)_{\Gamma}$ surjects onto $T(F)$ via the norm map
$\Nm$.  This, together with the fact that $\HT{-1}(\Gamma, T(E)) \neq 0$,
gives us that $\HT{-1}(\Gamma,T(E))$ and
$T(\mathcal{O}_E)_{\Gamma}$ together generate $T(E)_{\Gamma}$.

Since $\phi_o|_{I_F} \equiv 1$, $\xi_{\phi_o}$ is trivial on
$T(\mathcal{O}_E)_{\Gamma}$.  By lemma
\ref{grossanddebackerreedercompatibility}, we have that
$(\xi_{\phi} \otimes \xi_{\phi_o}^{-1})|_{T(\mathcal{O}_E)_{\Gamma}} = \chi_{\phi}|_{T(\mathcal{O}_F)}$.
Therefore, $\xi_{\phi} \otimes \xi_{\phi_o}^{-1}$ is correct on
$T(\mathcal{O}_E)_{\Gamma}$.  Moreover, by Lemma
\ref{toralmodification}, we also have that $\xi_{\phi}$ and $\xi_{\phi_o}$ have
the same restriction to $\HT{-1}(\Gamma, T(E))$.  Therefore,
$\xi_{\phi} \otimes \xi_{\phi_o}^{-1}$ is a character of $T(F)$.  Since
$\xi_{\phi} \otimes \xi_{\phi_o}^{-1}$ is correct on $T(\mathcal{O}_E)_{\Gamma}$, and
since $T(\mathcal{O}_E)_{\Gamma} \xrightarrow{\Nm} T(F)$ is surjective,
we are done.
\qed

In the case that $G$ is semisimple, there was nothing special about
$\phi_o$.  In fact, by the same arguments, $\xi_{\phi_o} = \xi_{\phi'}$
for any Langlands parameter $\phi'$ such that $\phi'|_{I_F} \equiv 1$
and $\phi'(\Phi) = w'$, where $w'$ was any lift of $\hat{w}$ to $N$.
However, we will see that the \MAxxx{the Tits group lift or a Tits group
 lift?} Tits group lift is forced on us when we
consider groups that are not necessarily semisimple, such as $GL(n,F)$.

\section{Rectifiers for $GL(n,F)$}\label{GL(n)}

We first recall the structure of the rectifiers for $GL(n,F)$ in our
setting (see \cite{bushnellhenniart}).

\begin{proposition}\label{rectifierofbushnellhenniart}
  Suppose that $(E/F, \xi)$ is an admissible pair such that $E/F$ is
  unramified of degree $n$ and $\xi|_{\mathcal{O}_E} \equiv 1$.
  Then ${}_F \mu_{\xi}$ is unramified and
  ${}_F \mu_{\xi}(\varpi_E) = (-1)^{n-1}$.
\end{proposition}

\proof
See \cite[Proposition 21]{bushnellhenniart}.
\qed

\begin{proposition}
$\HT{0}(\Gamma, X_*(T)) = 0$.
\end{proposition}

\proof
As $T = Res_{E/F}(GL_1)$, $\Gamma$ acts on $X_*(T)$ by cyclic shift.
Therefore, $X_*(T)^{\Gamma} = \mathbb{Z}$, embedded diagonally in
$\mathbb{Z}^n$.  Note that $\Nm(1,0,0,\cdots,0) = (1,1,\cdots,1)$, so
$X_*(T)^{\Gamma} \subset \Nm(X_*(T))$.
\qed

\begin{proposition}
$\HT{-1}(\Gamma, X_*(T)) = 0$.
\end{proposition}

\proof
$(a_1, a_2, \cdots, a_n) \in \ker(\Nm)$ if and only if $\displaystyle\sum_{i=1}^n a_i = 0$.
It is then easy to see that $\ker(\Nm)$ is generated by $e_i - e_j$ for $i < j$, where
$e_i$ is the standard basis of $\mathbb{Z}^n$.  But $e_i - e_j = (1-\sigma^{j-i}) e_i$,
where $\sigma$ is the element of $\Gamma$ that cyclically shifts to the right by
one space.  Thus $\ker(\Nm) \subset I_{\Gamma}(X_*(T))$.
\qed

The Tate cohomology exact sequence therefore reduces to
$$1 \rightarrow T(E)_{\Gamma} \xrightarrow{\sim} T(F) \rightarrow 1$$
\begin{proposition}\label{powersoflifts}
Let $\hat{w} = s_{\alpha_1} s_{\alpha_2} \cdots s_{\alpha_n}$ where
$s_{\alpha_i}$ are the simple reflections of $GL(n,\mathbb{C})$,
for $i = 1, 2, \cdots, n$.  Then $\tilde{w}^n = (-Id)^{n-1}$, where $Id$
denotes the identity element in $GL(n,\mathbb{C})$.
\end{proposition}

\proof
See \cite[\S3.1]{zaremsky}.
\qed

Since $\Phi_E = \Phi_F^n$ and $\Phi_E$ corresponds to $\varpi_E$ under local
class field theory, we obtain

\begin{corollary}\label{agreemenetofrectifiers}
$\xi_{\phi_o} = {}_F \mu_{\xi}$.
\end{corollary}

\proof
Note that $T(E) = E^{\times} \times E^{\times} \times \cdots \times E^{\times}$, and
that $T(F) = (w, \sigma(w), \sigma^2(w), \cdots, \sigma^{n-1}(w))$, where
$Gal(E/F) = \langle \sigma \rangle$.  A uniformizer $\varpi_E$ in $E^{\times}$
therefore corresponds to $(\varpi_E, \varpi_E, \cdots, \varpi_E) \in T(F)$, whose
preimage under $\Nm$ is the class of $(\varpi_E, 1, 1, \cdots, 1) \in T(E)_{\Gamma}$.
The local Langlands correpondence for tori says now that
$\xi_{\phi_o}(\varpi_E) = (-1)^{n-1}$, because of Proposition \ref{powersoflifts}.
\qed

We end this section by explaining why the Tits group lift of $w$ is forced upon us.
We may look no further than $G = GL(2)$.  Suppose we define
$\phi_o' : W_F \rightarrow GL(2,\mathbb{C})$ by $\phi_o'|_{I_F} \equiv 1$ and

\[
\phi_o'(\Phi_F) = \left( \begin{array}{cc}
0 & a\\
b & 0
\end{array} \right)
\]

We may conjugate $\phi_o'$ so that

\[
\phi_o'(\Phi_F) = \left( \begin{array}{cc}
0 & 1\\
b & 0
\end{array} \right)
\]

for some $b \in \mathbb{C}^{\times}$.  One can see that, by analogous arguments
as in Corollary \ref{agreementofrectifiers}
$\xi_{\phi_o'}(\varpi_E) = b$.  By Proposition \ref{rectifierofbushnellhenniart}, we are
forced to have $b = -1$.

\begin{thebibliography}{9}

\bibitem{adrian}
  Moshe Adrian,
  \emph{A new realization of the Langlands correspondence for $PGL(2,F)$}, Journal of Number Theory 133 (2013) 446-–474.

\bibitem{adrian1}
  Moshe Adrian
  \emph{On the Local Langlands Correspondences of DeBacker/Reeder and Reeder for $GL(\ell,F)$, where $\ell$ is prime}, Pacific Journal of Mathematics 255-2 (2012), 257--280.

\bibitem{adrianlansky}
  Moshe Adrian and Joshua Lansky,
  \emph{A real groups construction of the tame local Langlands correspondence for $PGSp(4,F)$}, preprint, arXiv:1209.6045.

\bibitem{amano}
  Kazuo Amano,
  \emph{A note on the Galois cohomology groups of algebraic tori}, Nagoya Math. J. Volume 34 (1969), 121-127.

\bibitem{blr}
  Siegfried Bosch, Werner L\"utkebohmert, and Michel Reynaud. \emph{N\'eron Models}. Springer-Verlag, Berlin, 1980.

\bibitem{bushnellhenniart}
  Colin Bushnell, Guy Henniart,
  \emph{The essentially tame local Langlands correspondence, III: the general case}, Proc. Lond. Math. Soc. (3) 101 (2010), no. 2, 497–553.

\bibitem{debackerreeder}
  Stephen DeBacker and Mark Reeder,
  \emph{Depth-zero supercuspidal $L$-packets and their stability.}
  Ann. of Math. (2) 169 (2009), no. 3, 795--901.

\bibitem{geo}
  Geo Kam-Fai Tam,
  \emph{Transfer relations in essentially tame local Langlands correspondence}, Ph.D. thesis, University of Toronto, 2012.

\bibitem{grossreeder}
  B. Gross and M. Reeder,
  \emph{Arithmetic invariants of discrete Langlands parameters.}  Duke Math. Journal, 154, (2010), 431-508.

\bibitem{howe}
  Roger Howe,
  \emph{Tamely ramified supercuspidal representations of $GL_n(F)$},
   Pacific Journal of Math.  73  (1977),  437--460.

\bibitem{kaletha}
  Tasho Kaletha, \emph{Simple Wild L-packets}, J. Inst. Math. Jussieu (2013) 12(1), 43-75.

\bibitem{lang}
  Serge Lang, \emph{Algebraic groups over finite fields},  Amer. J. Math. (1956) 78, 555�563.

\bibitem{moyprasad}
  \emph{A. Moy and G. Prasad},
  Unrefined minimal $K$-types for $p$-adic groups,
   Invent. Math. 116, no. 1-3, 393-408 (1994).

\bibitem{moyprasad1}
  Allen Moy, Gopal Prasad,
  \emph{Jacquet functors and unrefined minimal $K$-types},
   Comment. Math. Helv. 71 (1996), no. 1, 98--121.

\bibitem{reeder}
  Mark Reeder,
  \emph{Supercuspidal $L$-packets of positive depth and twisted Coxeter elements},
  J. Reine Angew. Math. 620 (2008), 1-33.

\bibitem{serre2}
  Jean-Pierre Serre,
  \emph{Algebraic Groups and Class Fields}, Graduate Texts in Mathematics, 117. Springer-Verlag, New York, 1988.

\bibitem{serre}
  Jean-Pierre Serre,
  \emph{Local Fields}, Graduate Texts in Mathematics, 67. Springer-Verlag, New York-Berlin, 1979.

\bibitem{serre1}
  Jean-Pierre Serre,
  \emph{Local Class Field Theory}, 129-162, in Algebraic Number Theory, John W. S. Cassels and Albrecht Frohlich (eds.).  Academic Press, London, 1967.

\bibitem{tits}
  Jacques Tits,
  \emph{Normalisateurs de tores. I. Groupes de Coxeter etendus}, J. Algebra, 4:96-116,1966.

\bibitem{xarles}
  Xavier Xarles, \emph{The scheme of connected components of the N�ron model of an algebraic torus},
  Journal f�r die reine und angewandte Mathematik 437 (1993): 167-180.

\bibitem{yu}
  Jiu-Kang Yu, \emph{Smooth models associated to concave functions in Bruhat-Tits theory}. Preprint, 2003.

\bibitem{zaremsky}
  Matthew Zaremsky,
  \emph{Representatives of elliptic Weyl group elements in algebraic groups}, preprint, arxiv:1109.5487.

\end{thebibliography}

\end{document}
