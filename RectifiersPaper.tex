\documentclass{amsart}
\usepackage{amsmath,amscd,amssymb,latexsym,amsthm,amsfonts}
\usepackage{amsrefs}
\usepackage{mathtools}
\usepackage{color}
%\usepackage{hyperref}

\theoremstyle{plain}
\newtheorem{iassumption}{Assumption}
\newtheorem{theorem}{Theorem}[section]
\newtheorem{conjecture}[theorem]{Conjecture}
\newtheorem{proposition}[theorem]{Proposition}
\newtheorem{corollary}[theorem]{Corollary}
\newtheorem{hypothesis}[theorem]{Hypothesis}
\newtheorem{assumption}[theorem]{Assumption}
\newtheorem{lemma}[theorem]{Lemma}
\newtheorem{question}[theorem]{Question}
\newtheorem{exercise}[theorem]{Exercise}
\newtheorem{statement}[theorem]{Statement}
\newtheorem{example}[theorem]{Example}

\newcommand{\MAxxx}[1]{$\clubsuit$\footnote{#1}}
\newcommand{\DRxxx}[1]{$\spadesuit$\footnote{#1}}
\newcommand{\HT}[1]{\hat{\HH}{}^{#1}}

\theoremstyle{definition}
\newtheorem{definition}[theorem]{Definition}
\newtheorem{remark}[theorem]{Remark}
\numberwithin{equation}{section}

%\setlength{\oddsidemargin}{0.2in}
%\setlength{\evensidemargin}{0.2in}
%\setlength{\textwidth}{6.1in}

\DeclareMathOperator{\Gal}{Gal}
\DeclareMathOperator{\val}{val}
\DeclareMathOperator{\HH}{H}
\DeclareMathOperator{\Ad}{Ad}
\DeclareMathOperator{\Nm}{Nm}
\DeclareMathOperator{\Hom}{Hom}
\DeclareMathOperator{\Spec}{Spec}
\DeclareMathOperator{\Res}{Res}
\DeclareMathOperator{\Fr}{Fr}
\DeclareMathOperator{\Ind}{Ind}
\DeclareMathOperator{\gen}{gen}
\DeclareMathOperator{\Z}{Z}

\DeclareMathOperator{\GL}{GL}
\DeclareMathOperator{\PGL}{PGL}
\DeclareMathOperator{\SL}{SL}
\DeclareMathOperator{\GSp}{GSp}

\newcommand{\TT}{\mathcal{T}}
\newcommand{\C}{\mathcal{C}}
\newcommand{\CC}{\mathbb{C}}
\newcommand{\CCx}{\mathbb{C}^\times}
\newcommand{\OK}{\mathcal{O}_K}
\newcommand{\OKn}{\mathcal{O}_{K_n}}
\newcommand{\PK}{\mathcal{P}_K}
\newcommand{\PL}{\mathcal{P}_L}
\newcommand{\OL}{\mathcal{O}_L}
\newcommand{\ZZ}{\mathbb{Z}}
\newcommand{\QQ}{\mathbb{Q}}
\newcommand{\Gm}{\mathbb{G}_m}
\newcommand{\Kx}{K^\times}
\newcommand{\Lx}{L^\times}
\newcommand{\Fq}{\mathbb{F}_q}
\newcommand{\Fqb}{\bar{\mathbb{F}}_q}

\newcommand{\Weil}{\mathcal{W}}
\newcommand{\I}{\mathcal{I}}
\newcommand{\WD}{\mathcal{W}'}
\newcommand{\Lpack}{\mathcal{L}}
\newcommand{\Adm}{\Pi}
\newcommand{\LP}{\mathcal{P}}
\DeclareMathOperator{\Sup}{Sup}
\newcommand{\Pmin}{P_G^{\min}}
\newcommand{\bmu}{\boldsymbol\mu}
\newcommand{\mumin}{\bmu^{\min}}

\newcommand{\st}{\ensuremath{\ \ \ \vert\ }}
\newcommand{\la}{\langle}
\newcommand{\ra}{\rangle}

\newcommand{\invlim}[1]{\varprojlim_{#1}}
\newcommand{\Normalizer}[2]{\operatorname{N}_{#2}(#1)}

\newcommand{\Thadm}{T^*_{\operatorname{adm}}}
\newcommand{\Thinadm}{T^*_{\operatorname{in}}}
\newcommand{\hatT}{T^*}

\begin{document}
\title[Rectifiers and the local Langlands Correspondence]{Rectifiers and the local Langlands Correspondence: the unramified case}
\author{Moshe Adrian}
\email{madrian@math.utah.edu}
\address{Department of Mathematics, University of Toronto, Toronto, ON M5S 2E4, Canada}
\author{David Roe}
\email{roed.math@gmail.com}
\address{Department of Mathematics, University of British Columbia, Vancouver, BC V6K 1Z2, Canada}
\subjclass[2010]{22E50}
\keywords{Langlands, rectifiers}
\thanks{The second author was supported by the Pacific Institute for the Mathematical Sciences}


\begin{abstract}

Bushnell and Henniart define rectifiers, which provide a correction term
in the local Langlands correspondence for $\GL_n(K)$.  They also give a natural bijection
between essentially tame supercuspidal Langlands parameters
and characters of minisotropic tori, and a second bijection between characters
of minisotropic tori and supercuspidal representations of $\GL_n(K)$.  Rectifiers
bridge the gap, adding an intermediate step so that the composition agrees
with the local Langlands correspondence.
In this paper, we begin the process of generalizing rectifiers to other
connected reductive groups, focusing on the case of unramified minisotropic
tori that satisfy a certain cohomology condition.

\end{abstract}

\maketitle

\section{Introduction} \label{section:intro}

Let $G$ be a connected reductive group defined over a $p$-adic field $K$.
The local Langlands conjecture predicts the existence of a finite to one map
from the set of isomorphism classes of irreducible admissible representations
of $G(K)$ to the set of Langlands parameters for $G(K)$.

There has been a significant amount of progress in recent years
focusing on supercuspidal representations of $G(K)$.  Bushnell--Henniart \cite{bushnell-henniart:10a},
DeBacker--Reeder \cite{reeder-debacker:09a}, Kaletha \cite{kaletha:13a} and Reeder \cite{reeder:08a}
approach the task of constructing $L$-packets by first attaching
a character of an elliptic torus to a Langlands parameter, and
then associating a collection of supercuspidal representations to this character.
Their constructions all use the local Langlands correspondence for tori in some way.  However,
since the image of the Langlands parameter $\Weil_K \rightarrow {}^L G$ is not
necessarily contained within the $L$-group of a maximal torus, they need to make certain
adjustments in order to produce a character of an elliptic torus.  The different authors remedy this situation in various ways.

For example, consider the group $G = \PGL_{2}(K)$, and suppose that
$\varphi : \Weil_K \rightarrow \SL_{2}(\CC)$ is an
irreducible representation.  Then $\varphi$ is a Langlands parameter corresponding
to a supercuspidal representation of $\PGL_2(K)$.  Moreover, the image of $\varphi$
is contained in the normalizer of the dual torus $\hat{T}$, a non-split
extension of $\Gal(L/K)$ by $\hat{T}$, but not in the $L$-group of any torus.

If $p \neq 2$, then there is a tamely ramified quadratic
extension $L/K$ and a character
$\chi$ of $\Lx$ that is trivial on the norms $\Nm_{L/K}(\Lx)$ and nontrivial on $\Kx$, so that
$\varphi = \Ind_{\Weil_L}^{\Weil_K}(\chi)$ \cite{bushnell-henniart:06a}*{\S 34}.  In particular, $\chi$ is a character of the group $\Lx / \Nm_{L/K}(\Lx)$.
By Hilbert's theorem 90, the group $\Lx / \Nm_{L/K}(\Lx)$ appears as a covering group of the elliptic torus $L^1$ of norm $1$ elements in $L$:
\begin{align*}
1 \rightarrow \mathbb{Z} / 2 \mathbb{Z} \rightarrow \Lx / \Nm_{L/K}(\Lx) &\rightarrow L^1 \rightarrow 1 \\
x \Nm_{L/K}(L^{\times}) &\mapsto x / \sigma(x);
\end{align*}
here $\sigma$ generates $\Gal(L/K)$.  In particular, the Langlands parameter
$\varphi$ naturally provides a character $\chi$, not of the elliptic torus
$L^1 \subset \PGL_2(K)$ (since $\chi$ is nontrivial on $\Kx$),
but of the two-fold cover $\Lx / \Nm_{L/K}(\Lx)$.  We can obtain a character $\chi'$ of $L^1$
by tensoring $\chi$ by another character of
$\Lx / \Nm_{L/K}(\Lx)$ that is nontrivial on $\Kx$.  One can then attach a supercuspidal representation
of $\PGL_2(K)$ to $\chi'$ via the construction of Bushnell and Kutzko \cite{bushnell-kutzko:AdmissibleDual}.
The tensoring character that gives the correct supercuspidal representation of $\PGL_2(K)$
is precisely what appears in Bushnell and Henniart's rectifier.  In \cite{bushnell-henniart:10a},
Bushnell and Henniart compute this rectifier in the \emph{essentially tame} setting for $\GL_n(K)$.

Bushnell and Henniart motivate their rectifier as follows.
Suppose that $\varphi$ is an essentially tame supercuspidal Langlands parameter for
$\GL_n(K)$.  The local Langlands correspondence for tori then yields a degree $n$ extension
$L/K$ and a character $\xi$ of $L^{\times}$.  We now fix a construction $\chi \mapsto \pi_{\chi}$
of supercuspidal representations of $\GL_n(K)$ from \emph{admissible} characters of $L^{\times}$.
Then the rectifier of $\xi$ is a character $\mu_{\xi}$ of $L^{\times}$ such that
$\varphi \mapsto \pi_{\xi \cdot \mu_{\xi}}$ is the local Langlands correspondence for $\GL_n(K)$.
In the specific case of supercuspidal representations of $\GL_2(K)$ with trivial central character
(i.e. supercuspidal representations of $\PGL_2(K)$), one computes that the character $\mu_{\xi}$ is,
as discussed above, a character of $\Lx / \Nm_{L/K}(\Lx)$ that is nontrivial on $\Kx$.

In this paper, we initiate a program to generalize Bushnell and Henniart's
rectifier to groups other than $\GL_n(K)$. Suppose that $G$ is a connected reductive group defined over $K$.
Let $\varphi : \Weil_K \rightarrow {}^L G$ be a
Langlands parameter for $G(K)$, and suppose that $\varphi$ factors
through the normalizer of a maximal torus in the dual group.
Benedict Gross has recently used his theory of groups of type $L$ to show that one may attach to each such parameter a character of a group that covers a subgroup of a maximal torus in $G(K)$.  We briefly describe this construction, since it is needed in our definition of rectifier; for more details see \S \ref{section:groups_of_type_L}.
To $\varphi$, one can associate a maximal $K$-torus $T$ in $G$, unique up to
stable conjugacy.  Let $L$ be the splitting field of $T$ and set $\Gamma = \Gal(L/K)$.
Restricting $\varphi$ to $\Weil_L$, the local Langlands correspondence for tori associates to $\varphi|_{\Weil_L}$ a character $\xi$ of $T(L)$.  In fact, $\xi$ factors through the coinvariants $T(L)_{\Gamma}$.
Moreover, invariants and coinvariants are related by the norm map
\begin{align*}
N : T(L) &\rightarrow T(K),\\
t &\mapsto \displaystyle\prod_{\sigma \in \Gamma} \sigma(t),
\end{align*}
in the cohomology sequence
\begin{equation} \label{eq:coinv_seq}
1 \rightarrow \HT{-1}(\Gamma,T(L)) \rightarrow T(L)_{\Gamma} \xrightarrow{N} T(K) = T(L)^{\Gamma} \rightarrow \HT{0}(\Gamma,T(L)) \rightarrow 1,
\end{equation}
where $\HT{}$ denotes Tate cohomology.  We have attached to each $\varphi$ a character $\xi$ of $T(L)_{\Gamma}$, which is in general a covering group of a finite-index subgroup of $T(K)$.  In order to use existing constructions of representations of $G(K)$, we need a character of $T(K)$ itself.  The outer terms in \eqref{eq:coinv_seq} provide an obstruction in shifting to a character of $T(K)$.

In this paper, we sidestep half of the problem by assuming that $\HT{0}(\Gamma, T(L)) = 0$.  This condition holds for unramified minisotropic tori in both $\GL_n$
and in semisimple groups.  Such tori arise for discrete parameters, where $\varphi$ does not factor through a proper Levi subgroup.  In the case of $\GL_n$, we also have $\HT{-1}(\Gamma, T(L)) = 0$
for any unramified minisotropic torus, so $T(L)_{\Gamma} \cong T(K)$ can be identified with $L^{\times}$.  To construct the local Langlands
correspondence for $\GL_n(K)$, one would then proceed to
attach the supercuspidal representation $\pi_{\xi \cdot \mu_{\xi}}$ to $\xi$, via the construction of Bushnell and Henniart.

For many $G$,
there are constructions of supercuspidal $L$-packets
$\Lpack(\chi)$ associated to characters $\chi$ of elliptic tori $T(K) \subset G(K)$.  However, a Langlands parameter $\varphi$ does not naturally
provide a character of $T(K)$, but rather a character $\xi$ of
$T(L)_{\Gamma}$.  After fixing an association $\chi \mapsto \Lpack(\chi)$ of supercuspidal $L$-packets of $G$
to \emph{admissible} characters of $T(K)$ (for example, the association of \cite{reeder-debacker:09a}, \cite{kaletha:13a}, or \cite{reeder:08a}), the \emph{rectifier} will be a character
$\mu_{\xi}$ of the covering group $T(L)_{\Gamma}$ such that $\varphi \mapsto \Lpack(\xi \cdot \mu_{\xi})$
agrees with existing constructions of the local Langlands correspondence for $G(K)$, just as in the case of $\GL_n(K)$.  We note that if $G(K) = \PGL_2(K)$ and if $\varphi$ is supercuspidal, then the Tate cohomology sequence above is exactly the $2$-fold cover $\Lx / \Nm_{L/K}(\Lx)$ of $L^1$ described earlier.

The associations $\chi \mapsto \Lpack(\chi)$ that we use in our definition of rectifiers are those of DeBacker--Reeder and Reeder. We define rectifiers for
unramified minisotropic tori $T$ whose zeroth Tate cohomology is trivial.
We show in Theorem \ref{thm:unique_semisimple} that rectifiers for semisimple $G$ exist
and are unique up to equivalence.  In fact, the rectifier that we construct is canonical: it is a character of $T(L)_{\Gamma}$ that is constructed from
a canonical Langlands parameter (see Definition \ref{def:phiu}).  We show that our rectifier $\mu_{\xi}$ produces the correspondence of DeBacker--Reeder and Reeder.  We also show that our rectifier agrees with that of Bushnell--Henniart in the setting of depth zero
supercuspidal representations of $\GL_{n}(K)$ (see Theorem \ref{thm:bh_agreement}).

Note that there is an obstruction to proving the compatibility of our rectifier with Bushnell--Henniart's in the positive depth case.
In the depth zero case, Deligne--Lusztig representations provide a canonical way of
attaching supercuspidal representations to characters. However, in positive depth there are many:
Adler \cite{adler:98a}, Howe \cite{howe:77a}, Bushnell--Henniart \cite{bushnell-henniart:10a},
Bushnell--Kutzko \cite{bushnell-kutzko:AdmissibleDual}, and Yu \cite{yu:03a}.
In the positive depth setting, the rectifier will depend on the methods
used to construct representations from the character of $T(K)$, and
our rectifier indeed differs from that of Bushnell and Henniart in positive depth.

We would like to remark that the notion of a covering group of a torus occurring in the local Langlands
correspondence is an old one, dating back at least to work of Adams and Vogan \cite{adams-vogan:91}
in the setting of real groups.  In the theory of real groups, an admissible homomorphism
$W_{\mathbb{R}} \rightarrow {}^L G$ automatically has image inside the normalizer of a torus in ${}^L G$.
As such, it naturally produces a genuine character $\chi$ of the \emph{$\rho$-cover of $T(\mathbb{R})$},
denoted $T(\mathbb{R})_{\rho}$, a certain double cover of $T(\mathbb{R})$.  To $\chi$,
one can naturally attach a collection of representations of $G(\mathbb{R})$ to construct an $L$-packet.

We emphasize that our results do not extend the scope of DeBacker and Reeder's constructions to a broader class of parameters.
Instead, we aim to translate Bushnell and Henniart's notion of rectifier in such a way that it can be applied
to more general groups than $\GL_n$.  In this paper we only handle the case where we may attach a torus $T$
to the parameter that is both unramified and satisfies $\HT{0}(\Gamma, T(L)) = 0$.  We believe that these restrictions can be removed in future work.

We now present an outline of the paper.  In \S\ref{section:BH_recall} we recall
the notion of rectifier due to Bushnell and Henniart and describe
the rectifier in the setting that we will need.  In \S\ref{section:padic_tori}
we present some results about Tate cohomology of $p$-adic tori that will be used
in the rest of the paper.  In \S\ref{section:groups_of_type_L} we review
the theory of groups of type L.  In
\S\ref{section:gross_debacker_reeder} we describe the relationship between the
construction of Gross, via groups of type L, and the constructions $\chi \mapsto \Lpack(\chi)$ of
\cite{reeder-debacker:09a} and \cite{reeder:08a}.  In \S\ref{Q_T} we study how translation by a character affects
the map $\chi \mapsto \Lpack(\chi)$.
In \S\ref{section:general_rectifiers} we
introduce our notion of rectifier and prove our main result, Theorem \ref{thm:unique_semisimple}.
Finally, in \S\ref{section:BH_compat} we show that our rectifier is compatible with
the rectifier of Bushnell and Henniart in the setting of depth zero
supercuspidal representations of $\GL_n(K)$.

\subsection*{Acknowledgements}

This paper has benefited from conversations with Jeffrey Adams, Jeffrey Adler, Colin Bushnell,
Andrew Fiori, Guy Henniart, Gordan Savin, and Geo Kam-Fai Tam.  We thank them all.

\section{Notation and Preliminaries} \label{section:notation}

Throughout, $K$ will denote a nonarchimedean local field of
characteristic zero, $\OK$ its ring of integers, $k$ its residue field of cardinality $q$,
$\PK$ the maximal ideal in $\OK$ and $\varpi$ a fixed uniformizer.
Write $K_n$ for the unramified extension of $K$ of degree $n$, $k_n$ for
the degree $n$ extension of $k$,
and set $\Gamma_n = \Gal(K_n/K) = \Gal(k_n/k)$. Let $\bar{K}$ and $\bar{k}$ be algebraic closures of $K, k$, respectively.

A geometric Frobenius is an element of $\Gal(\bar{K}/K)$
inducing the automorphism $x \mapsto x^{1/p}$ of $\bar{k}$.  Under the
Artin reciprocity map of local class field theory the choice of $\varpi$
determines a geometric Frobenius $\Fr$ \cite{serre:LocalClassFieldThy}*{\S 2}.

If $\chi : K^{\times} \rightarrow \mathbb{C}^{\times}$ is a character, we define
the \emph{depth} of $\chi$ to be the smallest integer $r$ such that
$\chi|_{1 + \PK^{r+1}} \equiv 1$ and
$\chi|_{1 + \PK^{r}} \not\equiv 1$.

If $T$ is a torus defined over $K$ we write $X^*(T)$
for the character lattice $\Hom_{\bar{K}}(T, \Gm)$ and $X_*(T)$ for the
cocharacter lattice $\Hom_{\bar{K}}(\Gm, T)$ \cite{humphreys:LinAlgGrps}*{\S 16.2}.
$T$ will split over an extension
$L$ of $K$ if and only if $\Gal(\bar{K}/L)$ acts trivially on $X^*(T)$.
We may thus define \emph{the} splitting field $L$ of $T$ as the
minimal extension of $K$ splitting $T$; note that $L$ is necessarily
Galois over $K$.  Write $\Gamma$ for $\Gal(L/K)$. Then $X_*(T)$, $X^*(T)$ and $T(L)$
are all $\Gamma$-modules.

Suppose now that $T \subset G$ for a connected reductive group $G$ over $K$.
We will write $\hat{T} \subset \hat{G}$ for the dual torus in the complex dual group of $G$ \cite{borel:79a}*{\S I.2}.
Let $N$ be the normalizer $\Normalizer{T}{G}$ of $T$ in $G$ and define $W = N/T$;
set $\hat{N} = \Normalizer{\hat{T}}{\hat{G}}$ and
$\hat{W} = \hat{N}/\hat{T}$.  The identification of $X^*(T)$ and $X_*(\hat{T})$
yields a canonical anti-isomorphism between $W$ and $\hat{W}$.
Note that $W$ is a scheme over $K$; in general $W(K) \ne N(K) / T(K)$.

Write $\Nm$ for the norm map
\begin{align*}
T(L) &\rightarrow T(K) \\
t &\mapsto \prod_{\sigma \in \Gamma} \sigma(t)
\end{align*}
and for its restriction to $X_*(T)$.

The following theorem, due to Lang \cite{lang:56a}, underpins the facts in
\S\ref{section:padic_tori} on tori over $p$-adic fields.
Let $H$ be a commutative connected algebraic group over a
finite field $k$, and suppose $H$ splits over $k_n$.  Denote by $\HT{i}$ the $i^{\mathrm{th}}$
Tate cohomology group.

\begin{theorem} \label{thm:lang}
$\HT{i}(\Gamma_n, H(k_n)) = 0$ for all $i$.
\end{theorem}
\begin{proof}
Since $\Gamma_n$ is cyclic,
$\HT{i}(\Gamma_n, H(k_n)) \cong \HT{i+2}(\Gamma_n, H(k_n))$ \cite{atiyah-wall:CohomologyGrps}*{Thm. 5},
so it suffices to prove the result for $i=1$ and $i=2$, which is done
by Serre \cite{serre:AlgGrpsClassFields}*{\S VI.6}.
\end{proof}

\section{Rectifier for $\GL_{n}(K)$} \label{section:BH_recall}

In this section we recall the rectifier of Bushnell and Henniart and their construction of the
essentially tame local Langlands correspondence for $\GL_{n}(K)$.
An irreducible smooth representation of the Weil group $\Weil_K$ of $K$ is
called \emph{essentially tame} if its restriction to wild inertia is a
sum of characters.
\begin{definition}\label{admissiblepairhowe}
Let $L/K$ be an extension of degree $n$, with $n$ coprime to $p$.  A character
$\xi$ of $L^\times$ is \emph{admissible} if
\begin{enumerate}
\item $\xi$ doesn't factors through the norm from a subfield of $L$ containing $K$,
\item If $\xi|_{1 + \PL}$ factors through the norm from a proper subfield $L \supseteq M \supseteq K$, then
$L/M$ is unramified.
\end{enumerate}
\end{definition}
There is a natural bijection
$\varphi_{\xi} \leftrightarrow (L/K, \xi)$ between irreducible smooth essentially tame
$\varphi_{\xi} : \Weil_K \rightarrow \GL_{n}(\mathbb{C})$ and
\emph{admissible pairs} $(L/K, \xi)$.
Bushnell and Henniart
construct a map (see \cite{bushnell-henniart:10a})
\begin{equation*}
\left\{
\begin{array}{cc}
\mathrm{isomorphism \ classes \ of} \\
\mathrm{admissible \ pairs}
\end{array}
\right\} \rightarrow \left\{
\begin{array}{cc}
\mathrm{supercuspidal \ representations} \\
\mathrm{of} \ \GL_{n}(K)
\end{array} \right\}
\end{equation*}
$$\hspace{-.5in} (L/K, \xi) \mapsto \pi_{\xi}$$
However, the map $$\varphi_{\xi} \mapsto \pi_{\xi}$$
is not the local Langlands
correspondence because $\pi_{\xi}$ has the wrong central character.
Instead, the local Langlands correspondence is given by
\begin{equation}\label{llcgln}
\varphi_{\xi} \mapsto \pi_{\xi \cdot {}_K \mu_{\xi}} \tag{$\star$}
\end{equation}
for some subtle finite order
character ${}_K \mu_{\xi}$ of $\Lx$.  Since we will not be changing $K$
in this paper we will write $\mu_\xi$ for ${}_K \mu_{\xi}$.

The relation $\eqref{llcgln}$ does not determine $\mu_{\xi}$ uniquely.  As pointed out
in \cite{bushnell-henniart:10a}, the obstruction to uniqueness revolves around the
group $\GL_2(\mathbb{F}_3)$.  Bushnell and Henniart therefore make the following definition \cite{bushnell-henniart:10a}*{Def. 1}.

\begin{definition}\label{rectifierbushnellhenniart}
Let $L/K$ be a finite, tamely ramified field extension of degree $n$.  A \emph{rectifier}
for $L/K$ is a function
$$\bmu : (L/K, \xi) \mapsto \mu_{\xi}$$
which attaches to each admissible pair $(L/K, \xi)$ a character $\mu_{\xi}$ of $L^{\times}$
satisfying the following conditions:
\begin{enumerate}
\item The character $\mu_{\xi}$ is tamely ramified.
\item Writing $\xi' = \xi \cdot \mu_{\xi}$, the pair $(L/K, \xi')$ is admissible and
$\varphi_{\xi} \mapsto \pi_{\xi \cdot \mu_{\xi}}$ is the local Langlands correspondence
for $\GL_n(K)$.
\item If $(L/K, \xi_i), i = 1,2$, are admissible pairs such that $\xi_1^{-1} \xi_2$ is
tamely ramified, then $ \mu_{\xi_1} =  \mu_{\xi_2}$.
\end{enumerate}
\end{definition}

Bushnell and Henniart then prove \cite{bushnell-henniart:10a}*{Thm. A}:

\begin{theorem}
Any finite, tamely ramified, field extension $L/K$ admits a unique rectifier
$\bmu : (L/K, \xi) \mapsto \mu_{\xi}$.
\end{theorem}

Both the description of and the
intuition behind the rectifiers $\bmu$ have been
studied (see \cite{bushnell-henniart:10a}, \cite{tam:12a}, \cite{adrian:13a}).  In order to
generalize rectifiers to groups other than $\GL_n(K)$ we
will will need a description of the characters $\mu_{\xi}$ in certain cases.  The following result comes immediately from \cite{bushnell-henniart:10a}*{Prop. 21}.

\begin{proposition}\label{prop:BH_result1}
  Suppose that $(L/K, \xi)$ is an admissible pair, where $\xi$ has depth $0$.
  Then $\mu_{\xi}$ is unramified and
  $\mu_{\xi}(\varpi) = (-1)^{n-1}$.
\end{proposition}

\section{Tori over $p$-adic fields} \label{section:padic_tori}
We first recall the definition of a minisotropic torus.

\begin{definition}[{\cite{reeder:08a}*{\S3.1}}]\label{minisotropictorus}
If $T$ is an unramified maximal $K$-torus in $G$, we say that $T$ is $K$-\emph{minisotropic} if $T(K)/Z(K)$ is compact.
\end{definition}

Now let $T \subset G$ be a torus defined over $K$ with splitting field $L$, let $K_n$ be the maximal
unramified subextension $L/K$, set $\Gamma = \Gal(L/K)$ and $I = \Gal(L/K_n)$.
In studying rectifiers for groups other than $\GL_n$, the following exact sequence
in Tate cohomology plays a crucial role:

\begin{equation} \label{eq:tate4}
1 \rightarrow \HT{-1}(\Gamma,T(L)) \rightarrow T(L)_{\Gamma} \xrightarrow{\Nm} T(K)
  = T(L)^{\Gamma} \rightarrow \HT{0}(\Gamma,T(L)) \rightarrow 1.
\end{equation}

\noindent We make the following definition as a matter of notational convenience, since
it will serve as a running hypothesis on $T$ for most of the rest of the paper.

\begin{definition} \label{def:coverable}
We say that $T$ is \emph{coverable} if it is $K$-minisotropic and $\HT{0}(\Gamma, T(L)) = 0$.
\end{definition}

We require that $T$ be minisotropic in order to be able to use the local Langlands correspondence
given by Reeder \cite{reeder:08a} and Debacker--Reeder \cite{reeder-debacker:09a}; the condition
on $\HT{0}(\Gamma, T(L))$ will allow us to define characters of $T(K)$ from characters of
$T(L)_\Gamma$.  In the remainder of this section recall some tools for computing Tate cohomology
groups over tori and give examples of coverable and non-coverable tori.

Let $\TT$ be the N\'eron model of $T$, a canonical model of $T$
over $\OK$ \cite[Ch. 10]{bosch-lutkebohmert-reynaud:NeronModels}.
As a consequence of the N\'eron mapping
property, we may identify $\TT(\OK)$ with $T(K)$.  The connected
component of the identity, $\TT^\circ$, cuts out a subgroup
$T(K)_0 = \TT^\circ(\OK)$ of $T(K)$; we also write $T(K_n)_0$ for
$\TT^\circ(\OKn)$.

In fact, this subgroup of $T(K)$ is the first in a decreasing filtration.
Moy and Prasad \cite{moy-prasad:96a}
define one such filtration by
embedding $T$ into an induced torus and defining the filtration of
$\Res_{L/K} \Gm$ in terms of the valuation on $L$.  Yu \cite{yu:03a}*{\S 5}
describes a different filtration, agreeing with that of Moy and Prasad
in the case of tame tori but with nicer features in the presence of wild
ramification.  Let $\{\TT_r\}_{r \ge 0}$ be the integral models of $T$ defined in Yu's
minimal congruent filtration and let $\{T(K)_r\}_{r \ge 0}$ and
$\{T(K_n)_r\}_{r \ge 0}$ be the corresponding filtrations of $T(K)$ and
$T(K_n)$.

Let $\C$ be the scheme of
connected components of $\TT$,
which we may identify with the
components of $\TT \times \Spec(k)$ since $T = \TT \times \Spec(K)$
is connected.  The structure of $\C$ is described by Bertapelle and Gonz\'alez-Avil\'es:

\begin{proposition}[{\cite{bertapelle-gonzalez:13b}*{Thm. 1.1}}]
As $\Gal(\bar{k}/k)$-modules,
\[
\C \cong X_*(T)_I.
\]
\end{proposition}

Using our filtration of $T(K_n)$, we may relate the cohomology of $T(K_n)$
with that of $\C$.

\begin{proposition}\label{prop:T0_cohom_triv}
$\HT{i}(\Gamma_n, T(K_n)_0) = 0$ for all $i$.
\end{proposition}
\begin{proof}
Note that
$$T(K_n)_0 = \invlim{r} T(K_n)_0 / T(K_n)_r.$$
So by a result of Serre \cite{serre:LocalClassFieldThy}*{Lem. 3}, it suffices to prove that
\\ $\HT{i}(\Gamma_n, T(K_n)_r / T(K_n)_{r+}) = 0$ for all $i$.  But $T(K_n)_r / T(K_n)_{r+}$
is connected \cite{yu:03a}*{Prop. 5.2} and thus has trivial cohomology by
Theorem \ref{thm:lang}.
\end{proof}

\begin{corollary} \label{cor:cohom_tori}
$\HT{i}(\Gamma_n, T(K_n)) \cong \HT{i}(\Gamma_n, X_*(T)_I)$.
\end{corollary}

\begin{proof}
This follows from the long exact sequence in cohomology associated to the sequence
$$0 \rightarrow \TT^0 \rightarrow \TT \rightarrow \C \rightarrow 0.$$
\end{proof}

Using this corollary, we can give examples of coverable and non-coverable tori.

\begin{example} $ $
\begin{enumerate}
\item Each minisotropic torus in $\GL_m$ is of the form $T = \Res_{N/K} \Gm$ for a degree $m$ extension $N$ of $K$.
Write $L$ for the Galois closure of $N$ and $H = \Gal(L/N)$.  As usual, set $\Gamma=\Gal(L/K)$, let $K_n$ be the
maximal unramified subextension of $N$ and choose a Frobenius lift $\Fr \in \Gamma$.
We may take a basis $\{e_C\}$ for $X_*(T)$ indexed by left cosets $C$ of $H$ in $\Gamma$, with $\Gamma$
permuting the basis through left multiplication on the index.  Then $X_*(T)_I$ is spanned by the classes of
$e_{\Fr^i H}$ for $0 \le i < m$, permuted cyclically.  An easy computation using Corollary \ref{cor:cohom_tori} now shows that
\[
\HT{-1}(\Gamma_n, T(K_n)) = \HT{0}(\Gamma_n, T(K_n)) = 0.
\]

\item For an example of an unramified torus with nontrivial cohomology, let $T$ be a torus in $\GSp_4$ \cite{morris:91a}*{Prop. 1.3}, split over a quartic unramified extension
$L/K$ with $\Gamma = \Gal(L/K) = \langle \tau \rangle$, with cocharacter lattice
\begin{align*}
X_*(T) = &\{(a,b,c,d) \in \ZZ^4 \: : \: a + d = b + c\} \\
& \tau(a,b,c,d) = (c,a,d,b).
\end{align*}
Then $T$ is minisotropic but has $\HT{0}(\Gamma, T(L)) \cong \ZZ/2\ZZ$ by Corollary \ref{cor:cohom_tori}:
$X_*(T)^\Gamma$ is spanned by $(1,1,1,1)$ while the image of the norm map $(a,b,c,d) \mapsto (a+b+c+d,\ldots,a+b+c+d)$
is spanned by $(2,2,2,2)$.  A similar computation shows that $\HT{-1}(\Gamma, T(L)) = 0$.
\end{enumerate}
\end{example}

On the other hand, if $G$ is semisimple, examples as in (2) above do not occur, as seen from the following proposition.

\begin{proposition}\label{prop:vanishing_H0}
If $T$ is unramified and anisotropic, then $\HT{0}(\Gamma, T(L)) = 0$.
\end{proposition}

\begin{proof}
Since $T$ is anisotropic, $X_*(T)^{\Gamma} = 0$, giving $\HT{0}(\Gamma, T(L)) = 0$ by Corollary
\ref{cor:cohom_tori}.
\end{proof}

For unramified $T$ the jumps in the filtration on $T(K)$ and $T(L)$ occur at integers, and we write
\begin{align*}
T(\OK) &= T(K)_0, \\
T(\OL) &= T(L)_0, \\
T(\PK^r) &= T(K)_r\qquad \mbox{for $r > 0$}, \\
T(\PL^r) &= T(L)_r\qquad\,\,\mbox{for $r > 0$}.
\end{align*}

\section{Groups of type L} \label{section:groups_of_type_L}

We now review the theory of groups of type L \cite{roe:13a}*{\S 7}.
For a torus $T$ over $K$ recall that the dual torus $\hat{T}$ is equipped with
an action of $\Gamma$.

\begin{definition}
A \emph{group of type L} is a group extension of $\Gamma$ by $\hat{T}$.
\end{definition}

For such a group $D$ we have by definition an exact sequence
$$1 \rightarrow \hat{T} \rightarrow D \rightarrow \Gamma \rightarrow 1.$$

We now describe how we can naturally attach a character of the coinvariants
$T(L)_{\Gamma}$ to a Langlands parameter
$$\varphi : \Weil_K \rightarrow D$$
with values in a group of type L.
Restricting $\varphi$ to $\Weil_L$ we get a homomorphism
$$\varphi|_{\Weil_L} : \Weil_L \rightarrow \hat{T},$$
and by the Langlands correspondence for tori a character
$\xi_{\varphi} : T(L) \rightarrow \CCx$.  Since $\varphi|_{\Weil_L}$ extends
to $\varphi$ we have that
$$\xi_{\varphi}(\sigma(t)) = \xi_{\varphi}(t)\ \mbox{for all $\sigma \in \Gamma$.}$$
Thus $\xi_{\varphi}$ is trivial on the augmentation ideal $I_{\Gamma}(T(L))$
and descends to $$\xi_{\varphi} : T(L)_\Gamma \rightarrow \CCx.$$

Invariants
and coinvariants are related by the norm map
in the Tate cohomology sequence
$$1 \rightarrow \HT{-1}(\Gamma,T(L)) \rightarrow T(L)_{\Gamma} \xrightarrow{\Nm} T(K)
  = T(L)^{\Gamma} \rightarrow \HT{0}(\Gamma,T(L)) \rightarrow 1.$$
We will assume in \S \ref{section:general_rectifiers} that $\HT{0}(\Gamma,T(L)) = 0$, in which case
$\xi_\varphi$ is a character of a cover of $T(K)$.

We will need the following structural result about Langlands
parameters mapping to groups of type L for the proof of
Proposition \ref{prop:existenceofrectifier}.  Suppose now that $L/K$ is unramified of degree $n$ and that
$\varphi$ and $\varphi'$ are two Langlands parameters
with $\varphi'(\Fr) \varphi(\Fr)^{-1} \in \hat{T}$.
Let $\xi$ and $\xi'$ be the associated characters of $T(L)_{\Gamma}$.

\begin{lemma} \label{lem:toral_modification}
$\xi$ and $\xi'$ have the same restriction to $\HT{-1}(\Gamma, T(L))$.
\end{lemma}

\begin{proof}
Set $y = \varphi(\Fr) \in D$.  By Corollary \ref{cor:cohom_tori},
$\HT{-1}(\Gamma, T(L)) = \HT{-1}(\Gamma, X_*(T))$.
If suffices to show that $\xi(\lambda) = \xi'(\lambda)$ for any $\lambda \in \ker(\Nm : X_*(T) \rightarrow X_*(T))$.

Via the canonical identification $X_*(T) \cong X^*(\hat{T})$,
we regard $\lambda$ as an element of $X^*(\hat{T})$.  Note that $y^n \in \hat{T} \cong X_*(\hat{T}) \otimes \CC^\times$. We assume that $y^n$ is a simple tensor $y^n = \mu \otimes z$, for some $\mu \in X_*(\hat{T})$ and
$z \in \CC^{\times}$; for general $y^n$, the ensuing computations are analogous.  Write $\langle , \rangle$ for the canonical pairing $X_*(\hat{T}) \times X^*(\hat{T}) \to \ZZ$.  The local Langlands
correspondence for tori implies that $\xi(\lambda) = z^{\langle \mu, \lambda \rangle}$.

Let $t = \varphi'(\Fr)\varphi(\Fr)^{-1} \in \hat{T}$, and let $w$ be the image of $y$ in the Weyl group. Again we assume that $t$ is a simple tensor $t = \mu' \otimes z'$ for $\mu' \in X_*(\hat{T})$
and $z' \in \mathbb{C}^{\times}$.  Then
\begin{align*}
(ty)^n &= t yty^{-1} y^2 t y^{-2} \cdots y^{n-1} t y^{1-n} y^n \\
&= t w(t) w^2(t) \cdots w^{n-1}(t) y^n \\
&=(\Nm(\mu' ) \otimes z') \cdot (\mu \otimes z),
\end{align*}
where here $\Nm$ denotes norm map on $\hat{T}$ corresponding to the given Galois action on $\hat{T}$.

We conclude again by the local Langlands correspondence for tori that
\[
\xi'(\lambda) = z^{\langle \mu, \lambda \rangle} (z' )^{\langle \Nm(\mu'), \lambda \rangle}.
\]
Since $\langle , \rangle$ is Galois equivariant, $\langle \Nm(\mu'), \lambda \rangle = \langle\mu', \Nm(\lambda)\rangle$.
But $\lambda \in \ker(\Nm : X_*(T) \rightarrow X_*(T))$, so $\langle\mu', \Nm(\lambda)\rangle = 0$.
Therefore $\xi'(\lambda) = z^{\langle\mu, \lambda\rangle} = \xi(\lambda)$.
\end{proof}

We will also need the following lemma in order to define our notion of admissible pair
in \S\ref{section:general_rectifiers}.

\begin{lemma} \label{lem:weyl_groups}
Let $G$ be a connected reductive $K$-group and let $T$ be a maximal
$K$-torus of $G$ that has splitting field $L$.
\begin{enumerate}
\item $\Normalizer{T(K)}{G(L)} / T(L) \cong W(K)$.
\item The action of $\Normalizer{T(L)}{G(L)} / T(L)$ on $T(L)$ determines
actions of $\Normalizer{T(L)}{G(L)}^\Gamma / T(K)$ and $W(K)$
on $T(L)$ which factor naturally to actions on $T(L)_\Gamma$.
\end{enumerate}
\end{lemma}

\begin{proof}
See \cite{adrian-lansky:ppa}*{Lem. 9.1}.
\end{proof}

\section{The relationship between the Gross construction and the DeBacker--Reeder and Reeder construction}
\label{section:gross_debacker_reeder}

Let $\varphi : \Weil_K \rightarrow {}^L G$ be a regular semisimple elliptic Langlands
parameter for an unramified connected reductive group $G$
(see \cite{reeder-debacker:09a} and \cite{reeder:08a}).
Here, ${}^L G = \langle \hat{\theta} \rangle \ltimes \hat{G}$,
where $\hat{\theta}$ is the dual Frobenius automorphism on $\hat{G}$
(see \cite{reeder-debacker:09a}*{\S 3}).
Since $\varphi$ is elliptic, its image is contained in a group of type $L$ for an unramified $K$-minisotropic torus $T$.
Let $L$ be the splitting field of $T$ and let $\Gamma = \Gal(L/K)$ and $\xi_\varphi$ be as in Section \ref{section:groups_of_type_L}.
Then $\varphi(I_K) \subset \hat{T}$ and
$\varphi(\Fr) = \hat{\theta} f$ for some $f \in \hat{N}$.  Let $\hat{w}$
be the image of $f$ in $\hat{W}$.
DeBacker--Reeder \cite{reeder-debacker:09a} and Reeder \cite{reeder:08a}
associate a character $\chi_{\varphi}$ of $T(K)$ to $\varphi$.

We now recall the definition of
the Tits group and some of its properties.  Choose a set $\{ X_{\alpha} \}$ of root vectors
indexed by the set of simple roots of $\hat{T}$ in $\hat{B}$.  Then $(\hat{T}, \hat{B}, \{X_{\alpha} \})$
is a pinning as in \cite{reeder:10a}*{\S 3.1}.
For each simple root $\alpha$, define $\phi_{\alpha} : \SL_2 \rightarrow \hat{G}$
by
\begin{align*}
\phi_{\alpha}\begin{pmatrix}z & 0 \\ 0 & z^{-1}\end{pmatrix} &= \alpha^{\vee}(z) \\
d \phi_{\alpha}\begin{pmatrix}0 & 1 \\ 0 & 0\end{pmatrix} &= X_{\alpha}.
\end{align*}
Let $\sigma_{\alpha} = \phi_{\alpha}\begin{pmatrix}0 & 1 \\ -1 & 0\end{pmatrix}$.

\begin{definition}
  The Tits group $\widetilde{W}$ is the subgroup of $\hat{N}$
  generated by $\{\sigma_{\alpha} \}$ for simple roots $\alpha$.
\end{definition}

For each simple root $\alpha$, let $m_{\alpha} = \sigma_{\alpha}^2 = \alpha^{\vee}(-1)$ and
let $\hat{T}_2$ be the subgroup of $\hat{T}$ generated by the $m_{\alpha}$.

\begin{theorem}{\cite[Proposition 5.2]{adams-vogan:12}}
\begin{enumerate}

\item The kernel of the natural map $\widetilde{W} \rightarrow \hat{W}$
  is $\hat{T}_2$,
\item The elements $\sigma_{\alpha}$ satisfy the braid relations,
\item There is a canonical lifting of $\hat{W}$ to a subset of
  $\widetilde{W}$: take a reduced expression $w = s_{\alpha_1} \cdots s_{\alpha_n}$,
  and let $\tilde{w} = \sigma_{\alpha_1} ... \sigma_{\alpha_n}$.
\end{enumerate}
\end{theorem}
% Can we get a more precise reference, from Tits' article, for the above Theorem??
We remark that the lifting $\hat{W} \rightarrow \widetilde{W}$ is not necessarily a homomorphism,
as shown by the example of $\SL_2$.

\begin{definition} \label{def:phiu}
Given $\hat{u} \in \hat{W}$, let $\tilde{u}$ be its canonical lift to $\widetilde{W}$.
We define a homomorphism $\varphi_{\hat{u}} : \Weil_K \rightarrow {}^L G$ by
\begin{enumerate}
\item $\varphi_{\hat{u}}|_{I_K} \equiv 1$,
\item $\varphi_{\hat{u}}(\Fr) = \hat{\theta} \tilde{u}$.
\end{enumerate}
\end{definition}

By
\S\ref{section:groups_of_type_L}, $\varphi$ and $\varphi_{\hat{w}}$ give rise to characters
$\xi_{\varphi}$ and $\xi_{\varphi_{\hat{w}}}$ of $T(L)_{\Gamma}$ respectively.
%Note that $\varphi_{\hat{u}}$ arises in the theory of real groups as a ``base point.''

The following proposition relates the character $\xi_\varphi$ defined through groups of type $L$
to the character $\chi_\varphi$ constructed by DeBacker--Reeder and Reeder.

\begin{lemma} \label{lem:GDR_compat}
$\xi_{\varphi}$ and $\chi_{\varphi} \circ \Nm$ have the same restriction to $T(\OL)_{\Gamma}$.
\end{lemma}

\begin{proof}
We have the exact sequence
$$1 \rightarrow \HT{-1}(\Gamma, T(L)) \rightarrow T(L)_{\Gamma} \rightarrow T(K)
  \rightarrow \HT{0}(\Gamma, T(L)) \rightarrow 1.$$
Recall that the character $\xi_{\varphi}$ is associated to $\varphi$ by
the local Langlands correspondence for tori (see \S\ref{section:groups_of_type_L}).
Note that the above exact sequence restricts to an exact sequence
$$1 \rightarrow \HT{-1}(\Gamma, T(\OL)) \rightarrow T(\OL)_{\Gamma}
  \rightarrow T(\OK) \rightarrow \HT{0}(\Gamma, T(\OL)) \rightarrow 1.$$
Moreover, by Proposition \ref{prop:T0_cohom_triv}, we have
$\HT{-1}(\Gamma, T(\OL)) = \HT{0}(\Gamma, T(\OL)) = 1$.
Therefore, the map
$$T(\OL)_{\Gamma} \xrightarrow{\Nm} T(\OK)$$
is an isomorphism, so
$\xi_{\varphi}|_{T(\OL)_{\Gamma}}$ factors to a character of
$T(\OK)$ via this isomorphism.  But this is exactly how the character
$\chi_{\varphi}|_{T(\OK)}$ is constructed in \cite{reeder-debacker:09a} and \cite{reeder:08a}.
\end{proof}

In the case that $G$ is semisimple, we can say even more.

\begin{proposition}\label{prop:existenceofrectifier}
If $G$ is semisimple, then $\chi_{\varphi} \circ \Nm = \xi_{\varphi} \otimes \xi_{\varphi_{\hat{w}}}^{-1}$.
\end{proposition}

\begin{proof}
Since $G$ is semisimple, $T(K)$ is compact.  In particular,
$\HT{0}(\Gamma, T(L)) = 0$ by Proposition \ref{prop:vanishing_H0},
so we have the following exact sequence:
$$1 \rightarrow \HT{-1}(\Gamma, T(L)) \rightarrow T(L)_{\Gamma} \rightarrow T(K) \rightarrow 1.$$
Note that $T(K) = T(\OK)$ and thus
$T(\OL)_{\Gamma}$ surjects onto $T(K)$ via the norm map
$\Nm$.  Therefore $\HT{-1}(\Gamma,T(L))$ and
$T(\OL)_{\Gamma}$ together generate $T(L)_{\Gamma}$.  It thus suffices to check that
$\xi_{\varphi} \otimes \xi_{\varphi_{\hat{w}}}^{-1} = \chi_{\varphi} \circ \Nm$
on each of these two subgroups.

Since $\varphi_{\hat{w}}|_{I_K} \equiv 1$, $\xi_{\varphi_{\hat{w}}}$ is trivial on
$T(\OL)_{\Gamma}$ so Lemma
\ref{lem:GDR_compat} implies equality on $T(\OL)_{\Gamma}$.
Equality on $\HT{-1}(\Gamma,T(L))$ is Lemma \ref{lem:toral_modification}.
\end{proof}

We note that for semisimple $G$ we may replace $\tilde{w}$ by another
lift $w'$ of $\hat{w}$ to $\hat{N}$ in the definition of $\varphi_{\hat{w}}$.
In fact, if we define $\varphi'$ by
\begin{align*}
\varphi'|_{I_K} &\equiv 1 \\
\varphi'(\Fr) &= w'
\end{align*}
then Lemma \ref{lem:toral_modification} implies $\xi_{\varphi_{\hat{w}}} = \xi_{\varphi'}$.
We will justify the Tits group lift $\tilde{w}$ in \S\ref{section:BH_compat} for $\GL_n(K)$.

\section{$L$-packets fixed under translation by a character}\label{Q_T}

The general definition of rectifier is complicated by the fact that different
characters of a torus can yield the same $L$-packet.  Consider the following archetypical example.
Let $K = \QQ_3$, $G = \SL_2$ and $T$ be an unramified anisotropic torus in $G$.  There are four depth zero
characters: two admissible and two inadmissible, notions defined below.  Since the two admissible characters are interchanged
by the action of the Weyl group, the corresponding $L$-packets are isomorphic \cite{murnaghan:11}*{\S10}.
In this section we investigate depth zero characters of $T(K)$ that leave the association
$\chi \mapsto \Lpack(\chi)$ of \cite{reeder-debacker:09a} invariant upon translation:
$$\Lpack(\chi) = \Lpack(\alpha\cdot\chi) \mbox{ for all depth zero admissible $\chi$}.$$

\begin{definition} \label{def:admissible}
Let $T$ be a $K$-minisotropic torus, that splits over an unramified
extension $L$ (see \cite{reeder:08a}*{\S3}).  Suppose $\xi$ is a character of $T(L)_\Gamma$.
\begin{enumerate}
\item The pair $(T, \xi)$ is called \emph{admissible} if $\xi$ is not fixed
by any nontrivial element of $W(K)$ (c.f. Lemma \ref{lem:weyl_groups}); we
denote by $P_G(K)$ the set of admissible pairs in $G$ and write $P_{G,T}(K)$ for
the set of characters $\xi$ of $T(L)_\Gamma$ so that $(T, \xi)$ is admissible.
\item We call two admissible pairs $(T, \xi)$ and $(T', \xi')$ \emph{isomorphic} if there
exists a $g \in G(K)$ such that $gT(K)g^{-1} = T'(K)$ and $\xi(t) = \xi'(gtg^{-1})$
for all $t \in T(K)$.
\end{enumerate}
Similarly, we will call a character of $T(K)$ \emph{admissible} if
it is not fixed by any nontrivial element of $W(K)$
(c.f. \cite{reeder-debacker:09a}*{p. 802} and \cite{reeder:08a}*{\S3})
\end{definition}

Note that this definition of admissible pair generalizes
Bushnell--Henniart's notion of admissible pair \cite{bushnell-henniart:10a} in
the case of unramified tori.  Indeed,
if $G = \GL_n$, and $T$ is an elliptic torus in $G$ splitting over
an unramified extension $L/K$, then one can show that
$W(K) = \Gamma$.  In this case, the following are equivalent conditions
on a character $\xi$ of $T(K) = \Lx$:
\begin{enumerate}
\item $\xi$ is fixed by a nontrivial element of $W(K)$,
\item $\xi$ is fixed by a nontrivial subgroup of $\Gamma$,
\item $\xi$ factors through the norm map $\Nm_{L/M}$ for some intermediate field $K \subseteq M \subset L$.
\end{enumerate}

Note that for non-adjoint groups it is not sufficient to consider only reflections.
For example, the depth zero character of the split torus in $\SL_3(\QQ_7)$ inflated from
$$\begin{pmatrix} 3^x & & \\ & 3^y & \\ & & 3^{-x-y} \end{pmatrix} \mapsto \zeta_3^{x + y}$$
is fixed by a 3-cycle in the Weyl group and thus not admissible.

In the next section we will be particularly interested in depth zero characters; write $\hatT$ for the set of
depth zero characters of $T(\OK)$, $\Thadm$ for the admissible
ones and $\Thinadm$ for the inadmissible ones.  Each of these
sets is finite since they may be identified with characters of $T(k)$.

\begin{definition}
Write $Q_T$ for the set of $\alpha \in \hatT$ with the following property:
\begin{itemize}
\item For every $\chi \in \Thadm$ there is a $w \in W(K)$ with $\alpha = \frac{\chi}{w(\chi)}$.
\end{itemize}
\end{definition}

The $\SL_2(\QQ_3)$ example above has $Q_T$ of order two, but $Q_T$ is trivial for most tori.
We spend the rest of this section giving criteria constraining $Q_T$.

\begin{proposition} \label{irr-sub}
The set $Q_T$ is a subgroup of $\hatT$, contained within $\Thinadm$ and stable under the action of $W(K)$.
\end{proposition}
\begin{proof}
If $\alpha \in \Thadm \cap Q_T$ then there is some $w \in W(K)$ with
$\frac{\alpha}{w(\alpha)} = \alpha$, so $\alpha = 1$ which is not admissible.

We now show that $Q_T$ is a group.  Certainly $1 \in Q_T$.  Suppose
$\alpha, \alpha' \in Q_T$ and $\chi \in \Thadm$.  Then there are $w, w' \in W(K)$ with
\begin{align*}
\frac{\chi}{w(\chi)} &= \alpha, \\
\frac{w(\chi)}{w'(w(\chi))} &= \alpha'.
\end{align*}
Multiplying the two relations yields $\frac{\chi}{w'w(\chi)} = \alpha\alpha'$, so
$\alpha\alpha' \in Q_T$.  We finish by noting that $Q_T$ is finite and thus closure under
multiplication implies closure under inversion.

Finally, suppose $\tau \in W(K)$.  Given $\chi \in \Thadm$ with $\alpha = \frac{\chi}{w(\chi)}$ we have
$$\tau(\alpha) = \frac{\tau(\chi)}{\tau w(\chi)} = \frac{\tau(\chi)}{w' \tau(\chi)}$$
for some $w' \in W(K)$.  Since $\tau$ permutes the admissible characters we get that $\tau(\alpha) \in Q_T$.
\end{proof}

The condition on $\alpha \in Q_T$ is an extremely stringent one, and an abundance of admissible
characters will preclude a nontrivial $\alpha$.  We can make this statement precise:

\begin{proposition} \label{pigeonhole}
Suppose $\#\Thadm > (\# W(K) - 1) \cdot \# \Thinadm$.  Then $Q_T = \{ 1 \}$.
\end{proposition}
\begin{proof}
Suppose $\alpha \in Q_T$.  For $w \in W(K)$, set
$$S_w = \{\chi \in \Thadm \st  \frac{\chi}{w(\chi)} = \alpha\}.$$
Note that if $S_1$ is nonempty then we get $\alpha = 1$ immediately, so we may
assume the contrary.  Then by the pigeonhole principle, there is a $w \in W(K)$
with $\# S_w > \# \Thinadm$.  Pick $\chi \in S_w$; since $\# S_w > \#\Thinadm$
there is some $\chi' \in S_w$ with $\frac{\chi}{\chi'}$ admissible.  We now have
$$\frac{\chi}{w(\chi)} = \alpha = \frac{\chi'}{w(\chi')}$$
and therefore $\frac{\chi}{\chi'}$ is fixed by $w$.  Since $\frac{\chi}{\chi'}$ is admissible, we must have $w = 1$ and thus
$$\alpha = \frac{\chi}{\chi} = 1.$$
\end{proof}

Recall that Frobenius acts on $X^*(T)$ via an endomorphism $F = qF_0$, where
$F_0$ is an automorphism of finite order \cite{carter:93a}*{p. 82}.  So it makes sense
to vary $q$: we fix $F_0$ and consider the tori dual to the $\Gal(\Fqb/\Fq)$-modules
with Frobenius acting through $qF_0$.

\begin{corollary}[{cf \cite{carter:93a}*{Lem. 8.4.2}}]
Consider the family of tori $T_q$ with the same $F_0$.  Then for sufficiently large $q$,
$Q_{T_q} = \{ 1 \}$ (regardless of the $G$ in which $T_q$ is embedded).
\end{corollary}
\begin{proof}
We will write $T$ for a general torus in the family and $r$ for the common dimension.
Note that $\hatT$ is the set of $\Fq$ points of a dual torus, also of rank $r$ over $\Fq$.
For $w \in W(K)$ with $w \ne 1$ the centralizer $\Z_{\hatT}(w)$ is a proper $F$-stable
subgroup of $\hatT$, and thus $\dim(\Z_{\hatT}(w)) \le r - 1$.  By \cite{carter:93a}*{3.3.5},
$\# \hatT$ is a polynomial in $q$ of degree $r$ and $\# \Z_{\hatT}(w)$ is a polynomial
in $q$ of degree at most $r-1$.  Thus the ratio
$$\frac{\# \Thadm}{\# \Thinadm} = \frac{\# \hatT - \sum_{1 \ne w \in W} \# \Z_{\hatT}(w)}{\sum_{1 \ne w \in W} \# \Z_{\hatT}(w)}$$
grows without bound as $q$ does.  There are finitely many possibilities for the absolute
Weyl group of $T$, so Proposition \ref{pigeonhole} gives the desired result.
\end{proof}

In computing $Q_T$ for small $q$ the following result is useful:

\begin{proposition} \label{orderdiv}
If $\alpha \in Q_T$ has order $d$ and $\chi \in \Thadm$ has order $m$ then $d$ divides $m$.
\end{proposition}
\begin{proof}
There is a $w \in W(K)$ with
$$\frac{\chi}{w(\chi)} = \alpha.$$
Since $w(\chi)$ also has order $m$, raising both sides to the $m$th power  yields $\alpha^m = 1$.
\end{proof}

Finally, we note that Lemma 8 of Bushnell--Henniart \cite{bushnell-henniart:10a}*{p. 511} is equivalent to
the statement that $Q_T$ is trivial when $T$ is a $K$-minisotropic torus in $\GL_n$.

\section{Rectifiers for general reductive groups} \label{section:general_rectifiers}

Suppose that $G$ is a connected reductive group defined over a
$p$-adic field $K$.  Fix an unramified $K$-torus $T \subset G$ with splitting field $L$.
Let $\varphi : \Weil_K \rightarrow {}^L G$ be a
Langlands parameter for $G(K)$, and suppose that $\varphi$ factors
through a group of type L for $T$.  Any Langlands parameter with image in the normalizer
of a maximal torus will factor in this way for some $T$.

As in \S\ref{section:groups_of_type_L}, one can canonically
associate to $\varphi$ a character $\xi_{\varphi}$ of $T(L)_{\Gamma}$.
Recall again the Tate cohomology sequence
$$1 \rightarrow \HT{-1}(\Gamma,T(L)) \rightarrow T(L)_{\Gamma} \xrightarrow{\Nm} T(K)
= T(L)^{\Gamma} \rightarrow \HT{0}(\Gamma,T(L)) \rightarrow 1.$$
Suppose that $\HT{0}(\Gamma, T(L)) = 0$, in which case
$T(L)_{\Gamma}$ surjects onto $T(K)$.  Let us also suppose that
$\varphi$ does not factor through a proper Levi subgroup, so that the
representations in the $L$-packet associated to $\varphi$ are
conjecturally all supercuspidal (see \cite{reeder-debacker:09a}*{\S 3.5}).
When $G = \GL_n$ we show in \S\ref{section:BH_compat} that
$\HT{0}(\Gamma, T(L)) = \HT{-1}(\Gamma, T(L)) = 0$ and thus
$T(L)_{\Gamma} \cong T(K) \cong \Lx$.  In this case
$(L/K, \xi_{\varphi})$ is an admissible pair; to construct the local Langlands
correspondence one proceeds as in \S\ref{section:BH_recall} by
attaching the supercuspidal representation $\pi_{\xi_{\varphi} \cdot
  \mu_{\xi_{\varphi}}}$ to $\xi_{\varphi}$, via the construction of Bushnell and Henniart.

For other groups $G$ there are some constructions of supercuspidal $L$-packets $\Lpack(\chi)$
from characters $\chi$ of $T(K)$ \cites{reeder-debacker:09a, kaletha:13a, reeder:08a}.
However, as we have seen, a Langlands parameter $\varphi$ does not naturally
provide a character of $T(K)$, but rather a character of
$T(L)_{\Gamma}$.  We consider here the maps $\chi \mapsto \Lpack(\chi)$ of \cites{reeder-debacker:09a, reeder:08a}.

\begin{definition} \label{def:rectifier}
  Let $T$ be a coverable torus in $G$ with unramified splitting field $L$.  Let $\Adm: P_G(K) \to \Sup(G)$ be a construction of supercuspidal representations from admissible pairs.  Let $\LP_T$ be a set of Langlands parameters for $G$ that factor through the normalizer of $\hat{T} \subset {}^LG$, and let $\varphi \mapsto \Lpack(\varphi)$ be an association of supercuspidal L-packets $\Lpack(\varphi) \subset \Sup(G)$ to $\varphi \in \LP_T$.  Then a \emph{rectifier} for $T$ is a function
  \[
   \bmu_{T, \Adm, \Lpack} : \xi \mapsto \mu_{\xi}
  \]
  which attaches to each $\xi \in P_{G,T}(K)$ a character
  $\mu_{\xi}$ of $T(L)_{\Gamma}$ satisfying the following conditions:

\begin{enumerate}
\item The character $\mu_{\xi}$ is tamely ramified (i.e. trivial on
  $T(\PL)_{\Gamma}$),

\item The character $\xi \cdot \mu_{\xi}$ descends to $T(K)$ and $\xi \cdot \mu_{\xi} \in P_{G,T}(K)$.
\item For all $\varphi \in \LP$ we have $\Adm(T, \xi_{\varphi} \cdot \mu_{\xi_{\varphi}}) \in \Lpack(\varphi)$,

\item If $\xi_1, \xi_2 \in P_{G,T}(K)$ with
$\xi_1^{-1} \xi_2$ tamely ramified then
$\mu_{\xi_1} = \mu_{\xi_2}$.
\end{enumerate}
We say that two rectifiers $\bmu_{T, \Adm, \Lpack}$ and $\bmu_{T, \Adm, \Lpack}'$ are \emph{equivalent}
if there is some $\alpha \in Q_T$ so that
\begin{align*}
\mu_\xi' &= \alpha \mu_\xi \qquad \mbox{for depth zero $\xi$,} \\
\mu_\xi' &= \mu_\xi \qquad\ \  \mbox{for positive depth $\xi$.}
\end{align*}

\end{definition}

Since $T$ is coverable, the condition that $\xi \cdot \mu_\xi$
descends to $T(K)$ is equivalent to $\xi \cdot \mu_\xi$ vanishing on $\HT{-1}(\Gamma, T(L))$.  The
notion of equivalence is tailored for Theorem \ref{thm:unique_semisimple}; for some tori (such as the
$\SL_2(\mathbb{Q}_3)$ example at the beginning of \S \ref{Q_T}) there are
multiple equivalent rectifiers.

We first apply this notion of rectifier to the setting of DeBacker--Reeder \cite{reeder-debacker:09a} and Reeder \cite{reeder:08a}.
For an unramified coverable torus $T$, let $

%\begin{conjecture} \label{conj:unique_rectifier}
%For $T$ as in Definition \ref{def:rectifier}, $T$ admits a unique rectifier up to equivalence.
%\end{conjecture}

%We note that, as the local Langlands correspondence is not known in general, we must restrict
%ourselves to cases where supercuspidal $L$-packets have been constructed.
%Since we are in the present paper considering the situation when $T$ is unramified,
%we consider those $L$-packets constructed in \cite{reeder-debacker:09a} and \cite{reeder:08a}.
%In the setting of Reeder \cite{reeder:08a},
%we must further restrict our scope since his constructions do not apply to all admissible pairs.

\begin{definition}\label{def:general_pair}
Suppose $(T, \xi) \in P_G(K)$.
\begin{enumerate}
\item The \emph{depth} of $(T, \xi)$ is the integer $r$ so that $\xi$
is trivial on $T(\PL^{r+1})_{\Gamma}$ but nontrivial on
$T(\PL^{r})_{\Gamma}$
\item An admissible pair of depth $r$ is \emph{minimal}
if $\xi|_{T(\PL^{r})_{\Gamma}}$
is not fixed by any element of $W(K)$.
We denote by $\Pmin(K)$ the set
of minimal admissible pairs in $G$.
\item A \emph{weak rectifier} for $T \subset G$ is a function
\begin{align*}
\mumin : (T, \xi) \mapsto \mu_{\xi}
\end{align*}
which attaches to each $(T, \xi) \in \Pmin(K)$ a character
  $\mu_{\xi}$ of $T(L)_{\Gamma}$, satisfying conditions (1)-(3)
  of Definition \ref{def:rectifier}.
\end{enumerate}
We define equivalence of weak rectifiers as in Definition \ref{def:rectifier}.
\end{definition}

We note that this definition of minimal admissible pair generalizes
the definition of minimal admissible pair of Bushnell and Henniart in
the case of unramified tori (see \cite{bushnell-henniart:05a}*{\S2.2}).  We also note that the character of a minisotropic torus that is produced from a Langlands parameter considered in \cite{reeder:08a} can be seen to automatically be minimal.

\begin{theorem} \label{thm:unique_semisimple}
For $G$ unramified and semisimple, and $T$ as in Definition \ref{def:rectifier}, $T$ admits a unique weak rectifier up to equivalence.
\end{theorem}

\begin{proof}
We first prove existence.  We defined in
\S\ref{section:gross_debacker_reeder}
a Langlands parameter $\varphi_{\hat{w}} :\Weil _K \rightarrow {}^L G$ by
sending Frobenius to the canonical lift
$\tilde{w} \in \widetilde{W}$ of $\hat{w} \in \hat{W}$, and by setting
$\varphi_{\hat{w}}$ to be trivial on $I_K$.
For semisimple $G$
we proved in Proposition \ref{prop:existenceofrectifier}
that the function $$(T, \xi) \mapsto \xi_{\varphi_{\hat{w}}}^{-1}$$
satisfies condition (2) of Definition \ref{def:rectifier}.  Moreover, the function also
satisfies condition (1): $\varphi_{\hat{w}}|_{I_K} \equiv 1$ and thus
$\xi_{\varphi_{\hat{w}}}^{-1}$ is unramified.
Finally, $\xi_{\varphi_{\hat{w}}}$ is independent of $\xi$
and thus condition (3) is automatically satisfied.
We may therefore set
$\mumin(T,\xi) = \xi_{\varphi_{\hat{w}}}^{-1}$.

We now prove uniqueness.
Let $\xi$ range over the set of characters of $T(L)_{\Gamma}$
such that $(T, \xi) \in \Pmin(K)$, and let
$\bmu$ and $\bmu'$ be weak rectifiers for
$T \subset G$.  By hypothesis, we have
$$\Lpack(\mu_{\xi} \cdot \xi) = \Lpack(\mu'_{\xi} \cdot \xi).$$
By \cite[Proposition 10.8]{murnaghan:11}, there exists $w_{\xi} \in W(K)$,
depending on $\xi$, such that
$${}^{w_{\xi}} (\mu_\xi \cdot \xi) = \mu'_\xi \cdot \xi.$$
Suppose that $\xi$ has positive depth.
Restricting the equation
${}^{w_\xi} (\mu_\xi \cdot \xi) = \mu'_\xi \cdot \xi$
to $T(\PL)_{\Gamma}$, we get that ${}^{w_{\xi}} (\xi) = \xi$,
by condition (1) of Definition \ref{def:rectifier}.
Since $\xi$ is minimal, we get that $w_{\xi} = 1$, which implies
that $\mu_{\xi} = \mu'_{\xi}$.

Now suppose that $\xi$ has depth zero, and suppose without loss of generality that $\bmu = \mumin$. Define $\lambda$ on $T(\OL)_{\Gamma} \cong T(\OK)$ by $\lambda = (({}^{w_\xi} (\mu_\xi))^{-1} \cdot \mu'_\xi)|_{T(\OL)_{\Gamma}}$.  We claim that $\lambda$ is independent of $\xi$.  To see this, let $\xi_1, \xi_2$ be depth zero characters.  We wish to show that
\begin{equation}\label{equivalencedepthzero}
(({}^{w_{\xi_1}} (\mu_{\xi_1}))^{-1} \cdot \mu'_{\xi_1})|_{T(\OL)_{\Gamma}} =
(({}^{w_{\xi_2}} (\mu_{\xi_2}))^{-1} \cdot \mu'_{\xi_2})|_{T(\OL)_{\Gamma}}.
\end{equation}
But since $\xi_1, \xi_2$ are depth zero, we have that $\xi_1 \xi_2^{-1}$ is also of depth zero, and so condition (3) implies that $\mu'_{\xi_1} = \mu'_{\xi_2}$. Therefore, \eqref{equivalencedepthzero} is equivalent to
\begin{equation}\label{equivalencedepthzero2}
({}^{w_{\xi_1}} (\mu_{\xi_1}))|_{T(\OL)_{\Gamma}} = ({}^{w_{\xi_2}} (\mu_{\xi_2}))|_{T(\OL)_{\Gamma}}.
\end{equation}
Since $\bmu = \mumin$, we have that $\mu_{\xi_1}|_{T(\OL)_{\Gamma}} \equiv \mu_{\xi_2}|_{T(\OL)_{\Gamma}} \equiv 1$. Finally, since $W(K)$ preserves $T(\OL)_{\Gamma}$, \eqref{equivalencedepthzero2} is proven since both sides of the equality are the trivial character.

Since $\lambda$ is seen to be independent of $\xi$, the equation ${}^{w_\xi} (\mu_\xi \cdot \xi) = \mu'_\xi \cdot \xi$
implies that $\lambda \in Q_T$.  Since $\mu_\xi \cdot \xi$ and $\mu_\xi' \cdot \xi$ descend to
$T(K)$ by condition (2) of Definition \ref{def:rectifier}, $\mu_\xi$ and $\mu_\xi'$ have the
same restriction to $\hat{H}^{-1}(\Gamma, T(L))$.  Since $G$ is semisimple we may pull $\lambda$ back to
a character on $T(L)_\Gamma$, vanishing on $\HT{-1}(\Gamma, T(L))$.  Note also that
\[
\mu_{\xi}'|_{T(\OL)_{\Gamma}} \equiv (\lambda \mu_{\xi})|_{T(\OL)_{\Gamma}}.
\]
by definition of $\lambda$ and the fact that $\mu_\xi|_{T(\OL)_{\Gamma}} \equiv 1$ and that $W(K)$ preserves $T(\OL)_{\Gamma}$.  Therefore, we finally obtain that $\mu'_\xi = \lambda\mu_\xi$ on all of $T(L)_{\Gamma}$
and thus $\bmu$ is equivalent to $\bmu'$.
\end{proof}

\begin{remark} $ $
\begin{enumerate}
\item The condition $\HT{0}(\Gamma, T(L)) = 0$ was necessary in order to obtain a character on $T(K)$ rather
than the image of the norm map $T(L) \mapsto T(K)$.  For non-semisimple groups where $\HT{0}(\Gamma, T(L))$
is nontrivial we hope that the recipe for the central character in \cite{gross-reeder:09a} will provide an extension to all of $T(K)$.
\item The rectifier in our setting is constant as a function of $\xi$.  We expect a dependence on $\xi$ for ramified tori.
\item The behavior of rectifiers under change of group is not yet clear to us.  There may be a natural relationship between
rectifiers when a torus is embedded into two
different reductive groups with isomorphic Weyl groups.  Similarly, when given an embedding
$H \subset G$, a natural relationship between the rectifiers for tori in $H$ and $G$ would allow us to apply the
results of \cite{bushnell-henniart:10a} to rectifiers for general groups.
\end{enumerate}
\end{remark}

\section{Compatibility with Bushnell--Henniart} \label{section:BH_compat}

In this section we show that our function $\mumin$
agrees with the rectifier of Bushnell--Henniart in the depth
zero setting: see Theorem \ref{thm:bh_agreement}.
Let $L = K_n$ and set
$T = \Res_{L/K}(\Gm)$.  We begin by computing the Tate cohomology groups of $T$.

\begin{proposition}
$\HT{0}(\Gamma, X_*(T)) = 0$.
\end{proposition}

\begin{proof}
Since $\Gamma$ acts on $X_*(T)$ by permuting basis vectors,
$X_*(T)^{\Gamma}$ is the copy of $\ZZ$ embedded diagonally in
$X_*(T) = \mathbb{Z}^n$.  Note that $$\Nm(1,0,0,\cdots,0) = (1,1,\cdots,1),$$ so
$X_*(T)^{\Gamma} \subset \Nm(X_*(T))$.
\end{proof}

\begin{proposition}
$\HT{-1}(\Gamma, X_*(T)) = 0$.
\end{proposition}

\begin{proof}
We note that $(a_1, a_2, \cdots, a_n) \in \ker(\Nm)$ if and only if $\sum_{i=1}^n a_i = 0$.
It is then easy to see that $\ker(\Nm)$ is generated by $e_i - e_j$ for $i < j$, where
$e_i$ are the standard basis of $\mathbb{Z}^n$.  But $e_i - e_j = (1 - \tau)e_i$ for some
$\tau \in \Gamma$, since $\Gamma$ acts by cyclic shift.  Thus $\ker(\Nm) \subset I_{\Gamma}(X_*(T))$.
\end{proof}

The Tate cohomology exact sequence for $T$ therefore reduces to
$$1 \rightarrow T(L)_{\Gamma} \xrightarrow{\sim} T(K) \rightarrow 1$$ by
Corollary \ref{cor:cohom_tori}.  We now need a basic result about powers of lifts
of Coxeter elements in $\GL_{n}(\CC)$.

\begin{proposition}\label{prop:powers_of_lifts}
Let $\hat{w}$ be a Coxeter element of $\GL_{n}(\CC)$, and let $\tilde{w}$ be the
canonical lift of $\hat{w}$ to $\widetilde{W}$. Then $\tilde{w}^n = (-1)^{n-1}$ as
as scalar matrix in $\GL_{n}(\CC)$.
\end{proposition}

\begin{proof}
See \cite{zaremsky:ppa}*{\S3.1}.
\end{proof}

We can now describe the image of $\mumin$ in the setting of
depth zero supercuspidal representations of $\GL_{n}(K)$.  Write
$\varphi$ for $\varphi_{\hat{w}}$ (see Definition \ref{def:phiu}) and $\mu$ for $\xi_{\varphi}^{-1}$ .

\begin{proposition} \label{prop:rectifier_agreement}
$\mu$ is unramified and
$\mu(\varpi) = (-1)^{n-1}$.
\end{proposition}

\begin{proof}
Let $\sigma$ generate $\Gal(L/K)$.  Then $T(L) \cong \Lx \times \Lx \times \cdots \times \Lx$ and
$$T(K) = \{(x, \sigma(x), \sigma^2(x), \cdots, \sigma^{n-1}(x)) : x \in \Lx \} \cong \Lx.$$
A uniformizer $\varpi$ in $\Kx \subset \Lx$
therefore corresponds to $(\varpi, \varpi, \cdots, \varpi) \in T(K)$, whose
preimage under $\Nm$ is the class of $(\varpi, 1, 1, \cdots, 1)$ in $T(L)_{\Gamma}$.
By \cite{serre:LocalClassFieldThy}*{\S 2.4}, $\varpi$ corresponds to $\Fr^n$ under the Artin
reciprocity map for $L$.  Now by Proposition \ref{prop:powers_of_lifts}
and the local Langlands correspondence for tori we get
$\mu(\varpi) = (-1)^{n-1}$.
Finally, $\varphi|_{I_K} \equiv 1$ implies that $\mu$ is
unramified.
\end{proof}

\begin{theorem} \label{thm:bh_agreement}
  If $G = \GL_{n}(K)$ and fixed $T$, the constant function $(T,\xi) \mapsto \mu$ agrees with
  the rectifier of Bushnell--Henniart for depth zero $\xi$.
\end{theorem}

\begin{proof}
This result follows from Proposition \ref{prop:rectifier_agreement} and Proposition \ref{prop:BH_result1}.
\end{proof}

We end this section by explaining why the Tits group lift $\tilde{w}$ is forced upon us.
Suppose we define
$\varphi' : \Weil_K \rightarrow \GL_{n}(\CC)$ by $\varphi'|_{I_K} \equiv 1$ and
$\varphi'(\Fr)$ to be a lift of an elliptic element $\hat{w}$ in $\hat{W}$.
Then \cite{reeder-debacker:09a}*{p. 824} and \cite{reeder:08a}*{\S6} imply that the characteristic
polynomial of $\varphi'(\Fr)$ is $X^n - a$, for some $a \in \CCx$.  One can see that,
by arguments analogous to those in Proposition \ref{prop:rectifier_agreement},
$\xi_{\varphi'}(\varpi) = a$.  By Proposition \ref{prop:BH_result1}, we are
forced to set $a = (-1)^{n-1}$.  Finally, one can show by an inductive argument that the
canonical lift $\tilde{w}$ of $\hat{w}$ to $\widetilde{W}$ has characteristic polynomial $X^n - (-1)^{n-1}$,
so that $\varphi'(\Fr)$ is indeed the canonical lift of $\hat{w}$ to $\widetilde{W}$ up to conjugacy.

%\bibliography{Biblio}
\bibliography{roebib/Biblio,extras}

\end{document}
