\documentclass[11pt]{amsart}
\usepackage{amsmath,amscd,amssymb,latexsym, amsfonts}
\theoremstyle{plain}
\newtheorem{theorem}[enumi]{Theorem}
\newtheorem{corollary}[enumi]{Corollary}
\newtheorem{proposition}[enumi]{Proposition}
\newtheorem{lemma}[enumi]{Lemma}
\newtheorem{Main}{Main Theorem}
\newtheorem{definition}[enumi]{Definition}

\newcommand{\MAxxx}[1]{$\clubsuit$\footnote{#1}}
\newcommand{\DRxxx}[1]{$\spadesuit$\footnote{#1}}

\renewcommand{\theMain}{}
\setlength{\oddsidemargin}{0.2in}
\setlength{\evensidemargin}{0.2in}
\setlength{\textwidth}{6.1in}

\DeclareMathOperator{\Gal}{Gal}
\newcommand{\mat}[4]{\left( \begin{array}{cc} {#1} & {#2} \\ {#3} & {#4}
\end{array} \right)}

\begin{document}
\title{Rectifiers and the Tame Local Langlands Correspondence}
%\author{Gordan Savin}
%\address{Department of Mathematics, University of Utah, Salt Lake City, UT 84112, U.S.A.}

%\email{savin@math.utah.edu}




\maketitle

\section{Abstract}

In this note we compute the cohomology groups of $p$-adic tori.  This paper is motivated by recent work of Benedict Gross in which he gives a natural framework for understanding a general class of Langlands parameters for $p$-adic groups.  In his work, one is required to understand the cohomology groups of tori, hence the reason for our work here.

\section{Introduction}
In this paper, we propose a natural generalization of the notion of a rectifier.  In particular, we define rectifiers for Langlands parameters of arbitrary connected reductive groups that map into a ``group of type L''.
We then describe these rectifiers via their associated Langlands parameters, and show that they are compatible with the rectifiers of $GL(n,F)$ as in \cite{bushnellhenniart}.

An irreducible smooth representation of the Weil group $W_F$ of $F$ is called \emph{essentially tame} if its restriction to wild inertia is a sum of characters.  Let us recall the classical construction of the essentially tame local Langlands correspondence for $GL(n,F)$.  In the essentially tame case, an irreducible representation $\varphi : W_F \rightarrow GL(n,\mathbb{C})$ naturally provides an \emph{admissible pair} $(E/F, \xi)$.  Here, $E/F$ is a degree $n$ separable extension and $\xi$ is a character (with certain conditions) of $E^{\times}$, which one can view as a maximal elliptic torus in $GL(n,F)$.  Howe constructs a map (see \cite{howe})
\begin{equation*}
\left\{
\begin{array}{ll}
isomorphism \ classes \ of \\
admissible \ pairs
\end{array}
\right\} \rightarrow \left\{
\begin{array}{ll}
supercuspidal \ representations \\
of \ GL(n,F)
\end{array} \right\}
\end{equation*}
$$\hspace{-.5in} (E/F, \xi) \mapsto \pi_{\xi}$$
The problem is that the obvious map, $$\varphi \mapsto \pi_{\xi},$$ the so-called ``naive correspondence'', is not the local Langlands correspondence because $\pi_{\xi}$ has the wrong central character.  Instead, the local Langlands correspondence is given by $$\varphi \mapsto \pi_{\xi \cdot {}_F \mu_{\xi}}$$ for some subtle finite order character ${}_F \mu_{\xi}$ of $E^{\times}$.  The function $${}_F \boldsymbol\mu : (E/F, \xi) \mapsto {}_F \mu_{\xi}$$ is called the \emph{rectifier} associated to $E/F$.  Both the description of and the intuition behind the rectifiers ${}_F \boldsymbol\mu$ have been studied (see \cite{bushnellhenniart}, \cite{geo}, \cite{adrian}).

One may ask whether rectifiers exist for groups apart from $GL(n,F)$, and even moreso, how does one define the notion of rectifier for groups apart from $GL(n,F)$.  The answer can be gleamed from a recent construction of Benedict Gross, which we briefly describe now (for more details, see \S\ref{groupsoftypeL}).

Suppose that $G$ is a connected reductive group defined over a $p$-adic field $F$.  Let
$\phi : W_F \rightarrow {}^L G$ be a Langlands parameter for $G(F)$, and suppose that $\phi$ factors through the
normalizer of a maximal torus. To $\phi$, one can associate a maximal $F$-torus $T$ in $G$, the twisted $F$-torus associated to $\phi(\Phi)$, where $\Phi$ is the geometric Frobenius \MAxxx{geometric or arithmetic?  does it even make sense to say ``the geometric Frobenius of $W_F$''?} of $W_F$.  Suppose that $T$
splits over the separable \MAxxx{separable or Galois?  What about the non-Galois case for $GL(n)$?  Then we wouldn't be able to define $\Gamma = Gal(E/F)$??} extension $E$ of $F$, and set $\Gamma = \Gal(E/F)$.  Restricting $\phi$ to $W_E$, by the local Langlands correspondence for tori,
one can canonically associate to $\phi$ a character $\xi$ of $T(E)_{\Gamma}$,
the group of coinvariants of $T(E)$ with respect to $\Gamma$ (see~\S\ref{groupsoftypeL}).  We will denote the association of $\xi$ to $\phi$ by the map $$G : \phi \mapsto \xi.$$
Invariants and coinvariants are related by the norm map
$$N : T(E) \rightarrow T(F)$$ $$t \mapsto \displaystyle\prod_{\sigma \in \Gamma} \sigma(t)$$
in the cohomology sequence
$$1 \rightarrow \hat{H}^{-1}(\Gamma,T(E)) \rightarrow T(E)_{\Gamma} \xrightarrow{N} T(F)
= T(E)^{\Gamma} \rightarrow \hat{H}^0(\Gamma,T(E)) \rightarrow 1,$$ where $\hat{H}$ denotes Tate cohomology.

Suppose that $\hat{H}^0(\Gamma, T(E)) = 0$, in which case $T(E)_{\Gamma}$ is then a cover of $T(F)$.  Let us also suppose that $\phi$ does not factor through a proper Levi subgroup, so that the representations in the $L$-packet associated to $\phi$ are conjecturally all supercuspidal (see \cite[\S 3.5]{debackerreeder}).  If $G$ happens to be $GL(n)$, one can compute that $\hat{H}^{0}(\Gamma, T(E)) = \hat{H}^{-1}(\Gamma, T(E)) = 0$, so that $T(E)_{\Gamma} \cong T(F)$ can be identified with $E^{\times}$, and $(E/F, \xi)$ is an admissible pair.  To construct the local Langlands correspondence, one would then proceed (as explained earlier) to attach the supercuspidal representation $\pi_{\xi \cdot {}_F \mu_{\xi}}$ to $\xi$, via the construction of Howe.

If $G$ were arbitrary, then in analogy to the case of $GL(n,F)$, there exist in certain cases constructions of supercuspidal $L$-packets $L(\xi)$ associated to characters $\xi$ of tori $T(F)$ in $G(F)$ (see \cite{debackerreeder}, \cite{kaletha}, \cite{reeder}).  However, as we have seen, a Langlands parameter $\phi$ as above does not naturally provide a character of $T(F)$, but rather a character of $T(E)_{\Gamma}$.

\begin{definition}\label{rectifierdefinition}
Let $T$ be a torus, defined over $F$, that splits over an unramified extension $E$.  A \emph{rectifier} for $T$ is a function $${}_F \boldsymbol\mu : (T, \xi) \mapsto {}_F \mu_{\xi}$$
which attaches, to each (admissible pair?)\MAxxx{Define notion of regular pair via Adler?  Look in Reeder's positive depth paper.  Adler I think might only be positive depth, so we would have to make a definition that includes characters of tori that are needed in depth zero supercuspidals} $(T, \xi)$, a character ${}_F \mu_{\xi}$ of $T(E)_{\Gamma}$ satisfying the following conditions.

(1) $\xi {}_F \mu_{\xi} \in \widehat{T(F)}$

(2) The character ${}_F \mu_{\xi}$ is unramified (i.e. trivial on $T(\mathfrak{o}_E)_{\Gamma}$)

(3) $\phi \mapsto \xi \mapsto L(\xi {}_F \mu_{\xi})$ is the local Langlands correspondence.

(4) If $(T, \xi_i), i = 1,2$ are (admissible pairs?) \MAxxx{Fix this when we know how.} such that $\xi_1^{-1} \xi_2$ is unramified, then ${}_F \mu_{\xi_1} = {}_F \mu_{\xi_2}$.
\end{definition}

We prove:

\begin{theorem} Let $T$ be as in Definition \ref{rectifierdefinition}.  Then $T$ admits a unique rectifier ${}_F \boldsymbol\mu : (T, \xi) \mapsto {}_F \mu_{\xi}$.
\end{theorem}

Once phrased in this manner, uniqueness is a straightforward matter (see \S\ref{}) \MAxxx{Fix when we can.}.  The existence of a rectifier will be proven by interpreting the rectifier in terms of a canonical Langlands parameter.  More explicitly, we will exhibit a canonical \MAxxx{Canonical?  Is the Tits group lift canonical?} Langlands parameter $\phi_{o}$, such that if $\xi_o = G(\phi_o)$, then the function $$(T, \xi) \mapsto \xi_o$$ is a rectifier.

Let $\phi, T, E, \xi$ as before, assuming that $\phi$ is TRSELP. \MAxxx{Do we need to assume tame?  That is, does our paper work for Reeder's positive depth paper as well or not?  I think the $\phi_o$ we have works for Reeder's paper as well.  That is, we don't need to change $\phi_o$ on inertia. The same proofs work.  If so, let's change TRSELP to RSELP.}  The projection of $\phi(\Phi)$ onto the connected component of ${}^L G$ is a lift of some Weyl group element $\hat{w}$ to the normalizer of the dual torus.  Associated to $\phi$, DeBacker/Reeder construct a character $\chi_{\phi}$ of $T(F)$, to which they associate a conjectural $L$-packet of representations $L(\chi_{\phi})$ (see \S \ref{preliminaries}).

\begin{definition}
Given $\phi$, we define a homomorphism $\phi_o : W_F \rightarrow {}^L G$ by
\begin{enumerate}
\item $\phi_o|_{I_F} \equiv 1$
\item $\phi_o(\Phi) = \hat{\theta} \tilde{w}$
\end{enumerate}
\end{definition}

where $\hat{\theta}$ is the geometric \MAxxx{geometric or arithmetic?  or neither?} Frobenius element generating the action of Galois on the dual group, and where $\tilde{w}$ is the canonical lift \MAxxx{Is this lift really canonical?} of $\hat{w}$ to the Tits group (see \S\ref{mainresults}). Let $\xi_o = G(\phi_o)$.  Our main theorem is the following.

\begin{theorem}
If $G$ is semisimple, then $${}_F \boldsymbol\mu : (T, \xi) \mapsto \xi_o$$ is a rectifier for $T$.
\end{theorem}

In the case that $G$ is semisimple, there was nothing special about $\phi_o$.  In fact, by the same arguments that we will make, a rectifier is obtained from any Langlands parameter $\phi'$ such that $\phi'|_{I_F} \equiv 1$ and $\phi'(\Phi) = w'$, where $w'$ was any lift of $\hat{w}$ to the normalizer of the dual torus.  However, the following theorem (which we will prove) will show that $\phi_o$ is forced upon us. \MAxxx{Up to a sign, or something like that?}

\begin{theorem}
If $G = GL(n,F)$, the character ${}_F \boldsymbol\mu$ agrees with the rectifier of Bushnell/Henniart.  Moreover, $\phi_o$ is the unique \MAxxx{up to a sign, or something like that} Langlands parameter such that $G(\phi_o) = {}_F \mu_{\xi}$.
\end{theorem}

We would like to note that in the case that $\hat{H}^0(\Gamma, T(E)) \neq 0$, the situation seems more difficult $\xi$ is defined on the image of $T(E)_{\Gamma}$ under the norm map.  However, one might be able to remedy this with a prediction of central character, as in \cite{grossreeder}, for example.

Acknowledgements: Part of this paper was heavily influenced by conversations with Gordan Savin.  We wish to thank him for these conversations.  We would also like to thank Jeffrey Adams and Geo Kam Fai for helpful conversations as well.
\section{Notation and Definitions}

Throughout, $\mathfrak{f}$ will denote a finite field, and $\mathfrak{f}_n$ will denote the degree $n$ extension of $\mathfrak{f}$.  $k$ will denote a nonarchimedean local field of charcteristic zero, and $\mathfrak{f}$ will be its residue field.  Fix a uniformizer $\varpi$ of $k$.  Let $k^u$ denote the maximal unramified extension of $k$, and fix a valuation $val : (k^u)^* \rightarrow \mathbb{Z}$ normalized so that $val(\varpi) = 1$.

\section{Preliminaries}

If $S$ is a torus that is defined over an arbitrary field $F$ and $E/F$ is a Galois extension, then we have the norm map $$N_{E/F} : S(E) \rightarrow S(F)$$ $$\ \ \ \ \ \ \ \ \ \ \ \ \ \ \ \ \ s \mapsto \prod_{\sigma \in Gal(E/F)} \sigma(s)$$

\section{Tori over finite fields}

Let $\mathbb{T}$ be a torus that is defined over a finite field $\mathfrak{f}$.  Suppose $\mathbb{T}$ splits over an extension $\mathfrak{f}_n$ and set $G_n := Gal(\mathfrak{f}_n/\mathfrak{f})$.  We wish to compute
$\hat{H}^i(G_n,\mathbb{T}(\mathfrak{f}_n))$.  Recall that, since $G_n$ is cyclic, $\hat{H}^{i}(G_n,A)\cong \hat{H}^{i+2}(G_n,A)$ for all $i$, and all $G_n$-modules $A$.

\begin{proposition}\label{trivialH1finitefields}
$\hat{H}^1(G_n,\mathbb{T}(\mathfrak{f}_n))$ is trivial.
\end{proposition}

\proof
This is Lang's theorem for connected algebraic groups over finite fields.
\qed

\begin{corollary}
$\hat{H}^0(G_n,\mathbb{T}(\mathfrak{f}_n))$ is trivial.
\end{corollary}

\proof
It is well-known that the Herbrand quotient of a finite group is trivial.  Hence, the result follows.
\qed

\section{Unramified Tori over p-adic fields}

Let $T$ be a torus defined over $k$ that splits over an unramified extension $K/k$ of degree $n$, set $G := Gal(K/k)$, and let $g$ be a generator of $G$.  Recall that, since $G$ is cyclic, $\hat{H}^{i}(G,A)\cong \hat{H}^{i+2}(G,A)$ for all $i$, and all $G$-modules $A$.  Let $X$ denote the cocharacter lattice of $T$ and $Y$ denote the character lattice of $T$.  As in \cite[Section 3]{moyprasad1}, we define $T_0$ to be the maximal bounded subgroup of $T(k^u)$, and $$T_r := \{t \in T_0 : val(\chi(t) - 1) \geq r \ \ \forall \chi \in Y \}$$ for any natural number $r$.  We may furthermore define $T_r^L := T_r \cap T(L)$ for any field $L$ such that $k \subset L \subset k^u$.  We wish to compute $\hat{H}^i(G, T(K))$.  We use the filtration $T_r^K$ on $T(K)$ in the following way.

It is clear that $N_{K/k}$ maps $T_r^K$ into $T_r^k$ for all $r \geq 0$.  Therefore, we get an induced map $$N_{K/k} : T_s^K / T_{r}^K \rightarrow T_s^k / T_{r}^k$$ for all $r > s \geq 0$.

\begin{lemma}\label{H1compactpart}
$\hat{H}^1(G, T_{s}^K) = 0 \ \forall s \geq 0$.
\end{lemma}

\proof
We prove the case $s = 0$, as the other cases are similar.  Recall that $T_0^K = \underleftarrow{\mathrm{lim}} \ T_0^K / T_{r+1}^K$.  We need the following Lemma:

\begin{lemma}\label{abstractcohomology}
Let $M$ be a $G$-module complete with respect to a topology given by a sequence of $G$-invariant subgroups $M = M_0 \supset M_1 \supset M_2 \supset ...$, so $\cap M_i = 0$.  Thus, $M = \underleftarrow{\mathrm{lim}} \ M / M_i$.  If $H^1(G, M_i / M_{i+1}) = 0 \ \forall i \geq 0$, then $H^1(G, M_j) = 0 \forall j \geq 0$.
\end{lemma}

\proof
We do the case $j = 0$.  The argument for $j \geq 1$ is similar.  Let $f : G \rightarrow M$ be an element in $H^1(G,M)$.  We have a natural map $H^1(G, M ) \rightarrow H^1(G, M / M_1) = 0$.  Thus, there exists $m_0 \in M$ such that $f(\sigma) = (\sigma(m_0) - m_0) (\mathrm{mod}  \ M_1)$.  Then define $f_1 := f - (\sigma(m_0) - m_0)$.  This is in $H^1(G, M_1)$. But then again we have the natural map $H^1(G, M_1) \rightarrow H^1(G, M_1 / M_2) = 0$.  Thus, again, there exists $m_1 \in M_1$ such that $f_1(\sigma) = (\sigma(m_1) - m_1) (\mathrm{mod} \ M_2)$.  So define $f_2 = f - (\sigma(m) - m) - (\sigma(m_1) - m_1)$.  Continue this process indefinitely.  Then define $\tilde{m} := (m_0, m_0 + m_1, m_0 + m_1 + m_2, m_0 + m_1 + m_2 + m_3, ...) \in \underleftarrow{\mathrm{lim}} \ M / M_i = M$.  Then, $f(\sigma) = \sigma(\tilde{m}) - \tilde{m}$, so we get finally that $H^1(G,M) = 0$.
\qed

Let $r \geq 1$.  Note that $T_r^K / T_{r+1}^K = X \otimes (1 + \mathfrak{p}_K^r) / (1 + \mathfrak{p}_K^{r+1}) = X \otimes \mathfrak{f}_n$.  Well, $X \otimes \mathfrak{f}_n$ is a vector space over $\mathfrak{f}$.  But it is known that all forms of vector spaces are equivalent.  Therefore, the vector space $X \otimes \mathfrak{f}_n$ with the above Galois action is equivalent to $X \otimes \mathfrak{f}_n$ with the standard action, i.e. $g \in G$ acts on $x \otimes z$ by $x \otimes gz$.   Thus, we wish to calculate $\hat{H}^1(G, X \otimes \mathfrak{f}_n)$ where $X \otimes \mathfrak{f}_n$ is given the standard action.  Since $X \otimes \mathfrak{f}_n$ is just a direct sum of copies of $\mathfrak{f}_n$, we are reduced to calculating $\hat{H}^1(G, \mathfrak{f}_n)$.  But since the trace map over finite fields is surjective, we get that $\hat{H}^0(G, \mathfrak{f}_n) = 0$.  Therefore, since the Herbrand quotient of a finite group is trivial, we get that $\hat{H}^1(G, \mathfrak{f}_n) = 0$.  Finally, $\hat{H}^1(G, T_0^K / T_1^K) = 0$ by Lang's theorem since $T_0^K / T_1^K$ is a connected algebraic group over a finite field.  By Lemma \ref{abstractcohomology}, we have concluded the proof of Lemma \ref{H1compactpart}.
\qed

\begin{lemma}\label{H0compactpart}
$\hat{H}^0(G,T_0^K) = 0$.
\end{lemma}

\proof
We claim that the natural maps $$N_{K/k} : T_r^K / T_{r+1}^K \rightarrow T_r^k / T_{r+1}^k$$ are surjective for $r \geq 0$.  First let $r \geq 1$.  First note that $T_0^L = X \otimes \mathfrak{o}_L^*$ and $T_r^L = X \otimes (1 + \mathfrak{p}_L^r)$ for $r \geq 1$.  Then $$N_{K/k} : T_r^K / T_{r+1}^K \rightarrow T_r^k / T_{r+1}^k$$ becomes $$N_{K/k} : X \otimes \left( (1 + \mathfrak{p}_K^r) / (1 + \mathfrak{p}_K^{r+1}) \right) \rightarrow (X \otimes \left( (1 + \mathfrak{p}_K^r) / (1 + \mathfrak{p}_K^{r+1}) \right) )^G$$
since $T_r^k / T_{r+1}^k = T_r^G / T_{r+1}^G = (T_r / T_{r+1})^G$ since $H^1(G, T_{r+1}) = 0$ by Lemma \ref{H1compactpart}.  But $(1 + \mathfrak{p}_K^r) / (1 + \mathfrak{p}_K^{r+1}) \cong \mathfrak{o}_K / \mathfrak{p}_K \cong \mathfrak{f}_n$.  Thus, we get
$$N_{K/k} : X \otimes \mathfrak{f}_n \rightarrow (X \otimes \mathfrak{f}_n)^G$$
Well, $X \otimes \mathfrak{f}_n$ is a vector space over $\mathfrak{f}$.  But it is known that all forms of vector spaces are equivalent.  Therefore, the vector space $X \otimes \mathfrak{f}_n$ with the above Galois action is equivalent to $X \otimes \mathfrak{f}_n$ with the standard action, i.e. $g \in G$ acts on $x \otimes z$ by $x \otimes gz$.  Therefore, $$N_{K/k} : X \otimes \mathfrak{f}_n \rightarrow (X \otimes \mathfrak{f}_n)^G$$ becomes
$$N_{K/k} : X \otimes \mathfrak{f}_n \rightarrow X \otimes \mathfrak{f},$$
which is surjective since the trace map $\mathfrak{f}_n \rightarrow \mathfrak{f}$ is surjective.  Now let $r = 0$.  We wish to show that $$N_{K/k} : T_0^K / T_1^K \rightarrow T_0^k / T_1^k$$ is surjective.  But this map becomes
$$N_{K/k} : X \otimes \left( \mathfrak{o}_K^* / (1 + \mathfrak{p}_K) \right) \rightarrow (X \otimes \left( \mathfrak{o}_K^* / (1 + \mathfrak{p}_K) \right) )^G$$ which is just
$$N_{K/k} : \mathbb{T}(\mathfrak{f}_n) \rightarrow \mathbb{T}(\mathfrak{f}).$$
This last map is surjective by Lemma \ref{normtorifinitefields}.  Finally, by \cite[Lemma 2, p. 81]{serre1}, we have our result.
\qed

\begin{corollary}\label{reductiontori}
$\hat{H}^i(G, X \otimes K^*) = \hat{H}^i(G, T(K)) = \hat{H}^i(G,X_*(T))$ for $i=0,1$.
\end{corollary}

\proof
Recall that $X \otimes K^* \xrightarrow{\sim} T(K)$ given by evaluation, restricting to an isomorphism $X \otimes \mathfrak{o}_K^* \xrightarrow{\sim} T_0^K$.  We may thus identify $X$ with the quotient $T(K) / T_0^K$, giving rise to the short exact sequence $$1 \rightarrow T_0^K \rightarrow T(K) \rightarrow X \rightarrow 1.$$  We therefore have the long exact sequence $$... \rightarrow \hat{H}^0(G, T_0^K) \rightarrow \hat{H}^0(G, T(K)) \rightarrow \hat{H}^0(G, X) \rightarrow \hat{H}^1(G, T_0^K) \rightarrow $$ $$\hat{H}^1(G, T(K)) \rightarrow \hat{H}^1(G, X) \rightarrow \hat{H}^0(G, T_0^K) \rightarrow ...$$
By corollary \ref{H0compactpart} and Lemma \ref{H1compactpart}, we get the result.
\qed

\begin{corollary}
If $T(k)$ is compact, we have $\hat{H}^0(G, T(K)) = 0$.
\end{corollary}

\proof
$T(k)$ being compact means that $X^G = 0$, hence the result follows by corollary \ref{reductiontori}
\qed

\begin{corollary} Let $\Phi$ be the characteristic polynomial of $g$ acting on $X$. If $\Phi(1)\neq 0$,
i.e. $T_X(k)$ is compact,
then $H^0(G, X\otimes K^{\times})=0$ and the order of $H^1(G, X\otimes K^{\times})$ is $\Phi(1)$.
\end{corollary}

\proof
Since $0 = g^n - 1 = (1 + g + g^2 + ... + g^{n-1})(g-1)$ and since $1$ is not an eigenvalue of $g$ on $X$, we get that $1 + g + g^2 + ... + g^{n-1}$ is the zero map on $X$.  Now, we have that $H^1(G, X\otimes K^{\times}) = H^1(G, X) = ker(N_{K/k}) / (Id - g)X$ where $N_{K/k}$ is the norm map $N_{K/k} : X \rightarrow X$ on $X$. Since $1 + g + g^2 + ... + g^{n-1}$ is the zero map on $X$, we get that $H^1(G, X) = X / (Id - g)X$.
It is known by standard lattice theory that the cardinality of $X / (Id - g)X$ is equal to the volume of a fundamental domain for the lattice $(Id - g)X$.  It is also known that the volume of a fundamental domain for the lattice $(Id - g)X$ is equal to $|det(Id - g)|$.  But $|det(Id - g)|$ is just $|\Phi(1)|$.  But $|\Phi(1)| = \Phi(1)$ for the following reason.  Note that since $1 + g + g^2 + ... + g^{n-1} = 0$, $\Phi$ divides a power of $1 + x + x^2 + ... + x^{n-1}$ (since the minimal polynomial of $g$ divides $1 + x + x^2 + ... + x^{n-1}$).  Since $1 + 1 + 1^2 + ... + 1^{n-1} > 0$, we have that $\Phi(1) > 0$, so that $|\Phi(1)| = \Phi(1)$.
\qed

\section{Totally (not necessarily tamely?) ramified tori with splitting field of prime degree, the anisotropic case - Gordan's section}

Let $k$ be a $p$-adic field and $K$ a cylic extension of degree $n$, and $G=G(K/k)$ the
Galois group. Let $g$ be a generator of $G$.
 Let $X$ be a lattice (i.e. a finitely generated, torsion-free, abelian group) with
an action of $G$. Every such $X$ defines a torus  $T_X$ over $k$ such that
\[
T_X(k)=(X\otimes K^{\times})^G.
\]
We would like to compute Tate cohomology groups $H^i(G, X\otimes K^{\times})$. Recall that, since $G$ is
cyclic, $\hat{H}^{i}(G,A)\cong \hat{H}^{i+2}(G,A)$ for all $i$, and all $G$-modules $A$.



Let $\ell$ be a prime, and assume that $n=\ell$. Then we have an exact sequence
\[
0 \rightarrow \mathbb Z \rightarrow \mathbb Z[G]\rightarrow R \rightarrow 0
\]
where $R$ is the ring of integers in the cyclotomic field obtained by attaching $\ell$-th roots
of $1$ to $\mathbb Q$. Since $\mathbb Z[G]\otimes K^{\times}$ is a projective $G$-module, the exact sequence
gives an isomorphism of Tate cohomology
groups $H^i(G, R\otimes K^{\times})\cong H^{i+1}(G,K^{\times})$ for all integers $i$. Since
$H^1(G,K^{\times})=0$ (Hilbert Theorem 90) and $H^{0}(G,K^{\times})=k^{\times}/N(K^{\times})
\cong \mathbb Z/\ell\mathbb Z$, we have proved a special case $(I=R)$ of the following
proposition:

\begin{proposition} Let $I\subseteq R$ be a non-zero ideal. Then $H^0(G,I\otimes K^{\times})=0$
and $H^1(G,I\otimes K^{\times})\cong \mathbb Z/\ell\mathbb Z$.
\end{proposition}
\begin{proof} The natural map $I\otimes K^{\times}\rightarrow R\otimes K^{\times}$ has a
finite kernel and co-kernel. Thus the Herbrand quotient of
$I\otimes K^{\times}$ is equal to the Herbrand quotient of $R \otimes K^{\times}$. Let
$J\subseteq R$ be an ideal belonging to the inverse class of $I$. Then the $R$-module
$I\oplus J$ is isomorphic to $R\oplus R$. It follows that $H^0(G,I\otimes K^{\times})=0$
so $H^1(G,I\otimes K^{\times})$ must be of order $\ell$. The proposition is proved.
\end{proof}

Now it is easy to determine cohomology groups if $T_X(k)$ is compact. In that case $1$ is not
an eigenvalue of $g$ acting on $X$. Thus $X$ is an $R$-module. Since $R$ is a Dedekind
domain, $X\cong I_1\oplus \ldots \oplus I_m$ where $I_1, \ldots, I_m$ are ideals in $R$, and
the proposition implies that $H^0(G,X\otimes K^{\times})=0$
and $H^1(G,X\otimes K^{\times})\cong (\mathbb Z/\ell\mathbb Z)^m$.

\vskip 10pt

\section{Totally (not necessarily tamely?) ramified tori with splitting field of prime degree, the non-anisotropic case}

\begin{theorem}
Let $T$ be a torus defined over $k$ that splits over $K$, a totally ramified extension of prime degree $q$ of $k$.  Then $\hat{H}^i(G, T_0^K) = 0 \ \forall i$.
\end{theorem}

\proof
See \cite[Theorem 3]{amano}.
\qed

\begin{corollary}
$\hat{H}^i(G, T(K)) = \hat{H}^i(G,X)$.
\end{corollary}

\section{Review of Construction of DeBacker and Reeder}\label{preliminaries}

We first review some of the basic theory from \cite{debackerreeder}.  We first fix a pinning $(\hat{T}, \hat{B}, \{x_{\alpha} \})$ for the dual group $\hat{G}$.  The operator $\hat{\theta}$ dual to $\theta$ extends to an automorphism of $\hat{T}$.  There is a unique extension of $\hat{\theta}$ to an automorphism of $\hat{G}$, satisfying $\hat{\theta}(x_{\alpha}) = x_{\theta \cdot \alpha}$ (see \cite[section 3.2]{debackerreeder}).  Following \cite{debackerreeder}, we may form the semidirect product ${}^L G := \ <\hat{\theta}> \ltimes \hat{G}$.

\begin{definition}
Let $W_F'$ denote the Weil-Deligne group.  A Langlands parameter $\phi : W_F' \rightarrow {}^L G$ is called a \emph{tame regular semisimple elliptic Langlands parameter} (abbreviated TRSELP) if

(1) $\phi$ is trivial on $I_F^+$,

(2) The centralizer of $\phi(I_F)$ in $\hat{G}$ is a torus.

(3) $C_{\hat{G}}(\phi)^o = (\hat{Z}^{\hat{\theta}})^o$, where $\hat{Z}$ denotes the center of $\hat{G}$.
\end{definition}

Condition (2) forces $\phi$ to be trivial on $SL(2,\mathbb{C})$.  Let $\hat{N} = N_{\hat{G}}(\hat{T})$.  After conjugating by $\hat{G}$, we may assume that $\phi(I_F) \subset \hat{T}$ and $\phi(\Phi) = \hat{\theta} f$, where $f \in \hat{N}$.  Let $\hat{w}$ be the image of $f$ in $\hat{W}_o$, and let $w$ be the element of $W_o$ corresponding to $\hat{w}$.

Let $\phi$ be a TRSELP with associated $w$ and set $\sigma = w \theta$. $\sigma$ is an automorphism of $X$.  Let $\hat{\sigma}$ be the automorphism dual to $\sigma$, and let $n$ be the order of $\sigma$.  We set $\hat{G}_{ab} := \hat{G} / \hat{G}'$, where $\hat{G}'$ denotes the derived group of $\hat{G}$. Let ${}^L T_{\sigma} := \langle \hat{\sigma} \rangle \ltimes \hat{T}$.  Associated to $\phi$, DeBacker--Reeder (see \cite[Chapter 4]{debackerreeder}) define a $\hat{T}$-conjugacy class of Langlands parameters

\begin{equation}
\phi_T : W_F \rightarrow {}^L T_{\sigma} \ \label{phiT}
\end{equation}

\noindent as follows.  Set $\phi_T := \phi$ on $I_F$, and $\phi_T(\Phi) := \hat{\sigma} \ltimes \tau$ where $\tau \in \hat{T}$ is any element whose class in $\hat{T} / (1 - \hat{\sigma}) \hat{T}$ corresponds to the image of $f$ in $\hat{G}_{ab} / (1 - \hat{\theta}) \hat{G}_{ab}$ under the bijection

\begin{equation}
\hat{T} / (1 - \hat{\sigma}) \hat{T} \stackrel{\sim}{\rightarrow} \hat{G}_{ab} / (1 - \hat{\theta}) \hat{G}_{ab} \ \label{bijectionfortau}
\end{equation}

In \cite[Chapter 4]{debackerreeder}, DeBacker and Reeder construct a canonical bijection between $\hat{T}$-conjugacy classes of admissible homomorphisms $\phi : W_t \rightarrow {}^L T_{\sigma}$ and depth-zero characers of $T^{\Phi_{\sigma}}$ where $\Phi_{\sigma} := \sigma \otimes \Phi^{-1}$.  We briefly summarize this construction. Let $\mathbb{T} := X \otimes \mathfrak{F}^*$.  Given automorphisms $\alpha, \beta$ of abelian groups $A,B$, respectively, let $Hom_{\alpha, \beta}(A,B)$ denote the set of homomorphisms $f : A \rightarrow B$ such that $f \circ \alpha = \beta \circ f$.  The twisted norm map $$N_{\sigma} : \mathbb{T}^{\Phi_{\sigma}^n} \rightarrow \mathbb{T}^{\Phi_{\sigma}}$$ given by $N_{\sigma}(t) = t \Phi_{\sigma}(t) \Phi_{\sigma}^2(t) \cdots \Phi_{\sigma}^{n-1}(t)$ induces isomorphisms $$\mathrm{Hom}(\mathbb{T}^{\Phi_{\sigma}}, \mathbb{C}^*) \stackrel{\sim}{\rightarrow}  \mathrm{Hom}_{\Phi_{\sigma}, Id}(\mathbb{T}^{\Phi_{\sigma}^n}, \mathbb{C}^*) \stackrel{\sim}{\rightarrow} \mathrm{Hom}_{\Phi_{\sigma}, Id}(X \otimes \mathfrak{f}_n^*, \mathbb{C}^*)$$
Moreover, the map $s \mapsto \chi_s$ gives an isomorphism $$\mathrm{Hom}_{\Phi, \hat{\sigma}}(\mathfrak{f}_n^*, \hat{T}) \stackrel{\sim}{\rightarrow} \mathrm{Hom}_{\Phi_{\sigma}, Id}(X \otimes \mathfrak{f}_n^*, \mathbb{C}^*)$$ where $\chi_s(\lambda \otimes a) := \lambda(s(a))$.  The canonical projection $I_t \rightarrow \mathfrak{f}_m^*$ induces an isomorphism as $\Phi$-modules $I_t / (1 - Ad (\Phi)^m)I_t \stackrel{\sim}{\rightarrow} \mathfrak{f}_m^*$ where $Ad$ denotes the adjoint action.  Since $\hat{\sigma}$ has order $n$, we have $\mathrm{Hom}_{\Phi, \hat{\sigma}}(\mathfrak{f}_n^*, \hat{T}) \cong \mathrm{Hom}_{Ad(\Phi), \hat{\sigma}}(I_t, \hat{T})$.  Therefore, the map $s \mapsto \chi_s$ is a canonical bijection $$\mathrm{Hom}_{Ad(\Phi), \hat{\sigma}}(I_t, \hat{T}) \stackrel{\sim}{\rightarrow} \mathrm{Hom}(\mathbb{T}^{\Phi_{\sigma}}, \mathbb{C}^*)$$
Moreover, we have an isomorphism $${}^0 T^{\Phi_{\sigma}} \times X^{\sigma} \stackrel{\sim}{\rightarrow} T^{\Phi_{\sigma}}$$ $$(\gamma, \lambda) \mapsto \gamma \lambda(\varpi)$$ where ${}^0 T$ is the group of $\mathfrak{o}_{F^u}$-points of $\mathbf{T}$.

Finally, note that $\hat{T} / (1 - \hat{\sigma}) \hat{T}$ is the character group of $X^{\sigma}$, whereby $\tau \in \hat{T} / (1 - \hat{\sigma}) \hat{T}$ corresponds to $\chi_{\tau} \in \mathrm{Hom}(X^{\sigma}, \mathbb{C}^*),$ where $ \chi_{\tau}(\lambda) := \lambda(\tau)$.  Therefore, we have a canonical bijection between $\hat{T}$-conjugacy classes of admissible homomorphisms $\phi : W_t \rightarrow {}^L T_{\sigma}$ and depth-zero characters

\begin{equation}
\chi_{\phi} := \chi_s \otimes \chi_{\tau} \in \mathrm{Irr}(T^{\Phi_{\sigma}}) \ \ \label{chitau}
\end{equation}

\noindent where $s := \phi|_{I_t}$, $\phi(\Phi) = \hat{\sigma} \ltimes \tau$, and where we have inflated $\chi_s$ to ${}^0 T^{\Phi_{\sigma}}$.

\

To get the depth-zero $L$-packet associated to $\phi$, one implements the component group $$\mathrm{Irr}(C_{\phi}) \cong [X / (1 - w \theta) X]_{\mathrm{tor}}$$ as follows. We set $X_w$ to be the preimage of $[X / (1 - w \theta) X]_{\mathrm{tor}}$ in $X$.  To $\lambda \in X_w$, DeBacker and Reeder associate a 1-cocycle $u_{\lambda}$, hence a twisted Frobenius $\Phi_{\lambda} = Ad(u_{\lambda}) \circ \Phi$.  Moreover, to $\lambda$, they associate a facet $J_{\lambda}$, and hence a parahoric subgroup $G_{\lambda}$ associated to $J_{\lambda}$.  Let $\mathbb{G}_{\lambda} := G_{\lambda} / G_{\lambda}^+$.  Let $W_{\lambda}$ be the subgroup of $W^o$ generated by reflections in the hyperplanes containing $J_{\lambda}$.  Then to $\lambda$, DeBacker--Reeder also associate an element $w_{\lambda} \in W_{\lambda}$. Fix once and for all a lift $\dot{w}$ of $w$ to $N$.  Using this lift, DeBacker and Reeder also associate a lift $\dot{w}_{\lambda}$ of $w_{\lambda}$ to $N$. By Lang's theorem, there exists $p_{\lambda} \in G_{\lambda}$ such that $p_{\lambda}^{-1} \Phi_{\lambda} (p_{\lambda}) = \dot{w}_{\lambda}$.  We then define $T_{\lambda} := Ad(p_{\lambda}) T$, and set $\chi_{\lambda} := \chi_{\phi} \circ Ad(p_{\lambda})^{-1}$.  Since $\chi_{\lambda}$ is depth-zero, its restriction to ${}^0 T_{\lambda}^{\Phi_{\lambda}}$ factors through a character $\chi_{\lambda}^0$ of $\mathbb{T}_{\lambda}^{\Phi_{\lambda}}$, where $\mathbb{T}_{\lambda}^{\Phi_{\lambda}}$ is the projection of ${}^0 T^{\Phi_{\lambda}}$ in $\mathbb{G}_{\lambda}$.  Therefore, $\chi_{\lambda}^0$ gives rise to an irreducible cuspidal Deligne-Lusztig representation $\kappa_{\lambda}^0$ of $\mathbb{G}_{\lambda}^{\Phi_{\lambda}}$.  Inflate $\kappa_{\lambda}^0$ to a representation of $G_{\lambda}^{\Phi_{\lambda}}$, and define an extension to $Z^{\Phi_{\lambda}} G_{\lambda}^{\Phi_{\lambda}}$ by $$\kappa_{\lambda} := \chi_{\lambda} \otimes \kappa_{\lambda}^0$$ where $Z$ denotes the center of $G$.  This makes sense since $(Z \cap G_{\lambda})^{\Phi_{\lambda}}$ acts on $\kappa_{\lambda}^0$ via the restriction of $\chi_{\lambda}^0$.  Finally, form the representation $$\pi_{\lambda} := \mathrm{Ind}_{Z^{\Phi_{\lambda}} G_{\lambda}^{\Phi_{\lambda}}}^{G^{\Phi_{\lambda}}} \kappa_{\lambda}$$ where Ind denotes smooth induction.  Then DeBacker--Reeder construct a packet $\Pi(\phi)$ of representations on the pure inner forms of $G$, parameterized by $\mathrm{Irr}(C_{\phi})$, using the above construction, where $C_{\phi}$ is the component group of $\phi$.

\section{Groups of type L}\label{groupsoftypeL}
We now review the theory of ``groups of type L'' due to Benedict Gross.  Let $F$ be a field, $F^{\mathrm s}$ a separable closure, and $T$ a torus defined over $F$ that splits over an extension $E \subset F^s$.
Let $\Gamma = \Gal(E/F)$.  Let $X^*(T)$ be the character module of $T$ and $X_*(T)$ the cocharacter
module of $T$.  Define $\hat{T} = X^*(T) \otimes \mathbb{C}^*$.
The group $\Gamma$ acts on $\hat{T}$ via its action on $X^*(T)$.

\begin{definition}
A \emph{group of type L} is a group extension of $\Gamma$ by $\hat{T}$.
\end{definition}

Let $D$ be such a group.  Then we have an exact sequence
$$1 \rightarrow \hat{T} \rightarrow D \rightarrow \Gamma \rightarrow 1$$
We now describe how, given a Langlands parameter $$\phi : W_F \rightarrow D,$$ where $D$ is a group of type L, we can naturally attach a character of $T(E)_{\Gamma} := T(E) / I_{\Gamma}(T(E))$, where $I_{\Gamma}(T(E)) = \{(1 - \xi)t \ : t \in T(E), \xi \in \Gamma \}$.
Restricting $\phi$ to $W_E$ we get a homomorphism $$\phi|_{W_E} : W_E \rightarrow \hat{T}$$
By the Langlands correspondence for tori, this gives us a character $\chi : T(E) \rightarrow \mathbb{C}^*$.  Since $\phi|_{W_E}$ extends to $\phi$, one can see that
$$\chi(t^{\sigma}) = \chi(t)\ \mbox{for all $\sigma \in \Gamma$.}$$
Therefore, $\chi(t^{\sigma - 1}) = 1$ for all $\sigma \in \Gamma$.  Thus, $\chi$ is trivial on the augmentation ideal $I_{\Gamma}(T(E))$
and gives $$\chi : T(E)_\Gamma \rightarrow \mathbb{C}^*$$
Invariants and coinvariants are related by the norm map $$N : T(E) \rightarrow T(F)$$ $$t \mapsto \displaystyle\prod_{\xi \in \Gamma} \xi(t)$$ in the Tate cohomology sequence
$$1 \rightarrow \hat{H}^{-1}(\Gamma,T(E)) \rightarrow T(E)_{\Gamma} \xrightarrow{N} T(F) = T(E)^{\Gamma} \rightarrow \hat{H}^0(\Gamma,T(E)) \rightarrow 1$$ (note that the norm map $N$ factors to $T(E)_{\Gamma}$).
We have thus constructed a character $\chi$ of $T(E)_{\Gamma}$ from a Langlands parameter $\phi$. We note that $T(E)_{\Gamma}$ is a cover of $N(T(E)_{\Gamma})$, which is a subgroup of $T(F)$.  It is sometimes the case that $N$ is surjective, in which case $\chi$ is then a character of $T(E)_{\Gamma}$, which is a cover of $T(F)$.

\section{The relationship between the Gross construction and the DeBacker--Reeder construction}\label{grossdebackerreeder}

We consider here a TRSELP $\phi$ for any unramified connected reductive group G.  Let $T$ be as in \S\ref{groupsoftypeL}.  Let $E$, $\chi$, etc.~be as in~\S\ref{groupsoftypeL}.  Let $w$ be the Weyl group element associated to $\phi$, and set $\sigma = w \theta\in {\rm Aut}(T)$.  Let $\chi_{\phi}$ be the character of $T(F) = T^{\Phi_{\sigma}}$ that DeBacker and Reeder attach to $\phi$ (see section \ref{preliminaries}).

We have the exact sequence $$1 \rightarrow \hat{H}^{-1}(\Gamma, T(E)) \rightarrow T(E)_{\Gamma} \rightarrow T(F) \rightarrow \hat{H}^0(\Gamma, T(E)) \rightarrow 1$$ where $\Gamma = \Gal(E/F)$.  Recall that $\phi$ canonically gives rise to a character $\chi$ of $T(E)_{\Gamma}$, by local Langlands for tori (see section \ref{groupsoftypeL}).  Note that the above exact sequence restricts to an exact sequence
$$1 \rightarrow \hat{H}^{-1}(\Gamma, T(\mathfrak{o}_E)) \rightarrow T(\mathfrak{o}_E)_{\Gamma} \rightarrow T(\mathfrak{o}_F) \rightarrow \hat{H}^0(\Gamma, T(\mathfrak{o}_E)) \rightarrow 1$$
Moreover, one can show using a profinite version of Lang's theorem and various results about tori over finite fields, that since $T$ is unramified, $\hat{H}^{-1}(\Gamma, T(\mathfrak{o}_E)) = \hat{H}^0(\Gamma, T(\mathfrak{o}_E)) = 1$.  Therefore, the map $$T(\mathfrak{o}_E)_{\Gamma} \xrightarrow{N} T(\mathfrak{o}_F)$$ is an isomorphism, and we may view $\chi|_{T(\mathfrak{o}_E)_{\Gamma}}$ as a character of $T(\mathfrak{o}_F)$ via this isomorphism.  Now, since $\chi$ was obtained from $\phi|_{W_E}$ via the local Langlands correspondence, we get that $\chi_{\phi} \circ N = \chi$ on $T(\mathfrak{o}_E)_{\Gamma}$.  We have therefore proven the following lemma.

\begin{lemma}\label{grossanddebackerreedercompatibility}
The restriction to $T(\mathfrak{o}_E)_{\Gamma} \xrightarrow{\sim} T(\mathfrak{o}_F)$ of the genuine character arising from the Gross construction coincides with the character of $T(\mathfrak{o}_F)$ that is constructed from $\phi$ via the construction of DeBacker--Reeder construction.
\end{lemma}

\section{Rectifiers and DeBacker/Reeder}

Recall that we have fixed a splitting $(\hat{T}, \hat{B}, \{x_{\alpha} \})$ for the dual group $\hat{G}$ and that $\hat{N} = N_{\hat{G}}(\hat{T})$.  For each simple root $\alpha$, let $\phi_{\alpha} : SL(2) \rightarrow \hat{G}$ be defined by $\phi_{\alpha}(\mathrm{diag}(z,1/z)) = \alpha^{\vee}(z)$ and $\phi_{\alpha}\mat{1}{1}{0}{1} = x_{\alpha}$. Let $\sigma_{\alpha} = \phi_{\alpha}\mat{0}{1}{-1}{0}$.

\begin{definition}
The Tits group $\widetilde{W}$ is the subgroup of $\hat{N}$ generated by $\{\sigma_{\alpha} \}$ for $\alpha$ simple.
\end{definition}

For each simple root $\alpha$, let $m_{\alpha} = \sigma_{\alpha}^2 = \alpha^{\vee}(-1)$.  Let $\hat{T}_2$ be the subgroup of $\hat{T}$ generated by the $m_{\alpha}$.

\begin{theorem}{(Tits \cite{tits})}
\begin{enumerate}
\item The kernel of the natural map $\widetilde{W_o} \rightarrow \hat{W}_o$ is $\hat{T}_2$,
\item The elements $\sigma_{\alpha}$ satisfy the braid relations,
\item There is a canonical lifting of $\hat{W}_o$ to a subset of $\widetilde{W}$: take a reduced expression $w = s_{\alpha_1} ... s_{\alpha_n}$, and let $\tilde{w} = \sigma_{\alpha_1} ... \sigma_{\alpha_n}$.
\end{enumerate}
\end{theorem}

Let $\phi$ be a TRSELP.  Recall that $\phi(I_F) \subset \hat{T}$ and $\phi(\Phi) = \hat{\theta} f$, where $f \in \hat{N}$.  Let $\hat{w}$ be the image of $f$ in $\hat{W}_o$, and let $w$ be the element of $W_o$ corresponding to $\hat{w}$.  Write $\hat{w} = s_{\alpha_1} ... s_{\alpha_n}$ as a product of simple reflections, and let $\tilde{w} = \sigma_{\alpha_1} ... \sigma_{\alpha_n}$ be the canonical lift of $\hat{w}$ to $\widetilde{W}$.

\begin{definition}
Given $\phi$, we define a homomorphism $\phi_o : W_F \rightarrow {}^L G$ by
\begin{enumerate}
\item $\phi_o|_{I_F} \equiv 1$
\item $\phi_o(\Phi) = \hat{\theta} \tilde{w}$
\end{enumerate}
\end{definition}

By the theory in section \ref{preliminaries}, $\phi$ and $\chi_{\phi_o}$ give rise to a character $\chi_{\phi}$ and $\chi_{\phi_o}$ of $T(F) = T^{\Phi_{\sigma}}$, respectively.  By the theory in section \ref{groupsoftypeL}, $\phi$ and $\phi_o$ gives rise to  characters $\chi$ and $\chi_o$ of $T(E)_{\Gamma}$, respectively.

\begin{proposition}
$\chi \otimes \chi_o = \chi_{\phi}$.
\end{proposition}

\proof
We first note that the claim $\chi \otimes \chi_o = \chi_{\phi}$ is equivalent to showing that $\chi_{\phi_o} \equiv 1$.  But $\phi_o$ is trivial on $I_F$, so $\chi_{\phi_o}$ is trivial on ${}^0 T^{\Phi_{\sigma}}$ (see equation \ref{chitau}).

We claim that $\tilde{w} \in \hat{G}'$, where $\hat{G}'$ is the derived group of $\hat{G}$.  But $\sigma_{\alpha_i} = \phi_{\alpha_i}\mat{0}{1}{-1}{0}$ for $i = 1, 2, ..., n$.  Moreover, $\mat{0}{1}{-1}{0}$ is a commutator since $\mat{0}{1}{-1}{0} \in SL(2)$ (and in particular $SL(2)$ is its own derived group).  Therefore, $\sigma_{\alpha_i}$ is a commutator for all $i$, so $\tilde{w}$ is also a commutator.

Therefore, we may take $\tau = 1$ for $\phi_o$, as in section \ref{preliminaries}.  In particular, together with $\phi|_{I_F} \equiv 1$, this says that $\chi_{\phi_o} \equiv 1$.
\qed





\begin{thebibliography}{9}

\bibitem{adrian}
  Moshe Adrian,
  \emph{A new realization of the Langlands correspondence for $PGL(2,F)$}, Journal of Number Theory 133 (2013) 446-–474.

\bibitem{amano}
  Kazuo Amano,
  \emph{A note on the Galois cohomology groups of algebraic tori}, Nagoya Math. J. Volume 34 (1969), 121-127.

\bibitem{bushnellhenniart}
  Colin Bushnell, Guy Henniart,
  \emph{The essentially tame local Langlands correspondence, III: the general case}, Proc. Lond. Math. Soc. (3) 101 (2010), no. 2, 497–553.

\bibitem{debackerreeder}
  Stephen DeBacker and Mark Reeder,
  \emph{Depth-zero supercuspidal $L$-packets and their stability.}
  Ann. of Math. (2) 169 (2009), no. 3, 795--901.

\bibitem{geo}
  Geo Kam-Fai Tam,
  \emph{Transfer relations in essentially tame local Langlands correspondence}, Ph.D. thesis, University of Toronto, 2012.

\bibitem{howe}
  Roger Howe,
  \emph{Tamely ramified supercuspidal representations of $GL_n(F)$},
   Pacific Journal of Math.  73  (1977),  437--460.

\bibitem{kaletha}
  Tasho Kaletha, \emph{Simple Wild L-packets}, J. Inst. Math. Jussieu (2013) 12(1), 43-75.

\bibitem{moyprasad1}
  Allen Moy, Gopal Prasad,
  \emph{Jacquet functors and unrefined minimal $K$-types},
   Comment. Math. Helv. 71 (1996), no. 1, 98--121.

\bibitem{reeder}
  Mark Reeder,
  \emph{Supercuspidal $L$-packets of positive depth and twisted Coxeter elements},
  J. Reine Angew. Math. 620 (2008), 1-33.

\bibitem{serre1}
  Jean-Pierre Serre,
  \emph{Local Fields}, Graduate Texts in Mathematics, 67. Springer-Verlag, New York-Berlin, 1979.

\end{thebibliography}

\end{document}

