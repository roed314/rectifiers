\documentclass[11pt]{amsart}
\usepackage{amsmath,amscd,amssymb,latexsym, amsfonts}
\usepackage{mathtools}

\theoremstyle{plain}
\newtheorem{iassumption}{Assumption}
\newtheorem{theorem}{Theorem}[section]
\newtheorem{conjecture}[theorem]{Conjecture}
\newtheorem{proposition}[theorem]{Proposition}
\newtheorem{corollary}[theorem]{Corollary}
\newtheorem{hypothesis}[theorem]{Hypothesis}
\newtheorem{assumption}[theorem]{Assumption}
\newtheorem{lemma}[theorem]{Lemma}
\newtheorem{question}[theorem]{Question}
\newtheorem{exercise}[theorem]{Exercise}
\newtheorem{statement}[theorem]{Statement}
\newtheorem{example}[theorem]{Example}

\newcommand{\MAxxx}[1]{$\clubsuit$\footnote{#1}}
\newcommand{\DRxxx}[1]{$\spadesuit$\footnote{#1}}
\newcommand{\HT}[1]{\hat{\HH}{}^{#1}}

\theoremstyle{definition}
\newtheorem{definition}[theorem]{Definition}
\newtheorem{remark}[theorem]{Remark}

\setlength{\oddsidemargin}{0.2in}
\setlength{\evensidemargin}{0.2in}
\setlength{\textwidth}{6.1in}

\DeclareMathOperator{\Gal}{Gal}
\DeclareMathOperator{\val}{val}
\DeclareMathOperator{\HH}{H}
\DeclareMathOperator{\Ad}{Ad}
\DeclareMathOperator{\Nm}{Nm}
\DeclareMathOperator{\Hom}{Hom}
\DeclareMathOperator{\Spec}{Spec}
\DeclareMathOperator{\Res}{Res}
\DeclareMathOperator{\Fr}{Fr}
\DeclareMathOperator{\Ind}{Ind}
\DeclareMathOperator{\gen}{gen}

\DeclareMathOperator{\GL}{GL}
\DeclareMathOperator{\PGL}{PGL}
\DeclareMathOperator{\SL}{SL}

\newcommand{\mat}[4]{\left( \begin{array}{cc} {#1} & {#2} \\ {#3} & {#4}
\end{array} \right)}
\newcommand{\TT}{\mathcal{T}}
\newcommand{\C}{\mathcal{C}}
\newcommand{\CC}{\mathbb{C}}
\newcommand{\CCx}{\mathbb{C}^\times}
\newcommand{\OK}{\mathcal{O}_K}
\newcommand{\OKn}{\mathcal{O}_{K_n}}
\newcommand{\OL}{\mathcal{O}_L}
\newcommand{\ZZ}{\mathbb{Z}}
\newcommand{\QQ}{\mathbb{Q}}
\newcommand{\Gm}{\mathbb{G}_m}
\newcommand{\Lx}{L^\times}

\newcommand{\Weil}{\mathcal{W}}
\newcommand{\WD}{\mathcal{W}'}
\newcommand{\Lpack}{\mathcal{L}}
\newcommand{\Pgen}{P_G^{\gen}}
\newcommand{\mugen}{\boldsymbol\mu^{\gen}}

\newcommand{\la}{\langle}
\newcommand{\ra}{\rangle}

\newcommand{\invlim}[1]{\varprojlim_{#1}}
\newcommand{\Normalizer}[2]{\operatorname{N}_{#2}(#1)}

\begin{document}
\title{Rectifiers and the local Langlands Correspondence, I : the unramified case}
\author{Moshe Adrian and David Roe}
%\address{Department of Mathematics, University of Utah, Salt Lake City, UT 84112, U.S.A.}

%\email{savin@math.utah.edu}


\begin{abstract}

In this paper we initiate a program to generalize the notion of the
rectifier that occurs in the local Langlands correspondence for
$\GL_{n}(K)$.  Our generalization begins with unramified
connected reductive
groups, and certain classes of Langlands parameters.

\end{abstract}

\maketitle

\section{Introduction} \label{section:intro}

Let $G$ be a connected reductive group, defined over a $p$-adic field $K$.
The local Langlands conjecture predicts the existence of a finite to one map,
from the set of isomorphism classes of irreducible admissible representations
of $G(K)$, to the set of equivalence classes of admissible homomorphisms
from the Weil-Deligne group $\WD_K$, to the $L$-group ${}^L G$.

In the case of supercuspidal representations of $G(K)$, there has been a
significant amount of progress in the local Langlands correspondence in
recent years.  In the recent work of \cite{bushnellhenniart}, \cite{debackerreeder},
\cite{kaletha}, and \cite{reeder}, the strategy of constructing a local Langlands
correspondence is to first attach a character of an elliptic torus to a Langlands
parameter $\Weil_K \rightarrow {}^L G$, where $\Weil_K$ is the Weil group of
$K$, and then to construct a putative supercuspidal $L$-packet associated to
this character.

Some of the difficulty arises in the construction of the character of an elliptic
torus.  The difficulty arises from the fact that a Langlands parameter
$\Weil_K \rightarrow {}^L G$ does not necessarily have image inside the
$L$-group of an elliptic torus.  If it did, then the local Langlands correspondence
for tori could give us a desired character of an elliptic torus.
Instead, one has to rectify the
situation in some way.

The simplest example of this difficulty arising is in the case of supercuspidal
representations of $\PGL_{2}(K)$.  Suppose that $p \neq 2$, and that that
$\varphi : \Weil_K \rightarrow \SL_{2}(\CC)$ is a homomorphism that is
irreducible as a representation.  Then $\varphi = \Ind_{\Weil_L}^{\Weil_K}(\chi)$,
where $L/K$ is a tamely ramified quadratic extension, for some character
$\chi$ of $\Lx$ that is trivial on $\Nm_{L/K}(\Lx)$, where $\Nm_{L/K}$
denotes the norm map from $L$ to $K$.  The image of $\varphi$
is contained in the normalizer of the dual split torus $\hat{T}$.  This normalizer
is a non-split extension of $\Gal(L/K)$ by $\hat{T}$.
In particular, the image of $\varphi$ is not contained in the $L$-group of a torus.

Moreover, we have the exact sequence
$$1 \rightarrow \mathbb{Z} / 2 \mathbb{Z} \rightarrow \Lx / \Nm_{L/K}(\Lx) \rightarrow L^1 \rightarrow 1$$
$$\ \ \ \ \ \ \ \ \ \ \ \ \ \ \ \ \ \ \ \ \ \ w \Nm_{L/K}(L^{\times}) \mapsto w / \sigma(w)$$
where $L^1$ denotes the norm one elements in $L$, and where $\sigma$ is
the nontrivial element of $Gal(L/K)$.  We note that $L^1$
is an elliptic torus in $\PGL_{2}(K)$.  In particular, the Langlands parameter
$\varphi$ naturally provides a character $\chi$, not of the elliptic torus 
$L^1 \subset PGL(2,K)$,
but of the two-fold cover $\Lx / \Nm_{L/K}(\Lx)$.  To construct a supercuspidal
representation of $\PGL_{2}(K)$, we need a character of $L^1$.  
To rectify the situation at hand, we can twist $\chi$ by a genuine character of
$\Lx / \Nm_{L/K}(\Lx)$ in order to obtain a character of $L^1$.  This twist
is exactly what appears in the work of \cite{bushnellhenniart}, and is the
central subject of Bushnell and Henniart's notion of a rectifier.

In this paper, we propose a generalization of Bushnell and Henniart's
notion of a
rectifier.  In particular, we define the notion of a rectifier for
unramified minisotropic tori.  We use a recent construction of Benedict
Gross, called ``groups of type L'' in order to give a framework that
will allow us to define a rectifier.  This construction allows us to define
a notion of admissible pair.   We will show that a rectifier exists, is
unique, and is compatible
with the rectifier in the setting $\GL_{n}(K)$ as in \cite{bushnellhenniart}.\MAxxx{Insert a paragraph describing each section of the paper.}

\subsection*{Acknowledgements}

Part of this paper was heavily influenced by
conversations with Gordan Savin.  We wish to thank him for these
conversations.  We would also like to thank Jeffrey Adams and Geo Kam
Fai for helpful conversations as well.

\section{Notation and Preliminaries} \label{section:notation}

Throughout, $k$ will denote a finite field, and
$k_n$ will denote the degree $n$ extension of
$k$.  $K$ will denote a nonarchimedean local field of
charcteristic zero, $\mathcal{O}_K$ its ring of integers,
$\mathfrak{p}_K$ the maximal ideal in $\mathcal{O}_K$,
and $k$ will also denote the residue field of $K$.  Fix
a uniformizer $\varpi$ of $K$.  Let $K^u$ denote the maximal
unramified extension of $K$, and fix a valuation $\val : (K^u)^\times
\rightarrow \mathbb{Z}$ normalized so that $\val(\varpi) = 1$.
Write $K_n$ for the unramified extension of $K$ of degree $n$, $k_n$ for
the degree $n$ extension of $k$,
and set $\Gamma_n = \Gal(K_n/K) = \Gal(k_n/k)$.

A geometric Frobenius is an element of $\Gal(\bar{K}/K)$
inducing the automorphism $x \mapsto x^{1/p}$ of $\bar{k}$.  Under the
Artin reciprocity map of local class field theory the choice of uniformizer $\varpi$
determines a geometric Frobenius $\Fr$ \cite[\S 2]{serre1}.

If $\chi : K^{\times} \rightarrow \mathbb{C}^{\times}$ is a character, we define
the \emph{level} of $\chi$ to be the smallest integer $r$ such that
$\chi|_{1 + \mathfrak{p}_K^{r+1}} \equiv 1$, with
$\chi|_{1 + \mathfrak{p}_K^{r}} \not\equiv 1$.

If $T$ is a torus defined over $K$ we write $X^*(T)$
for the character lattice $\Hom_{\bar{K}}(T, \Gm)$ and $X_*(T)$ for the
cocharacter lattice $\Hom_{\bar{K}}(\Gm, T)$ \cite[\S 16.2]{humphreys}.
$T$ will split over an extension
$L$ of $K$ if and only if $\Gal(\bar{K}/L)$ acts trivially on $X^*(T)$.
We may thus define \emph{the} splitting field $L$ of $T$ as the
minimal extension of $K$ splitting $T$; note that $L$ is necessarily
Galois over $K$.  Write $\Gamma$ for $\Gal(L/K)$; $X_*(T)$, $X^*(T)$ and $T(L)$
are all $\Gamma$-modules.

Let $\hat{T}$ be the dual torus over $\CC$ with $X_*(\hat{T}) = X^*(T)$ and
$X^*(\hat{T}) = X_*(T)$.  By transport of structure $\hat{T}$ is also a $\Gamma$-module.
For $\lambda \in X_*(T)$ and $\mu \in X^*(T)$ we write $\la \lambda, \mu \ra$
for the canonical $\Gamma$-equivariant pairing
$$\Hom_{\bar{K}}(\Gm, T) \times \Hom_{\bar{K}}(T, \Gm) \rightarrow \Hom_{\bar{K}}(\Gm, \Gm) \cong \ZZ.$$

Suppose now that $T \subset G$ for a connected reductive group $G$ over $K$.
We write $N$ for the normalizer $\Normalizer{T}{G}$ of $T$ in $G$ and define $W = N/T$;
set $\hat{N} = \Normalizer{\hat{T}}{\hat{G}}$ and
$\hat{W} = \hat{N}/\hat{T}$.  The identification of $X^*(T)$ and $X_*(\hat{T})$
yields a canonical anti-isomorphism between $W$ and $\hat{W}$.
Note that $W$ is a scheme over $K$; in general $W(K) \ne N(K) / T(K)$.

Write $\Nm$ for the map
\begin{align*}
T(L) &\rightarrow T(K) \\
t &\mapsto \prod_{\sigma \in \Gamma} \sigma(t)
\end{align*}
and for its restriction to $X_*(T)$.

The following theorem, due to Lang \cite{lang}, underpins the facts in
Section \ref{section:padic_tori} on tori over $p$-adic fields.
Let $H$ be a commutative connected algebraic group over a
finite field $k$, and suppose $H$ splits over $k_n$.

\begin{theorem} \label{thm:lang}
$\HT{i}(\Gamma_n, H(k_n)) = 0$ for all $i$.
\end{theorem}
\begin{proof}
Since $\Gamma_n$ is cyclic,
$\HT{i}(\Gamma_n, H(k_n)) \cong \HT{i+2}(\Gamma_n, H(k_n))$ \cite[Thm. 5]{atiyah-wall},
so it suffices to prove the result for $i=1$ and $i=2$, which is done
by Serre \cite[\S VI.6]{serre2}.
\end{proof}

\section{Rectifiers for $\GL_{n}(K)$} \label{section:BH_recall}

In this section, we recall the notion of a rectifier, 
as in \cite{bushnellhenniart}, and describe the rectifier in the setting
that we will need.

An irreducible smooth representation of the Weil group $\Weil_K$ of $K$ is
called \emph{essentially tame} if its restriction to wild inertia is a
sum of characters.  Let us recall the classical construction of the
essentially tame local Langlands correspondence for $\GL_{n}(K)$.  In the
essentially tame case, there is a natural bijection 
$\varphi_{\xi} \leftrightarrow (L/K, \xi)$, between irreducible representations
$\varphi_{\xi} : \Weil_K \rightarrow \GL_{n}(\mathbb{C})$, and 
\emph{admissible pairs} $(L/K, \xi)$. 
Here, $L/K$ is a degree $n$ separable extension
and $\xi$ is a character (with certain conditions) of $\Lx$.
Note that one can view 
$L^{\times}$ as a maximal elliptic torus in $\GL_{n}(K)$.  Howe
constructs a map (see \cite{howe})
\begin{equation*}
\left\{
\begin{array}{cc}
\mathrm{isomorphism \ classes \ of} \\
\mathrm{admissible \ pairs}
\end{array}
\right\} \rightarrow \left\{
\begin{array}{cc}
\mathrm{supercuspidal \ representations} \\
\mathrm{of} \ \GL_{n}(K)
\end{array} \right\}
\end{equation*}
$$\hspace{-.5in} (L/K, \xi) \mapsto \pi_{\xi}$$
The problem is that the obvious map, $$\varphi_{\xi} \mapsto \pi_{\xi},$$
the so-called ``naive correspondence'', is not the local Langlands
correspondence because $\pi_{\xi}$ has the wrong central character.
Instead, the local Langlands correspondence is given by $$\varphi_{\xi}
\mapsto \pi_{\xi \cdot {}_K \mu_{\xi}}$$ for some subtle finite order
character ${}_K \mu_{\xi}$ of $\Lx$.  The function $${}_K
\boldsymbol\mu : (L/K, \xi) \mapsto {}_K \mu_{\xi}$$ is called the
\emph{rectifier} associated to $L/K$.  Both the description of and the
intuition behind the rectifiers ${}_K \boldsymbol\mu$ have been
studied (see \cite{bushnellhenniart}, \cite{geo}, \cite{adrian}).

We will need a description of the characters ${}_K \mu_{\xi}$ in certain cases.
Let us recall some notions from \cite[\S8]{bushnellhenniart}.

Let $(L/K,\xi)$ be an admissible pair, and let $i \in \mathbb{Z}, i \geq 0$.
Then there is a unique sub-extension $L_i/K$ of $L/K$ such that
$\xi|_{1 + \mathfrak{p}_L^{i+1}}$ factors through the norm $\Nm_{L/L_i}$,
and which is minimal for this property.  We say that $i \in \mathbb{Z}$
is a \emph{jump of} $\xi$ \emph{over} $K$ if $i \geq 1$ and $L_{i-1} \neq L_i$.

\begin{proposition}\label{prop:BH_result1}
  Suppose that $(L/K, \xi)$ is an admissible pair, where $L/K$
  is unramified and $\xi$ has level $0$.
  Then ${}_K \mu_{\xi}$ is unramified and
  ${}_K \mu_{\xi}(\varpi) = (-1)^{n-1}$.
\end{proposition}

\proof
It is clear that the set of jumps of $\xi$ over $F$ is empty.
Therefore, by \cite[Proposition 21]{bushnellhenniart}, we have the result.
\qed

\begin{proposition}\label{prop:BH_result2}
Suppose that $(L/K,\xi)$ is an admissible pair, where $L/K$ is unramified
of degree $n$ and where $\xi$ has
level $r \geq 1$.  Suppose that $\xi|_{1 + \mathfrak{p}_L^r}$ does not
factor through $\Nm_{L/L_i}$ for any proper subextension $L_i$ contained
in $L$, containing $K$.  Then ${}_K \mu_{\xi}$ is unramified and
  ${}_K \mu_{\xi}(\varpi) = (-1)^{n-1}$.
\end{proposition}

\proof
It is clear that $L_0 = L_1 = \cdots L_{r-1} = L$ and $L_{j} = K$
for all $j \geq r$.  By Theorem ??????????? \MAxxx{This is the 
Ramification Theorem of Essentially Tame, III, but once
we find out what the error in the paper is.} and 
\cite[Remark 21]{bushnellhenniart}, we have the result.
\qed

\section{Tori over $p$-adic fields} \label{section:padic_tori}

Let $T$ be a torus defined over $K$.    Let $K_n$ be the maximal
unramified subextension of the splitting field $L$, and set $I = \Gal(L/K_n)$.

Let $\TT$ be the N\'eron model of $T$, a canonical model of $T$
over $\OK$ \cite[Ch. 10]{blr}.  As a consequence of the N\'eron mapping
property, we may identify $\TT(\OK)$ with $T(K)$.  The connected
component of the identity, $\TT^\circ$, cuts out a subgroup
$T(K)_0 = \TT^\circ(\OK)$ of $T(K)$; we also write $T(K_n)_0$ for
$\TT^\circ(\OKn)$.

In fact, this subgroup of $T(K)$ is the first in a decreasing filtration.
Moy and Prasad \cite{moyprasad}
define one such filtration by
embedding $T$ into an induced torus and defining the filtration of
$\Res_{L/K} \Gm$ in terms of the valuation on $L$.  Yu \cite[\S 5]{yu}
describes a different filtration, agreeing with that of Moy and Prasad
in the case of tame tori but with nicer features in the presence of wild
ramification.  Let $\TT_r$ be the integral model of $T$ defined in Yu's
minimal congruent filtration, and let $\{T(K)_r\}_{r \ge 0}$ and
$\{T(K_n)_r\}_{r \ge 0}$ be the corresponding filtrations of $T(K)$ and
$T(K_n)$.

Let $\C$ be the scheme of
connected components of $\TT$, which we may identify with the
components of $\TT \times \Spec(k)$ since $T = \TT \times \Spec(K)$
is connected.  The structure of $\C$ is described by Xarles:

\begin{proposition}[{\cite[Cor. 2.12]{xarles}}]
There is an exact sequence of $\Gamma_n$-modules
$$0 \rightarrow \Hom_{\ZZ}(\HH^1(I, X^*(T)), \QQ/\ZZ) \rightarrow
\C \rightarrow \Hom_{\ZZ}(X^*(T)^I, \ZZ) \rightarrow 0.$$
\end{proposition}

\begin{corollary}[{\cite[Thm. 1.1]{xarles}}] \label{cor:unram_components}
If $T$ is unramified, then $\C \cong X_*(T)$.
\end{corollary}

Using our filtration of $T(K_n)$, we may relate the cohomology of $T(K_n)$
with that of $\C$.

\begin{proposition}\label{prop:T0_cohom_triv}
$\HT{i}(\Gamma_n, T(K_n)_0) = 0$ for all $i$.
\end{proposition}
\begin{proof}
Note that
$$T(K_n)_0 = \invlim{r} T(K_n)_0 / T(K_n)_r.$$
So by a result of Serre \cite[Lem. 3]{serre1}, it suffices to prove that
$\HT{i}(\Gamma_n, T(K_n)_r / T(K_n)_{r+}) = 0$ for all $i$.  But $T(K_n)_r / T(K_n)_{r+}$
is connected \cite[Prop. 5.2]{yu} and thus has trivial cohomology by
Theorem \ref{thm:lang}.
\end{proof}

\begin{corollary}
$\HT{i}(\Gamma_n, T(K_n)) \cong \HT{i}(\Gamma_n, \C)$.
\end{corollary}

\begin{proof}
This follows from the long exact sequence in cohomology associated to the sequence
$$0 \rightarrow \TT^0 \rightarrow \TT \rightarrow \C \rightarrow 0.$$
\end{proof}

\begin{corollary} \label{cor:cohom_tori}
If $T$ is unramified, then $\HT{i}(\Gamma_n, T(K_n)) \cong \HT{i}(\Gamma_n, X_*(T))$
for all $i$.
\end{corollary}

\begin{proof}
This follows from the previous corollary together with Corollary \ref{cor:unram_components}.
\end{proof}

\begin{corollary}\label{cor:vanishing_H0}
If $T$ is unramified and anisotropic, then $\HT{0}(\Gamma_n, T(K_n)) = 0$.
\end{corollary}

\begin{proof}
Since $T$ is anisotropic $\HH^0(\Gamma_n, T(K_n)) \cong X_*(T)^{\Gamma_n} = 0$.
\end{proof}

\section{Groups of type L} \label{section:groups_of_type_L}

We now review the theory of ``groups of type L'' due to Benedict
Gross.  For a torus $T$ over $K$, recall that the dual torus $\hat{T}$ is equipped with
an action of $\Gamma$.

\begin{definition}
A \emph{group of type L} is a group extension of $\Gamma$ by $\hat{T}$.
\end{definition}

For such a group $D$ we have by definition an exact sequence
$$1 \rightarrow \hat{T} \rightarrow D \rightarrow \Gamma \rightarrow 1.$$

We now describe how, given a Langlands parameter
$$\varphi : \Weil_K \rightarrow D,$$
where $D$ is a group of type L, we can naturally attach a character of the coinvariants
$T(L)_{\Gamma}.$
Restricting $\varphi$ to $\Weil_L$ we get a homomorphism
$$\varphi|_{\Weil_L} : \Weil_L \rightarrow \hat{T}.$$
By the Langlands correspondence for tori, this gives us a character
$\xi_{\varphi} : T(L) \rightarrow \CCx$.  Since $\varphi|_{\Weil_L}$ extends
to $\varphi$, one can see that
$$\xi_{\varphi}(t^{\sigma}) = \xi_{\varphi}(t)\ \mbox{for all $\sigma \in \Gamma$.}$$
Therefore, $\xi_{\varphi}(t^{\sigma - 1}) = 1$ for all $\sigma \in \Gamma$.
Thus, $\xi_{\varphi}$ is trivial on the augmentation ideal $I_{\Gamma}(T(L))$
and gives $$\xi_{\varphi} : T(L)_\Gamma \rightarrow \CCx$$ Invariants
and coinvariants are related by the norm map
$$\Nm : T(L) \rightarrow T(K)$$ $$t \mapsto \displaystyle\prod_{\sigma \in \Gamma} \sigma(t)$$
in the Tate cohomology sequence
$$1 \rightarrow \HT{-1}(\Gamma,T(L)) \rightarrow T(L)_{\Gamma} \xrightarrow{\Nm} T(K)
  = T(L)^{\Gamma} \rightarrow \HT{0}(\Gamma,T(L)) \rightarrow 1$$
(note that the norm map $\Nm$ factors to $T(L)_{\Gamma}$).
We note that $T(L)_{\Gamma}$ is a cover of
$\Nm(T(L)_{\Gamma})$, which is a subgroup of $T(K)$.  It is sometimes
the case that $\Nm$ is surjective, in which case $\xi_{\varphi}$ is then a
character of $T(L)_{\Gamma}$, which is a cover of $T(K)$.

In order to define our notion of admissible pair in section \ref{section:general_rectifiers},
we will need the following lemma.

\begin{lemma} \label{lem:weyl_groups}
Let $G$ be a connected reductive $K$-group and let $T$ be a maximal
$K$-torus of $G$.
\begin{enumerate}
\item $\Normalizer{T(K)}{G(L)} / T(L) \cong W(K)$.
\item The standard action of $\Normalizer{T(L)}{G(L)} / T(L)$ on $T(L)$ determines
well-defined actions of $\Normalizer{T(L)}{G(L)}^\Gamma / T(K)$ and $W(K)$
on $T(L)$, which factor naturally to actions on $T(L)_\Gamma$.
\end{enumerate}
\end{lemma}

\begin{proof}
See \cite[Lem. 9.1]{adrianlansky}.
\end{proof}

We will need the following structural result about Langlands
parameters mapping to groups of type L later.  Suppose that $L/K$ is now unramified.
Suppose $\varphi$ and $\varphi'$ are two Langlands parameters,
with $\varphi'(\Fr) \varphi(\Fr)^{-1} \in \hat{T}$.
Let $\xi_{\varphi}$ and $\xi_{\varphi'}$ be the associated characters of $T(L)_{\Gamma}$.

\begin{lemma} \label{lem:toral_modification}
$\xi_{\varphi}$ and $\xi_{\varphi'}$ have the same restriction to $\HT{-1}(\Gamma, T(L))$.
\end{lemma}

\begin{proof}
It suffices to prove that $\xi_{\varphi'} \cdot \xi_{\varphi}^{-1}$ vanishes on
$\ker(\Nm : T(L) \rightarrow T(K))$.  Define $g \in D$ and $t \in \hat{T}$ by
$\varphi(\Fr) = g$, $\varphi'(\Fr) = tg$.  Then
\begin{align*}
\varphi'(\Fr^n) \varphi(\Fr^n)^{-1} &= (tg)^n g^{-n} \\
&= \prod_{i=0}^{n-1} g^i t g^{-i} \\
&= \prod_{i=0}^{n-1} \Fr^i(t)
\end{align*}
since $g$ projects to $\Fr \in \Gamma$.  Define $\varphi_i \colon \Weil_L \rightarrow \hat{T}$
by $\varphi_i(z) = 1$ for $z \in I_L$ and
$\varphi_i(\Fr^n) = \Fr^i(t)$; let $\xi_i$ be the associated character
of $T(L)$.  By \cite[Lem. 4.3.1]{debackerreeder}, $\xi_i = \xi_0 \circ \Fr^i.$
Suppose that $x \in T(L)$ with $\Nm(x) = 1$.  Then
\begin{align*}
\xi_{\varphi'}(x) \xi_{\varphi}(x)^{-1} &= \prod_{i=0}^{n-1} \xi_i(x) \\
&= \xi_0 \left(\prod_{i=0}^{n-1}\Fr^i(x)\right) \\
&= 1.\\
\end{align*}
\end{proof}

\section{The relationship between the Gross construction and the DeBacker--Reeder and Reeder construction}
\label{section:gross_debacker_reeder}

Let $\varphi : \Weil_K \rightarrow {}^L G$ be a regular semisimple elliptic Langlands
parameter for an unramified connected reductive group $G$
(see \cite{debackerreeder} and \cite{reeder}).
Here, ${}^L G = \langle \hat{\theta} \rangle \ltimes \hat{G}$,
where $\hat{\theta}$ is the dual Frobenius automorphism on $\hat{G}$
(see \cite[\S 3]{debackerreeder}).
Note that $\varphi$ has image in a group of
type L.  Let $L,K,T,\hat{T}, \Gamma, \xi_{\varphi}$ be as in
\S\ref{section:groups_of_type_L}.
Then $\varphi(I_K) \subset \hat{T}$ and
$\varphi(\Fr) = \hat{\theta} f$ for some $f \in \hat{N}$.  Let $\hat{w}$
be the image of $f$ in $\hat{W}$.
DeBacker--Reeder and Reeder (in \cite{debackerreeder}
and \cite{reeder}) associate a character $\chi_{\varphi}$ of $T(K)$ to $\varphi$.

Let us fix a splitting $(\hat{T}, \hat{B}, \{X_{\alpha} \})$
for the dual group $\hat{G}$.  Here $\{ X_{\alpha} \}$ is a set of root vectors
indexed by the set of simple roots of $\hat{T}$ in $\hat{B}$.
For each simple root $\alpha$, let $\phi_{\alpha} : \SL_2 \rightarrow \hat{G}$
be defined by $\phi_{\alpha}(\mathrm{diag}(z,1/z)) = \alpha^{\vee}(z)$
and $d \phi_{\alpha}\mat{0}{1}{0}{0} = X_{\alpha}$. Let
$\sigma_{\alpha} = \phi_{\alpha}\mat{0}{1}{-1}{0}$.

\begin{definition}
  The Tits group $\widetilde{W}$ is the subgroup of $\hat{N}$
  generated by $\{\sigma_{\alpha} \}$ for $\alpha$ simple.
\end{definition}

For each simple root $\alpha$, let $m_{\alpha} = \sigma_{\alpha}^2 = \alpha^{\vee}(-1)$.
Let $\hat{T}_2$ be the subgroup of $\hat{T}$ generated by the $m_{\alpha}$.

\begin{theorem}{(\cite{tits})}
\begin{enumerate}

\item The kernel of the natural map $\widetilde{W} \rightarrow \hat{W}$
  is $\hat{T}_2$,
\item The elements $\sigma_{\alpha}$ satisfy the braid relations,
\item There is a canonical lifting of $\hat{W}$ to a subset of
  $\widetilde{W}$: take a reduced expression $w = s_{\alpha_1} \cdots s_{\alpha_n}$,
  and let $\tilde{w} = \sigma_{\alpha_1} ... \sigma_{\alpha_n}$.
\end{enumerate}
\end{theorem}

\begin{definition}
Given $\hat{u} \in \hat{W}$, let $\tilde{u}$ be its canonical lift to $\widetilde{W}$.
We define a homomorphism $\varphi_{u} : \Weil_K \rightarrow {}^L G$ by
\begin{enumerate}
\item $\varphi_{u}|_{I_K} \equiv 1$
\item $\varphi_{u}(\Fr) = \hat{\theta} \tilde{u}$
\end{enumerate}
\end{definition}

By 
\S\ref{section:groups_of_type_L}, $\varphi$ and $\varphi_{w}$ gives rise to characters
$\xi_{\varphi}$ and $\xi_{\varphi_{w}}$ of $T(L)_{\Gamma}$, respectively.

\begin{lemma} \label{lem:GDR_compat}
$\xi_{\varphi}$ and $\chi_{\varphi} \circ \Nm$ have the same restriction to $T(\OL)_{\Gamma}$.
\end{lemma}

\proof
We have the exact sequence

$$1 \rightarrow \HT{-1}(\Gamma, T(L)) \rightarrow T(L)_{\Gamma} \rightarrow T(K)
  \rightarrow \HT{0}(\Gamma, T(L)) \rightarrow 1.$$

Recall that the character $\xi_{\varphi}$ is associated to $\varphi$ by
the local Langlands correspondence for tori (see \S\ref{section:groups_of_type_L}).
Note that the above exact sequence restricts to an exact sequence

$$1 \rightarrow \HT{-1}(\Gamma, T(\OL)) \rightarrow T(\OL)_{\Gamma}
  \rightarrow T(\OK) \rightarrow \HT{0}(\Gamma, T(\OL)) \rightarrow 1$$

Moreover, by Proposition \ref{prop:T0_cohom_triv}, we have
$\HT{-1}(\Gamma, T(\OL)) = \HT{0}(\Gamma, T(\OL)) = 1$.
Therefore, the map
$$T(\OL)_{\Gamma} \xrightarrow{\Nm} T(\OK)$$
is an isomorphism, so
$\xi_{\varphi}|_{T(\OL)_{\Gamma}}$ factors to a character of
$T(\OK)$ via this isomorphism.  But this is exactly how the character
$\chi_{\varphi}|_{T(\OK)}$ is constructed in \cite{debackerreeder} and \cite{reeder}.
\qed

\begin{proposition}\label{existenceofrectifier}
If $G$ is semisimple, then $\xi_{\varphi} \otimes \xi_{\varphi_{w}}^{-1} = \chi_{\varphi} \circ \Nm$.
\end{proposition}

\begin{proof}
Since $G$ is semisimple, $T(K)$ is compact.  In particular,
$\HT{0}(\Gamma, T(L)) = 0$ by Corollary \ref{cor:vanishing_H0},
so we have the following exact sequence:
$$1 \rightarrow \HT{-1}(\Gamma, T(L)) \rightarrow T(L)_{\Gamma} \rightarrow T(K) \rightarrow 1$$
Note that $T(K) = T(\OK)$, and so in particular
$T(\OL)_{\Gamma}$ surjects onto $T(K)$ via the norm map
$\Nm$.  Therefore, $\HT{-1}(\Gamma,T(L))$ and
$T(\OL)_{\Gamma}$ together generate $T(L)_{\Gamma}$.  It thus suffices to check that
$\xi_{\varphi} \otimes \xi_{\varphi_{w}}^{-1} = \chi_{\varphi} \circ \Nm$
on each of these two subgroups.

Since $\varphi_{w}|_{I_K} \equiv 1$, $\xi_{\varphi_{w}}$ is trivial on
$T(\OL)_{\Gamma}$ so Lemma
\ref{lem:GDR_compat} implies equality on $T(\OL)_{\Gamma}$.
Equality on $\HT{-1}(\Gamma,T(L))$ is Lemma \ref{lem:toral_modification}.
\end{proof}

\section{Rectifiers for general reductive groups} \label{section:general_rectifiers}

In this section, we introduce the notion of a rectifier for groups outside
of $GL_n(K)$.  

Suppose that $G$ is a connected reductive group defined over a
$p$-adic field $K$.  Let $\varphi : \Weil_K \rightarrow {}^L G$ be a
Langlands parameter for $G(K)$, and suppose that $\varphi$ factors
through the normalizer of a maximal torus. To $\varphi$, one can
associate a maximal $K$-torus $T$ in $G$, the twisted $K$-torus
associated to $\varphi(\Fr)$, where $\Fr$ is a geometric Frobenius
of $\Weil_K$ \MAxxx{Is this sentence correct?}.
As in \S\ref{section:groups_of_type_L}, one can canonically
associate to $\varphi$ a character $\xi_{\varphi}$ of $T(L)_{\Gamma}$,
the group of coinvariants of $T(L)$ with respect
to $\Gamma = Gal(L/K)$, where $L$ is the splitting field of $K$.
Recall again the Tate cohomology sequence
$$1 \rightarrow \HT{-1}(\Gamma,T(L)) \rightarrow T(L)_{\Gamma} \xrightarrow{\Nm} T(K)
= T(L)^{\Gamma} \rightarrow \HT{0}(\Gamma,T(L)) \rightarrow 1.$$
Suppose that $\HT{0}(\Gamma, T(L)) = 0$, in which case
$T(L)_{\Gamma}$ surjects onto $T(K)$.  Let us also suppose that
$\varphi$ does not factor through a proper Levi subgroup, so that the
representations in the $L$-packet associated to $\varphi$ are
conjecturally all supercuspidal (see \cite[\S 3.5]{debackerreeder}).
If $G$ happens to be $\GL_n$, one can compute that
$\HT{0}(\Gamma, T(L)) = \HT{-1}(\Gamma, T(L)) = 0$, so that
$T(L)_{\Gamma} \cong T(K)$ can be identified with $\Lx$, and
$(L/K, \xi_{\varphi})$ is an admissible pair.  To construct the local Langlands
correspondence, one would then proceed (as in \S\ref{section:BH_recall}) to
attach the supercuspidal representation $\pi_{\xi_{\varphi} \cdot {}_K
  \mu_{\xi_{\varphi}}}$ to $\xi_{\varphi}$, via the construction of Howe.

If $G$ were arbitrary, then in analogy to the case of $\GL_{n}(K)$, there
exist (in certain cases) constructions of supercuspidal $L$-packets
$\Lpack(\chi)$ associated to characters $\chi$ of $T(K)$ (see
\cite{debackerreeder}, \cite{kaletha}, \cite{reeder}).  However, as we
have seen, a Langlands parameter $\varphi$ does not naturally
provide a character of $T(K)$, but rather a character of
$T(L)_{\Gamma}$.

\begin{definition} \label{def:admissible}
Let $T$ be a $K$-minisotropic torus, 
that splits over an unramified
  extension $L$ (see \cite[\S3]{reeder}).  Suppose $\xi$ is a character of $T(L)_{\Gamma}$.
The pair $(T, \xi)$ is called \emph{admissible} if $\xi$ is not fixed
by any element of $W(K)$ (see Lemma \ref{lem:weyl_groups}). We say that
two admissible pairs $(T, \xi)$ and $(T', \xi')$ are \emph{isomorphic} if there
exists a $g \in G(K)$ such that $gT(K)g^{-1} = T'(K)$ and $\xi(t) = \xi'(gtg^{-1})$
for all $t \in T(K)$.  We denote by $P_G(K)$ the set of isomorphism classes
of admissible pairs in $G$.
\end{definition}

Recall that there is a notion of regularity for characters of $T(K)$
(see \cite[p. 802]{debackerreeder} and \cite[\S3]{reeder}).

\begin{definition} \label{def:rectifier}
  Let $T$ be a $K$-minisotropic torus in $G$, 
  that splits over an unramified
  extension $L$.  A \emph{rectifier} for $T$ is a function $${}_K
  \boldsymbol\mu : (T, \xi) \mapsto {}_K \mu_{\xi}$$ which attaches,
  to each admissible pair $(T, \xi) \in P_G(K)$, a character ${}_K
  \mu_{\xi}$ of $T(L)_{\Gamma}$ satisfying the following conditions:

\begin{enumerate}
\item The character ${}_K \mu_{\xi}$ is tamely ramified (i.e. trivial on
  $T(1 + \mathfrak{p}_L)_{\Gamma}$)

\item The character $\xi \cdot {}_K \mu_{\xi}$ descends to $T(K)$, is regular,
and $\varphi \mapsto \Lpack(\xi_{\varphi} \cdot {}_K \mu_{\xi_{\varphi}})$
  is the local Langlands correspondence.

\item If $(T, \xi_i), i = 1,2$ are admissible pairs  such that
$\xi_1^{-1} \xi_2$ is tamely ramified, then
${}_K \mu_{\xi_1} = {}_K \mu_{\xi_2}$.
\end{enumerate}
\MAxxx{Gordan, during my seminar that I gave about this stuff, asked me:
What if $T$ occurs as a torus of two different groups $G$?  Do you
get the same rectifying character attached to $(T, \xi)$?  That is,
is our function ${}_K \boldsymbol\mu$ well-defined?  Because $(T,\xi)$
could occur for different groups $G_1, G_2$.  If they do, is the associated
rectifier correct in LLC for $G_1$ and $G_2$??? I'm not even sure if
I formulated this question correctly, and to be honest, I don't know
if Gordan did either, since he didn't completely understand my talk, but
we should probably think about this type of thing...}

\end{definition}

We note that since we have assumed that $\HT{0}(\Gamma,T(L)) = 0$, the condition
that $\xi \cdot {}_K \mu_{\xi}$ descends to $T(K)$ is possible, so
condition (2) makes sense.  We believe that the assumption $\HT{0}(\Gamma,T(L)) = 0$
is not necessary (see \S\ref{section:questions}).

\begin{conjecture}
  Let $T$ be as in Definition \ref{def:rectifier}.  Then $T$
  admits a unique rectifier ${}_K \boldsymbol\mu : (T, \xi) \mapsto
  {}_K \mu_{\xi}$.
\end{conjecture}

We first note that the definition of admissible pair above generalizes
the notion of admissible pair of \cite{bushnellhenniart} in
the case of unramified tori.  Indeed,
if $G = \GL_n$, and $T$ is an elliptic torus in $G$ splitting over
an unramified extension $L/K$, then one can show that
$W(K) = \Gamma$.  Thus, for $\xi$ to not be fixed by any
element of $W(K)$ is equivalent to $\xi$ not being fixed by
any subgroup of $\Gamma$.  This can be seen to be equivalent to $\xi$
not coming from the norms $\Nm_{L/M}$ for any proper subfield $M$ contained
in $L$, containing $K$.

We next note that as the local Langlands correspondence is not known in general, we must restrict
ourselves to cases where supercuspidal $L$-packets have been constructed.
Since we are in the present paper considering the situation when $T$ is unramified,
we consider those $L$-packets constructed in \cite{debackerreeder} and \cite{reeder}.
We will show that the characters ${}_K \mu_{\xi}$ that arise in this setting
happen to be unramified (i.e. trivial on $T(\OL)_{\Gamma}$).  This implies,
in particular, that not all possible admissible pairs arise in the settings
of \cite{debackerreeder} and \cite{reeder}.
For example, if $G = \GL_n$, ${}_K \mu_{\xi}$ can be ramified,
even if $T$ is unramified.  Since $L$-packets have not been constructed in the generality that
we need in order to prove this conjecture, we determine
the rectifier in the situation where supercuspidal $L$-packets have been
constructed.

\begin{definition}\label{def:general_pair}
Suppose $(T, \xi) \in P_G(K)$.
\begin{enumerate}
\item The \emph{depth} of $(T, \xi)$ is the integer $r$ so that $\xi$
is trivial on $T(1+\mathfrak{p}_L^{r+1})_{\Gamma}$ but nontrivial on
$T(1+\mathfrak{p}_L^{r})_{\Gamma}$
\item An admissible pair of depth $r$ is in \emph{general position} 
if $\xi|_{T(1+\mathfrak{p}_L^{r})_{\Gamma}}$
is not fixed by any element of $W(K)$.  We denote by $\Pgen(K)$ the set
of admissible pairs in $G$ in general position.
\item A \emph{weak rectifier} for $T \subset G$ is a function
\begin{align*}
{}_K\mugen : (T, \xi) \mapsto {}_K \mu_{\xi}
\end{align*}
which attaches,
  to each admissible pair $(T, \xi) \in \Pgen(K)$, a character ${}_K
  \mu_{\xi}$ of $T(L)_{\Gamma}$, satisfying conditions (1)-(3) 
  of Definition \ref{def:rectifier}.
\end{enumerate}
\end{definition}
\MAxxx{IMPORTANT: These 3 conditions certainly fall into the set of admissible
pairs considered in DeBacker/Reeder and Reeder.  We don't need to find
conditions on admissible pairs $(T,\xi)$ that are equivallent to the 3
conditions on Langlands parameters that are considered on page 31
of Reeder's positive depth paper.  We just need our definition of $\Pgen(K)$
to be a subset of their pairs $(T,\xi)$ so that condition (3) of
Theorem \ref{thm:unique_semisimple} makes sense. Note that David has
an argument that condition (iii) is equivalent (or implied by?) condition (2)
on page 31 of Reeder's positive depth paper.}

\begin{theorem} \label{thm:unique_semisimple}
Let $G$ be semisimple, and $T$ as in Definition \ref{def:rectifier}.  Then
there is a unique weak rectifier for $T$.
\end{theorem}

\proof
We first prove existence.  
Let $G$ be an unramified connected reductive group.  If
$T$ is an unramified torus in $G$, then $T$ can be defined
via Galois twisting by a Weyl group element $w$.  We defined in
\S\ref{section:gross_debacker_reeder}
a Langlands parameter $\varphi_{w} :\Weil _K \rightarrow {}^L G$ by
sending Frobenius to the canonical lift 
$\tilde{w} \in \widetilde{W}$ of $\hat{w} \in \hat{W}$, and by setting
$\varphi_{w}$ to be trivial on $I_K$.
If $G$ is semisimple,
we proved in Proposition \ref{existenceofrectifier}
that the function $$(T, \xi) \mapsto \xi_{\varphi_{w}}^{-1}$$
satisfies condition (2).  Moreover, since $\varphi_{w}|_{I_K} \equiv 1$,
$\xi_{\varphi_{w}}^{-1}$ is unramified, so this function also 
satisfies condition (1).  Finally, $\xi_{\varphi_{w}}$ only depends on $T$,
and not on on $\xi$.  Therefore, condition (3) is automatically satisfied.
We may therefore set
${}_K \mugen(T,\xi) = \xi_{\varphi_{w}}^{-1}$.

We now prove uniqueness.
Let $\xi$ range over the set of characters of $T(L)_{\Gamma}$ 
such that $(T, \xi) \in P_G^{Reg}(K)$, and let 
$\xi \mapsto {}_K \mu_{\xi}^i$ be a rectifier for 
$T \subset G$, $i = 1,2$.  By hypothesis, we have 
$$\Lpack({}_K \mu_{\xi}^1 \cdot \xi) = \Lpack({}_K \mu_{\xi}^1 \cdot \xi).$$
By \cite[\S10]{murnaghan}, there exists $w_{\xi} \in W(K)$, 
depending on $\xi$, such that 
$$({}_K \mu_{\xi}^1 \cdot \xi)^{w_{\xi}} = {}_K \mu_{\xi}^2 \cdot \xi.$$  
Suppose that $\xi$ is not trivial on $T(1 + \mathfrak{p}_L)_{\Gamma}$.  
Restricting the equation 
$({}_K \mu_{\xi}^1 \cdot \xi)^{w_{\xi}} = {}_K \mu_{\xi}^2 \cdot \xi$ 
to $T(1 + \mathfrak{p}_L)_{\Gamma}$, we get that $\xi^{w_{\xi}} = \xi$, 
by condition (1) of Definition \ref{def:rectifier}.  
This contradicts admissibility of $\xi$.

Now suppose that $\xi$ is trivial on $T(1 + \mathfrak{p}_L)_{\Gamma}$.  
Consider the restriction of the equation 
$({}_K \mu_{\xi}^1 \cdot \xi)^{w_{\xi}} = {}_K \mu_{\xi}^2 \cdot \xi$ 
to $T(\mathcal{O}_L)_{\Gamma} \cong T(\mathcal{O}_K)$.  
Write $\mu = (({}_K \mu_{\xi}^1)^{w_{\xi}})^{-1} \cdot {}_K \mu_{\xi}^2$. 
Reducing mod $\mathfrak{p}_L$, we conclude that there exists a character, 
$\mu$, of $T(k)$, with the property that if $\xi$ is any 
$W(k)$-regular character of $T(k)$, then $\mu \xi$ is 
$W(k)$-conjugate to $\xi$. By ???????\MAxxx{Fill this in when we know
what to fill it in with}, $\mu$ is trivial.  
Therefore, $({}_K \mu_{\xi}^1)^{w_{\xi}} = {}_K \mu_{\xi}^2$ 
upon restriction to $T(\mathcal{O}_L)_{\Gamma}$.  
Finally, since ${}_K \mu_{\xi}^1 \cdot \xi$ and ${}_K \mu_{\xi}^2 \cdot \xi$ 
must descend to $T(K)$ by condition (2) of Definition 
\ref{def:rectifier}, we have that ${}_K \mu_{\xi}^1$ and 
${}_K \mu_{\xi}^2$ have the same restriction to 
$\hat{H}^{-1}(\Gamma, T(L))$.  Hence, 
${}_K \mu_{\xi}^1 = {}_K \mu_{\xi}^2$.
\qed

We note that in the
case that $G$ is semisimple, there was nothing special about
$\varphi_{w}$.  In fact, by the same arguments, $\xi_{\varphi_{w}} = \xi_{\varphi'}$
for any Langlands parameter $\varphi'$ such that $\varphi'|_{I_K} \equiv 1$
and $\varphi'(\Fr) = w'$, where $w'$ is any lift of $\hat{w}$ to $\hat{N}$.
However, we will see that the canonical Tits group lift $\tilde{w}$ is forced upon us when we
consider groups that are not necessarily semisimple, such as $\GL_{n}(K)$.

We also note for emphasis that even though ${}_K  \mugen((T, \xi))$ is
unramified for all pairs $(T, \xi) \in \Pgen(K)$, it will not be
unramified for more general pairs $(T, \xi)$, as noted earlier.

\section{Compatibility with Bushnell-Henniart} \label{section:BH_compat}

In this section, we show that our function ${}_K \mugen$
agrees with the rectifier of Bushnell/Henniart.
Let $L/K$ be the degree $n$ unramified extension of $K$.
Let $T = \Res_{L/K}(\Gm)$.

\begin{proposition}
$\HT{0}(\Gamma, X_*(T)) = 0$.
\end{proposition}

\begin{proof}
$\Gamma$ acts on $X_*(T)$ by cyclic shift.
Therefore, $X_*(T)^{\Gamma} = \mathbb{Z}$, embedded diagonally in
$\mathbb{Z}^n = X_*(T)$.  Note that $\Nm(1,0,0,\cdots,0) = (1,1,\cdots,1)$, so
$X_*(T)^{\Gamma} \subset \Nm(X_*(T))$.
\end{proof}

\begin{proposition}
$\HT{-1}(\Gamma, X_*(T)) = 0$.
\end{proposition}

\begin{proof}
$(a_1, a_2, \cdots, a_n) \in \ker(\Nm)$ if and only if $\displaystyle\sum_{i=1}^n a_i = 0$.
It is then easy to see that $\ker(\Nm)$ is generated by $e_i - e_j$ for $i < j$, where
$e_i$ are the standard basis of $\mathbb{Z}^n$.  But $e_i - e_j = (1 - \tau)e_i$ for some
$\tau \in \Gamma$, since $\Gamma$ acts by cyclic shift.  Thus $\ker(\Nm) \subset I_{\Gamma}(X_*(T))$.
\end{proof}

The Tate cohomology exact sequence for $T$ therefore reduces to
$$1 \rightarrow T(L)_{\Gamma} \xrightarrow{\sim} T(K) \rightarrow 1$$ by
Corollary \ref{cor:cohom_tori}.

\begin{proposition}\label{prop:powers_of_lifts}
Let $\hat{w}$ be a Coxeter element of $\GL_{n}(\CC)$.  Let $\tilde{w}$ be the
canonical lift of $\hat{w}$ to $\widetilde{W}$. Then $\tilde{w}^n = (-Id)^{n-1}$, where $Id$
denotes the identity element in $\GL_{n}(\CC)$.
\end{proposition}

\begin{proof}
See \cite[\S3.1]{zaremsky}.
\end{proof}

By \cite[\S 2.4]{serre1}, $\varpi$ corresponds to $\Fr^n$ under the Artin
reciprocity map for $L$, so we obtain

\begin{corollary} \label{cor:rectifier_agreement}
$\xi_{\varphi_{w}}^{-1}$ is unramified and
$\xi_{\varphi_{w}}^{-1}(\varpi) = (-1)^{n-1}$.
\end{corollary}

\begin{proof}
Note that $T(L) \cong \Lx \times \Lx \times \cdots \times \Lx$, and
that $T(K) = \{(w, \sigma(w), \sigma^2(w), \cdots, \sigma^{n-1}(w)) : w \in \Lx \} \cong \Lx$, where
$\sigma$ is a generator of $\Gal(L/K)$.  A uniformizer $\varpi$ in $\Lx$
therefore corresponds to $(\varpi, \varpi, \cdots, \varpi) \in T(K)$, whose
preimage under $\Nm$ is the class of $(\varpi, 1, 1, \cdots, 1)$ in $T(L)_{\Gamma}$.
The local Langlands correspondence for tori says now that
$\xi_{\varphi_{w}}(\varpi) = (-1)^{n-1}$, because of Proposition \ref{prop:powers_of_lifts}.
Finally, since $\varphi_{w}|_{I_F} \equiv 1$, we get that $\xi_{\varphi_{w}}$ is
unramified.
\end{proof}

By Propositions \ref{prop:BH_result1} and \ref{prop:BH_result2}, we obtain

\begin{theorem}
  If $G = \GL_{n}(K)$, the function $(T,\xi) \mapsto \xi_{\varphi_{w}}^{-1}$ agrees with
  the rectifier of Bushnell/Henniart, when $(T,\xi) \in \Pgen(K)$.  
\end{theorem}

We end this section by explaining why the Tits group lift $\tilde{w}$ is forced upon us.
Suppose we define
$\varphi' : \Weil_K \rightarrow \GL_{n}(\CC)$ by $\varphi'|_{I_K} \equiv 1$ and
$\varphi'(\Fr)$ to be a representative of an elliptic element $\hat{w}$ in $\hat{W}$.
Then \cite[p. 824]{debackerreeder} and \cite[\S6]{reeder} imply that the characteristic
polynomial of $\varphi'(\Fr)$ is $X^n - a$, for some $a \in \CCx$.  One can see that,
by analogous arguments as in Corollary \ref{cor:rectifier_agreement},
$\xi_{\varphi'}(\varpi) = a$.  By Propositions \ref{prop:BH_result1} and
\ref{prop:BH_result2}, we are
forced to set $a = (-1)^{n-1}$.  Finally, one can show by an inductive argument that the
canonical lift $\tilde{w}$ to $\widetilde{W}$ has characteristic polynomial $X^n - (-1)^{n-1}$,
so that $\varphi'(\Fr)$ is indeed the canonical lift of $\hat{w}$ to $\widetilde{W}$.

\section{Further Questions} \label{section:questions}

We would like to note that in the case that $\HT{0}(\Gamma, T(L))
\neq 0$, the situation seems more difficult because twisting $\xi$
by a character of $T(L)_{\Gamma}$
can only result in a character of the image
of $T(L)_{\Gamma}$ under the norm map.  However, one might be
able to remedy this with a prediction of central character, as in
\cite{grossreeder}, for example.

\begin{thebibliography}{9}

\bibitem{adrian}
  Moshe Adrian,
  \emph{A new realization of the Langlands correspondence for $PGL(2,F)$}, Journal of Number Theory 133 (2013) 446-–474.

\bibitem{adrian1}
  Moshe Adrian
  \emph{On the Local Langlands Correspondences of DeBacker/Reeder and Reeder for $GL(\ell,F)$, where $\ell$ is prime}, Pacific Journal of Mathematics 255-2 (2012), 257--280.

\bibitem{adrianlansky}
  Moshe Adrian and Joshua Lansky,
  \emph{A real groups construction of the tame local Langlands correspondence for $PGSp(4,F)$}, preprint, arXiv:1209.6045.

\bibitem{amano}
  Kazuo Amano,
  \emph{A note on the Galois cohomology groups of algebraic tori}, Nagoya Math. J. Volume 34 (1969), 121-127.

\bibitem{atiyah-wall}
  Michael Atiyah and Charles Wall,
  \emph{Cohomology of Groups}, 84-115, in Algebraic Number Theory, John W. S. Cassels and Albrecht Frohlich (eds.).  Academic Press, London, 1967.

\bibitem{blr}
  Siegfried Bosch, Werner L\"utkebohmert, and Michel Reynaud.
  \emph{N\'eron Models}. Springer-Verlag, Berlin, 1980.

\bibitem{bushnellhenniart}
  Colin Bushnell, Guy Henniart,
  \emph{The essentially tame local Langlands correspondence, III: the general case}, Proc. Lond. Math. Soc. (3) 101 (2010), no. 2, 497–553.

\bibitem{debackerreeder}
  Stephen DeBacker and Mark Reeder,
  \emph{Depth-zero supercuspidal $L$-packets and their stability.}
  Ann. of Math. (2) 169 (2009), no. 3, 795--901.

\bibitem{geo}
  Geo Kam-Fai Tam,
  \emph{Transfer relations in essentially tame local Langlands correspondence}, Ph.D. thesis, University of Toronto, 2012.

\bibitem{grossreeder}
  B. Gross and M. Reeder,
  \emph{Arithmetic invariants of discrete Langlands parameters.}  Duke Math. Journal, 154, (2010), 431-508.

\bibitem{howe}
  Roger Howe,
  \emph{Tamely ramified supercuspidal representations of $GL_n(F)$},
   Pacific Journal of Math.  73  (1977),  437--460.

\bibitem{humphreys}
  James Humphreys,
  \emph{Linear Algebraic Groups}, Graduate Texts in Mathematics, 21.  Springer-Verlag, New York, 1975.

\bibitem{kaletha}
  Tasho Kaletha, \emph{Simple Wild L-packets}, J. Inst. Math. Jussieu (2013) 12(1), 43-75.

\bibitem{lang}
  Serge Lang, \emph{Algebraic groups over finite fields},  Amer. J. Math. (1956) 78, 555�563.

\bibitem{moyprasad}
  \emph{A. Moy and G. Prasad},
  Unrefined minimal $K$-types for $p$-adic groups,
   Invent. Math. 116, no. 1-3, 393-408 (1994).

%\bibitem{moyprasad1}
%  Allen Moy, Gopal Prasad,
%  \emph{Jacquet functors and unrefined minimal $K$-types},
%   Comment. Math. Helv. 71 (1996), no. 1, 98--121.

\bibitem{murnaghan}
  Fiona Murnaghan,
  \emph{Parametrization of tame supercuspidal representations}, pp.439-470 in On Certain L-functions: A volume in honour of Freydoon Shahidi on the occasion of his 60th birthday , Clay Math. Proc. 13 (2011), edited by J. Arthur, J.W. Cogdell, S. Gelbart, D. Goldberg, D. Ramakrishnan.

\bibitem{reeder}
  Mark Reeder,
  \emph{Supercuspidal $L$-packets of positive depth and twisted Coxeter elements},
  J. Reine Angew. Math. 620 (2008), 1-33.

\bibitem{serre2}
  Jean-Pierre Serre,
  \emph{Algebraic Groups and Class Fields}, Graduate Texts in Mathematics, 117. Springer-Verlag, New York, 1988.

\bibitem{serre}
  Jean-Pierre Serre,
  \emph{Local Fields}, Graduate Texts in Mathematics, 67. Springer-Verlag, New York-Berlin, 1979.

\bibitem{serre1}
  Jean-Pierre Serre,
  \emph{Local Class Field Theory}, 129-162, in Algebraic Number Theory, John W. S. Cassels and Albrecht Frohlich (eds.).  Academic Press, London, 1967.

\bibitem{tits}
  Jacques Tits,
  \emph{Normalisateurs de tores. I. Groupes de Coxeter etendus}, J. Algebra, 4:96-116,1966.

\bibitem{xarles}
  Xavier Xarles, \emph{The scheme of connected components of the N�ron model of an algebraic torus},
  Journal f�r die reine und angewandte Mathematik 437 (1993): 167-180.

\bibitem{yu}
  Jiu-Kang Yu, \emph{Smooth models associated to concave functions in Bruhat-Tits theory}. Preprint, 2003.

\bibitem{zaremsky}
  Matthew Zaremsky,
  \emph{Representatives of elliptic Weyl group elements in algebraic groups}, preprint, arxiv:1109.5487.

\end{thebibliography}

\end{document}
