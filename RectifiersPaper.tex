\documentclass[11pt]{amsart}
\usepackage{amsmath,amscd,amssymb,latexsym, amsfonts}
\theoremstyle{plain}
\newtheorem{iassumption}{Assumption}
\newtheorem{theorem}{Theorem}[section]
\newtheorem{conjecture}[theorem]{Conjecture}
\newtheorem{proposition}[theorem]{Proposition}
\newtheorem{corollary}[theorem]{Corollary}
\newtheorem{hypothesis}[theorem]{Hypothesis}
\newtheorem{assumption}[theorem]{Assumption}
\newtheorem{lemma}[theorem]{Lemma}
\newtheorem{question}[theorem]{Question}
\newtheorem{exercise}[theorem]{Exercise}
\newtheorem{statement}[theorem]{Statement}
\newtheorem{example}[theorem]{Example}

\newcommand{\MAxxx}[1]{$\clubsuit$\footnote{#1}}
\newcommand{\DRxxx}[1]{$\spadesuit$\footnote{#1}}
\newcommand{\HT}[1]{\hat{\HH}{}^{#1}}

\theoremstyle{definition}
\newtheorem{definition}[theorem]{Definition}
\newtheorem{remark}[theorem]{Remark}

\setlength{\oddsidemargin}{0.2in}
\setlength{\evensidemargin}{0.2in}
\setlength{\textwidth}{6.1in}

\DeclareMathOperator{\Gal}{Gal}
\DeclareMathOperator{\val}{val}
\DeclareMathOperator{\HH}{H}
\DeclareMathOperator{\Ad}{Ad}
\DeclareMathOperator{\Nm}{Nm}
\DeclareMathOperator{\Hom}{Hom}

\newcommand{\mat}[4]{\left( \begin{array}{cc} {#1} & {#2} \\ {#3} & {#4}
\end{array} \right)}

\begin{document}
\title{Rectifiers and the local Langlands Correspondence : the unramified case}
\author{Moshe Adrian and David Roe}
%\address{Department of Mathematics, University of Utah, Salt Lake City, UT 84112, U.S.A.}

%\email{savin@math.utah.edu}


\begin{abstract}

\maketitle

In this paper we propose a natural generalization of rectifiers, the
character twists that appear in the local Langlands correspondence for
$GL(n,F)$.  Our generalization is to arbitrary connected reductive
groups, with certain classes of Langlands parameters.

\end{abstract}

\section{Introduction}
In this paper, we propose a natural generalization of the notion of a
rectifier.  In particular, we define rectifiers for Langlands
parameters of arbitrary connected reductive groups that map into a
``group of type L''.  We then describe these rectifiers via their
associated Langlands parameters, and show that they are compatible
with the rectifiers of $GL(n,F)$ as in \cite{bushnellhenniart}.

An irreducible smooth representation of the Weil group $W_F$ of $F$ is
called \emph{essentially tame} if its restriction to wild inertia is a
sum of characters.  Let us recall the classical construction of the
essentially tame local Langlands correspondence for $GL(n,F)$.  In the
essentially tame case, an irreducible representation $\varphi : W_F
\rightarrow GL(n,\mathbb{C})$ naturally provides an \emph{admissible
  pair} $(E/F, \xi)$.  Here, $E/F$ is a degree $n$ separable extension
and $\xi$ is a character (with certain conditions) of $E^{\times}$,
which one can view as a maximal elliptic torus in $GL(n,F)$.  Howe
constructs a map (see \cite{howe})
\begin{equation*}
\left\{
\begin{array}{ll}
isomorphism \ classes \ of \\
admissible \ pairs
\end{array}
\right\} \rightarrow \left\{
\begin{array}{ll}
supercuspidal \ representations \\
of \ GL(n,F)
\end{array} \right\}
\end{equation*}
$$\hspace{-.5in} (E/F, \xi) \mapsto \pi_{\xi}$$
The problem is that the obvious map, $$\varphi \mapsto \pi_{\xi},$$
the so-called ``naive correspondence'', is not the local Langlands
correspondence because $\pi_{\xi}$ has the wrong central character.
Instead, the local Langlands correspondence is given by $$\varphi
\mapsto \pi_{\xi \cdot {}_F \mu_{\xi}}$$ for some subtle finite order
character ${}_F \mu_{\xi}$ of $E^{\times}$.  The function $${}_F
\boldsymbol\mu : (E/F, \xi) \mapsto {}_F \mu_{\xi}$$ is called the
\emph{rectifier} associated to $E/F$.  Both the description of and the
intuition behind the rectifiers ${}_F \boldsymbol\mu$ have been
studied (see \cite{bushnellhenniart}, \cite{geo}, \cite{adrian}).

One may ask whether rectifiers exist for groups apart from $GL(n,F)$,
and even moreso, how does one define the notion of rectifier for
groups apart from $GL(n,F)$.  The answer can be gleamed from a recent
construction of Benedict Gross, which we briefly describe now (for
more details, see \S\ref{groupsoftypeL}).

Suppose that $G$ is a connected reductive group defined over a
$p$-adic field $F$.  Let $\phi : W_F \rightarrow {}^L G$ be a
Langlands parameter for $G(F)$, and suppose that $\phi$ factors
through the normalizer of a maximal torus. To $\phi$, one can
associate a maximal $F$-torus $T$ in $G$, the twisted $F$-torus
associated to $\phi(\Phi)$, where $\Phi$ is the geometric Frobenius
\MAxxx{geometric or arithmetic?  does it even make sense to say ``the
  geometric Frobenius of $W_F$''?} of $W_F$.  Suppose that $T$ splits
over the Galois extension $E$ of $F$, and set $\Gamma = \Gal(E/F)$.
Restricting $\phi$ to $W_E$, by the local Langlands correspondence for
tori, one can canonically associate to $\phi$ a character $\xi_{\phi}$
of $T(E)_{\Gamma}$, the group of coinvariants of $T(E)$ with respect
to $\Gamma$ (see~\S\ref{groupsoftypeL}).  Invariants and coinvariants
are related by the norm map
$$\Nm : T(E) \rightarrow T(F)$$ $$t \mapsto \displaystyle\prod_{\sigma \in \Gamma} \sigma(t)$$
in the cohomology sequence
$$1 \rightarrow \HT{-1}(\Gamma,T(E)) \rightarrow T(E)_{\Gamma} \xrightarrow{\Nm} T(F)
= T(E)^{\Gamma} \rightarrow \HT{0}(\Gamma,T(E)) \rightarrow 1,$$
where $\hat{H}$denotes Tate cohomology.

Suppose that $\HT{0}(\Gamma, T(E)) = 0$, in which case
$T(E)_{\Gamma}$ is then a cover of $T(F)$.  Let us also suppose that
$\phi$ does not factor through a proper Levi subgroup, so that the
representations in the $L$-packet associated to $\phi$ are
conjecturally all supercuspidal (see \cite[\S 3.5]{debackerreeder}).
If $G$ happens to be $GL(n)$, one can compute that
$\HT{0}(\Gamma, T(E)) = \HT{-1}(\Gamma, T(E)) = 0$, so that
$T(E)_{\Gamma} \cong T(F)$ can be identified with $E^{\times}$, and
$(E/F, \xi)$ is an admissible pair.  To construct the local Langlands
correspondence, one would then proceed (as explained earlier) to
attach the supercuspidal representation $\pi_{\xi \cdot {}_F
  \mu_{\xi}}$ to $\xi$, via the construction of Howe.

If $G$ were arbitrary, then in analogy to the case of $GL(n,F)$, there
exist in certain cases constructions of supercuspidal $L$-packets
$L(\xi)$ associated to characters $\xi$ of tori $T(F)$ in $G(F)$ (see
\cite{debackerreeder}, \cite{kaletha}, \cite{reeder}).  However, as we
have seen, a Langlands parameter $\phi$ as above does not naturally
provide a character of $T(F)$, but rather a character of
$T(E)_{\Gamma}$.

\begin{definition}
Let $T$ be a torus, defined over $F$, that splits over an unramified
  extension $E$.  Suppose $\xi$ is a character of $T(E)_{\Gamma}$.
The pair $(T, \xi)$ is called \emph{admissible} if $\xi$ is not fixed
by any element of $(N(G,T)/T)(F)$ (see Lemma \ref{weylgroups}).
\end{definition}

\begin{definition}\label{rectifierdefinition}
  Let $T$ be a torus, defined over $F$, that splits over an unramified
  extension $E$.  A \emph{rectifier} for $T$ is a function $${}_F
  \boldsymbol\mu : (T, \xi) \mapsto {}_F \mu_{\xi}$$ which attaches,
  to each admissible pair $(T, \xi)$, a character ${}_F
  \mu_{\xi}$ of $T(E)_{\Gamma}$ satisfying the following conditions.

\begin{enumerate}
\item $\xi \cdot {}_F \mu_{\xi} \in \widehat{T(F)}$

\item The character ${}_F \mu_{\xi}$ is tamely ramified (i.e. trivial on
  $T(1 + \mathfrak{p}_E)_{\Gamma}$)

\item $\phi \mapsto L(\xi_{\phi} \cdot {}_F \mu_{\xi_{\phi}})$
  is the local Langlands correspondence.

\item If $(T, \xi_i), i = 1,2$ are admissible pairs  such that
$\xi_1^{-1} \xi_2$ is tamely ramified, then
${}_F \mu_{\xi_1} = {}_F \mu_{\xi_2}$.
\end{enumerate}

\end{definition}

\begin{conjecture}
  Let $T$ be as in Definition \ref{rectifierdefinition}.  Then $T$
  admits a unique rectifier ${}_F \boldsymbol\mu : (T, \xi) \mapsto
  {}_F \mu_{\xi}$.\MAxxx{Should we assume that the Langlands parameters
  are ``regular''?  Since this is what we know for LLC.  Then we might
  as well assume that the rectifiers are unramified as well.}
\end{conjecture}

As the local Langlands correspondence is not known in general, we must restrict
ourselves to cases where supercuspidal $L$-packets have been constructed.
Since we are in the present paper considering the situation when $T$ is unramified,
we consider those $L$-packets constructed in \cite{debackerreeder} and \cite{reeder}.
We will show that the characters ${}_F \mu_{\xi}$ that arise in this setting
are actually unramified (i.e. trivial on $T(\mathcal{O}_E)_{\Gamma}$).  In particular,
the setting
of \cite{debackerreeder} and \cite{reeder} do not include all possible admissible
pairs $(T, \xi)$.  In particular, because of condition (3) above, we cannot hope
to say anything at present about the situation where ${}_F \mu_{\xi}$ is ramified,
except possibly in the case where $G = GL(n)$ (note that for
$G = GL(n)$, ${}_F \mu_{\xi}$ can be ramified even if $T$ is unramified).
We return to this case in another paper.

Since $L$-packets have not been constructed in the generality that
we need in order to prove this conjecture, we prove a theorem that
almost proves the conjecture in its entirety.  Let $Reg(T, \xi)$ denote
the set of admissible pairs that arise from RSELPs,
as in \cite{debackerreeder} and \cite{reeder}.  We can prove:

\begin{theorem}\label{regularrectifierdefinition}
Let $G$ be semisimple, and $T$ as in Definition \ref{rectifierdefinition}.
There exists a unique function $${}_F
  \boldsymbol\mu^{\mathrm{Reg}} : Reg(T, \xi) \mapsto {}_F \mu_{\xi}$$ which attaches,
  to each admissible pair $(T, \xi) \in Reg(T,\xi)$, a character ${}_F
  \mu_{\xi}$ of $T(E)_{\Gamma}$ satisfying the following conditions.

\begin{enumerate}
\item $\xi \cdot {}_F \mu_{\xi} \in \widehat{T(F)}$

\item The character ${}_F \mu_{\xi}$ is unramified (i.e. trivial on
  $T(\mathcal{O}_E)_{\Gamma}$)

\item $\phi \mapsto L(\xi_{\phi} \cdot {}_F \mu_{\xi_{\phi}})$
  agrees with \cite{debackerreeder} and \cite{reeder}

\item If $(T, \xi_i), i = 1,2$ are admissible pairs  such that
$\xi_1^{-1} \xi_2$ is unramified, then
${}_F \mu_{\xi_1} = {}_F \mu_{\xi_2}$.
\end{enumerate}
\end{theorem}

Once phrased in this manner, uniqueness is a straightforward matter
(see \S\ref{}) \MAxxx{Fix when we can.}.

As explained earlier, this is all that we can hope to do,
as more general $L$-packets in the case of $T$ unramified have not been
constructed.  Once $L$-packets are constructed more generally, we hope to
construct the rectifier in general.  The existence of this function
will be proven by interpreting the characters ${}_F \mu_{\xi}$
in terms of a canonical Langlands parameter.  More explicitly,
we will exhibit a canonical
\MAxxx{Canonical?  Is the Tits group lift canonical?} Langlands
parameter $\phi_{o}$, such that if $\xi_o = \xi_{\phi_o}$, then the
function $$(T, \xi) \mapsto \xi_o$$ is satisfies conditions (1)-(4).

Let $\phi, T, E, \xi$ as before, assuming that $\phi$ is RSELP.
The projection of $\phi(\Phi)$ onto the connected component of ${}^L
G$ is a lift of some Weyl group element $\hat{w}$ to the normalizer of
the dual torus.  Associated to $\phi$, DeBacker/Reeder construct a
character $\chi_{\phi}$ of $T(F)$, to which they associate a
conjectural $L$-packet of representations $L(\chi_{\phi})$ (see \S
\ref{preliminaries}).

\begin{definition}
Given $\phi$, we define a homomorphism $\phi_o : W_F \rightarrow {}^L G$ by
\begin{enumerate}
\item $\phi_o|_{I_F} \equiv 1$
\item $\phi_o(\Phi) = \hat{\theta} \tilde{w}$
\end{enumerate}
\end{definition}

where $\hat{\theta}$ is the geometric
\MAxxx{geometric or arithmetic? or neither?}
Frobenius element generating the action of Galois on
the dual group, and where $\tilde{w}$ is the canonical lift
\MAxxx{Is this lift really canonical?}
of $\hat{w}$ to the Tits group (see \S\ref{mainresults}).
Let $\xi_o = \xi_{\phi_o}$.  Our main theorem is the following.

\begin{theorem}
  If $G$ is semisimple, then $$(T, \xi) \mapsto
  \xi_o$$ satisfies conditions (1)-(4) of Definition \ref{regularrectifierdefinition}.
\end{theorem}

Therefore, we have may uniquely define 
${}_F \boldsymbol\mu^{\mathrm{Reg}}(T,\xi) = \xi_o$ for $T,\xi) \in Reg(T,\xi)$

In the case that $G$ is semisimple, there was nothing special about
$\phi_o$.  In fact, by the same arguments that we will make, a
rectifier is obtained from any Langlands parameter $\phi'$ such that
$\phi'|_{I_F} \equiv 1$ and $\phi'(\Phi) = w'$, where $w'$ was any
lift of $\hat{w}$ to the normalizer of the dual torus.  However, the
following theorem (which we will prove) will show that $\phi_o$ is
forced upon us.
\MAxxx{Up to a sign, or something like that?}

\begin{theorem}
  If $G = GL(n,F)$, the character ${}_F \boldsymbol\mu$ agrees with
  the rectifier of Bushnell/Henniart.  Moreover, $\phi_o$ is the
  unique \MAxxx{up to a sign, or something like that} Langlands
  parameter such that $\xi_{\phi_o} = {}_F \mu_{\xi}$.
\end{theorem}\MAxxx{This theorem is false, because our rectifier is unramified, but theirs can be ramified.
So when we define $\phi_o$, say that this is a definition in the case of maybe regular Langlands parameters
(where regular is condition 2 in Reeder's positive depth paper...probably we need some other conditions on
our Langlands parameters, like maybe conditions 1 and 3 of Reeder's positive depth paper), or something like
that. It's clearly not a definition for all rectifiers of unramified tori. Just in the case of at least
DeBacker/Reeder and Reeder}

We would like to note that in the case that $\HT{0}(\Gamma, T(E))
\neq 0$, the situation seems more difficult $\xi$ is defined on the
image of $T(E)_{\Gamma}$ under the norm map.  However, one might be
able to remedy this with a prediction of central character, as in
\cite{grossreeder}, for example.

Acknowledgements: Part of this paper was heavily influenced by
conversations with Gordan Savin.  We wish to thank him for these
conversations.  We would also like to thank Jeffrey Adams and Geo Kam
Fai for helpful conversations as well.

\section{Notation and Definitions}

Throughout, $k$ will denote a finite field, and
$k_n$ will denote the degree $n$ extension of
$k$.  $K$ will denote a nonarchimedean local field of
charcteristic zero, and $k$ will be its residue field.  Fix
a uniformizer $\varpi$ of $K$.  Let $K^u$ denote the maximal
unramified extension of $K$, and fix a valuation $\val : (K^u)^\times
\rightarrow \mathbb{Z}$ normalized so that $\val(\varpi) = 1$.

\section{Preliminaries}

If $S$ is a torus that is defined over an arbitrary field $F$ and
$E/F$ is a Galois extension, then we have the norm map
$$\Nm_{E/F} : S(E) \rightarrow S(F)$$ $$\ \ \ \ \ \ \ \ \ \ \ \ \ \ \ \ \ s \mapsto
\prod_{\sigma \in \Gal(E/F)} \sigma(s)$$

\section{Tori over finite fields}

Let $\mathbb{T}$ be a torus that is defined over a finite field
$k$.  Suppose $\mathbb{T}$ splits over an extension
$k_n$ and set $G_n = \Gal(k_n/k)$.
We wish to compute $\HT{i}(G_n,\mathbb{T}(k_n))$.
Recall that, since $G_n$ is cyclic, $\HT{i}(G_n,A)\cong
\HT{i+2}(G_n,A)$ for all $i$, and all $G_n$-modules $A$.

\begin{proposition}\label{trivialH1finitefields}
$\HT{1}(G_n,\mathbb{T}(k_n))$ is trivial.
\end{proposition}

\proof
This is Lang's theorem for connected algebraic groups over finite fields.
\qed

\begin{corollary}
$\HT{0}(G_n,\mathbb{T}(k_n))$ is trivial.
\end{corollary}

\proof
It is well-known
\DRxxx{add a reference}
that the Herbrand quotient of a finite group is trivial.  Hence, the result follows.
\qed

\section{Unramified Tori over p-adic fields}

Let $T$ be a torus defined over $K$ that splits over an unramified
extension $L/K$ of degree $n$, set $G = \Gal(L/K)$, and let $g$ be a
generator of $G$.  Recall that, since $G$ is cyclic,
$\HT{i}(G,A)\cong \HT{i+2}(G,A)$ for all $i$, and all
$G$-modules $A$.  Let $X_*(T)$ denote the cocharacter lattice of $T$ and
$X^*(T)$ denote the character lattice of $T$.  As in \cite[Section 3]{moyprasad1},
we define $T_0$ to be the maximal bounded subgroup of
$T(K^u)$, and
$$T_r = \{t \in T_0 : \val(\chi(t) - 1) \geq r \mbox{ for } \chi \in X^*(T) \}$$
for any natural number $r$.  We may
furthermore define $T_r^L = T_r \cap T(L)$ for any field $L$ such
that $K \subset L \subset K^u$.  We wish to compute $\HT{i}(G, T(L))$.
We use the filtration $T_r^L$ on $T(L)$ in the following way.

It is clear that $\Nm_{L/K}$ maps $T_r^L$ into $T_r^K$ for all $r \geq 0$.
Therefore, we get an induced map
$$\Nm_{L/K} : T_s^L / T_{r}^L \rightarrow T_s^K / T_{r}^K$$
for all $r > s \geq 0$.

\begin{lemma}\label{H1compactpart}
$\HT{1}(G, T_{s}^L) = 0$ for $s \geq 0$.
\end{lemma}

\proof
We prove the case $s = 0$, as the other cases are similar.  Recall
that $T_0^L = \underleftarrow{\mathrm{lim}} \ T_0^L / T_{r+1}^L$.  We
need the following Lemma:

\begin{lemma}\label{abstractcohomology}
  Let $M$ be a $G$-module complete with respect to a topology given by
  a sequence of $G$-invariant subgroups $M = M_0 \supset M_1 \supset
  M_2 \supset ...$, so $\cap M_i = 0$.  Thus,
  $M = \underleftarrow{\mathrm{lim}} \ M / M_i$.
  If $\HH^1(G, M_i / M_{i+1}) = 0$ for $i \geq 0$,
  then $\HH^1(G, M_i) = 0$ for $i \geq 0$.
\end{lemma}

\proof

We do the case $j = 0$.  The argument for $j \geq 1$ is similar.  Let
$f : G \rightarrow M$ be an element in $\HH^1(G,M)$.  We have a natural
map $\HH^1(G, M ) \rightarrow \HH^1(G, M / M_1) = 0$.  Thus, there exists
$m_0 \in M$ such that $f(\sigma) = (\sigma(m_0) - m_0) (\mathrm{mod} \
M_1)$.  Then define $f_1 = f - (\sigma(m_0) - m_0)$.  This is in
$\HH^1(G, M_1)$. But then again we have the natural map
$\HH^1(G, M_1) \rightarrow \HH^1(G, M_1 / M_2) = 0$.  Thus, again, there
exists $m_1 \in M_1$ such that $f_1(\sigma) = (\sigma(m_1) - m_1) (\mathrm{mod} \ M_2)$.
So define $f_2 = f - (\sigma(m) - m) - (\sigma(m_1) - m_1)$.  Continue
this process indefinitely.  Then define
$\tilde{m} = (m_0, m_0 + m_1, m_0 + m_1 + m_2, m_0 + m_1 + m_2
+ m_3, \ldots) \in \underleftarrow{\mathrm{lim}} \ M / M_i = M$.  Then,
$f(\sigma) = \sigma(\tilde{m}) - \tilde{m}$, so we get finally that
$\HH^1(G,M) = 0$.  \qed

Let $r \geq 1$.  Note that $T_r^L / T_{r+1}^L = X_*(T) \otimes (1 +
\mathfrak{p}_L^r) / (1 + \mathfrak{p}_L^{r+1}) = X_*(T) \otimes
k_n$.  Well, $X_*(T) \otimes k_n$ is a vector space
over $k$.  But it is known that all forms of vector spaces
are equivalent.  Therefore, the vector space $X_*(T) \otimes
k_n$ with the above Galois action is equivalent to $X_*(T)
\otimes k_n$ with the standard action, i.e. $g \in G$ acts
on $x \otimes z$ by $x \otimes gz$.  Thus, we wish to calculate
$\HT{1}(G, X_*(T) \otimes k_n)$ where $X_*(T) \otimes
k_n$ is given the standard action.  Since $X_*(T) \otimes
k_n$ is just a direct sum of copies of $k_n$, we
are reduced to calculating $\HT{1}(G, k_n)$.  But since
the trace map over finite fields is surjective, we get that
$\HT{0}(G, k_n) = 0$.  Therefore, since the Herbrand
quotient of a finite group is trivial, we get that $\HT{1}(G,
k_n) = 0$.  Finally, $\HT{1}(G, T_0^L / T_1^L) = 0$ by
Lang's theorem since $T_0^L / T_1^L$ is a connected algebraic group
over a finite field.  By Lemma \ref{abstractcohomology}, we have
concluded the proof of Lemma \ref{H1compactpart}.  \qed

\begin{lemma}\label{H0compactpart}
$\HT{0}(G,T_0^L) = 0$.
\end{lemma}

\proof
We claim that the natural maps
$$\Nm_{L/K} : T_r^L / T_{r+1}^L \rightarrow T_r^K / T_{r+1}^K$$

are surjective for $r \geq 0$.  First let $r \geq 1$.  First note that
$T_0^L = X_*(T) \otimes \mathcal{O}_L^\times$ and
$T_r^L = X_*(T) \otimes (1 + \mathfrak{p}_L^r)$ for $r \geq 1$.  Then
$$\Nm_{L/K} : T_r^L / T_{r+1}^L \rightarrow T_r^K / T_{r+1}^K$$
becomes
$$\Nm_{L/K} : X_*(T) \otimes \left( (1 + \mathfrak{p}_L^r) / (1 +
  \mathfrak{p}_L^{r+1}) \right) \rightarrow (X_*(T) \otimes \left( (1 +
  \mathfrak{p}_L^r) / (1 + \mathfrak{p}_L^{r+1}) \right) )^G$$
since $T_r^K / T_{r+1}^K = T_r^G / T_{r+1}^G = (T_r / T_{r+1})^G$ since
$\HH^1(G, T_{r+1}) = 0$ by Lemma \ref{H1compactpart}.  But
$(1 + \mathfrak{p}_L^r) / (1 + \mathfrak{p}_L^{r+1}) \cong \mathcal{O}_L / \mathfrak{p}_L \cong k_n$.
Thus, we get
$$\Nm_{L/K} : X_*(T) \otimes k_n \rightarrow (X_*(T) \otimes k_n)^G$$

Well, $X_*(T) \otimes k_n$ is a vector space over
$k$.  But it is known that all forms of vector spaces are
equivalent.  Therefore, the vector space $X_*(T) \otimes k_n$
with the above Galois action is equivalent to $X_*(T) \otimes k_n$
with the standard action, i.e. $g \in G$ acts on $x \otimes z$ by
$x \otimes gz$.  Therefore,
$$\Nm_{L/K} : X_*(T) \otimes k_n \rightarrow (X_*(T) \otimes k_n)^G$$
becomes
$$\Nm_{L/K} : X_*(T) \otimes k_n \rightarrow X_*(T) \otimes k,$$
which is surjective since the trace map $k_n \rightarrow k$
is surjective.  Now let $r = 0$.  We wish to show that
$$\Nm_{L/K} : T_0^L / T_1^L \rightarrow T_0^K / T_1^K$$
is surjective.  But this map becomes
$$\Nm_{L/K} : X_*(T) \otimes \left( \mathcal{O}_L^\times / (1 + \mathfrak{p}_L) \right) \rightarrow (X_*(T) \otimes \left( \mathcal{O}_L^\times / (1 + \mathfrak{p}_L) \right) )^G$$
which is just
$$\Nm_{L/K} : \mathbb{T}(k_n) \rightarrow \mathbb{T}(k).$$
This last map is surjective by Lemma \ref{normtorifinitefields}.
Finally, by \cite[Lemma 2, p. 81]{serre1}, we have our result.
\qed

\begin{corollary}\label{reductiontori}
$\HT{i}(G, X_*(T) \otimes L^\times) = \HT{i}(G, T(L)) = \HT{i}(G,X_*(T))$ for $i=0,1$.
\end{corollary}

\proof

Recall that $X_*(T) \otimes L^\times \xrightarrow{\sim} T(L)$ given by
evaluation, restricting to an isomorphism $X_*(T) \otimes \mathcal{O}_L^\times
\xrightarrow{\sim} T_0^L$.  We may thus identify $X_*(T)$ with the quotient
$T(L) / T_0^L$, giving rise to the short exact sequence
$$1 \rightarrow T_0^L \rightarrow T(L) \rightarrow X_*(T) \rightarrow 1.$$
We therefore have the long exact sequence
$$... \rightarrow \HT{0}(G, T_0^L) \rightarrow \HT{0}(G, T(L)) \rightarrow \HT{0}(G, X_*(T))
\rightarrow \HT{1}(G, T_0^L) \rightarrow $$ $$\HT{1}(G, T(L))
\rightarrow \HT{1}(G, X_*(T)) \rightarrow \HT{0}(G, T_0^L)
\rightarrow ...$$
By corollary \ref{H0compactpart} and Lemma \ref{H1compactpart}, we get the result.
\qed

\begin{corollary}
If $T(K)$ is compact, we have $\HT{0}(G, T(L)) = 0$.
\end{corollary}

\proof
$T(K)$ being compact means that $X_*(T)^G = 0$, hence the result follows by corollary \ref{reductiontori}
\qed

\begin{corollary}
  Let $\Phi$ be the characteristic polynomial of $g$ acting on $X_*(T)$. If
  $\Phi(1)\neq 0$, i.e. $T(K)$ is compact, then $\HH^0(G, X_*(T)\otimes
  L^{\times})=0$ and the order of $\HH^1(G, X_*(T)\otimes L^{\times})$ is
  $\Phi(1)$.
\end{corollary}

\proof

Since $0 = g^n - 1 = (1 + g + g^2 + ... + g^{n-1})(g-1)$ and since $1$
is not an eigenvalue of $g$ on $X_*(T)$, we get that $1 + g + g^2 + \cdots + g^{n-1}$
is the zero map on $X_*(T)$.  Now, we have that
$\HH^1(G, X_*(T)\otimes L^{\times}) = \HH^1(G, X_*(T)) = \ker(\Nm_{L/K}) / (1 - g)X_*(T)$
where $\Nm_{L/K}$ is the norm map $\Nm_{L/K} : X_*(T) \rightarrow X_*(T)$ on $X_*(T)$.
Since $1 + g + g^2 + \cdots + g^{n-1}$ is the zero map on $X_*(T)$,
we get that $\HH^1(G, X_*(T)) = X_*(T) / (1 - g)X_*(T)$.  It is known by standard lattice
theory that the cardinality of $X_*(T) / (1 - g)X_*(T)$ is equal to the volume of
a fundamental domain for the lattice $(1 - g)X_*(T)$.  It is also known that
the volume of a fundamental domain for the lattice $(1 - g)X_*(T)$ is equal to
$|\det(1 - g)|$.  But $|\det(1 - g)|$ is just $|\Phi(1)|$.  But
$|\Phi(1)| = \Phi(1)$ for the following reason.  Note that since
$1 + g + g^2 + \cdots + g^{n-1} = 0$, $\Phi$ divides a power of
$1 + x + x^2 + \cdots + x^{n-1}$ (since the minimal polynomial of $g$
divides $1 + x + x^2 + \cdots + x^{n-1}$).  Since
$1 + 1 + 1^2 + \cdots + 1^{n-1} > 0$, we have that $\Phi(1) > 0$,
so that $|\Phi(1)| = \Phi(1)$.
\qed

\section{Review of Construction of DeBacker and Reeder}\label{preliminaries}

We first review some of the basic theory from \cite{debackerreeder}.
We first fix a pinning $(\hat{T}, \hat{B}, \{x_{\alpha} \})$ for the
dual group $\hat{G}$.  The operator $\hat{\theta}$ dual to $\theta$
extends to an automorphism of $\hat{T}$.  There is a unique extension
of $\hat{\theta}$ to an automorphism of $\hat{G}$, satisfying
$\hat{\theta}(x_{\alpha}) = x_{\theta \cdot \alpha}$ (see
\cite[section 3.2]{debackerreeder}).  Following \cite{debackerreeder},
we may define ${}^LG$ as the semidirect product ${}^L G = \ <\hat{\theta}> \ltimes \hat{G}$.

\begin{definition}
  Let $W_F'$ denote the Weil-Deligne group.  A Langlands parameter
  $\phi : W_F' \rightarrow {}^L G$ is called a \emph{tame regular
    semisimple elliptic Langlands parameter} (abbreviated TRSELP) if

  \begin{enumerate}
    \item $\phi$ is trivial on $I_F^+$,

    \item The centralizer of $\phi(I_F)$ in $\hat{G}$ is a torus.

    \item $C_{\hat{G}}(\phi)^o = (\hat{Z}^{\hat{\theta}})^o$, where $\hat{Z}$ denotes the center of $\hat{G}$.
  \end{enumerate}
\end{definition}

Condition (2) forces $\phi$ to be trivial on $SL(2,\mathbb{C})$.  Let
$\hat{N} = N_{\hat{G}}(\hat{T})$.  After conjugating by $\hat{G}$, we
may assume that $\phi(I_F) \subset \hat{T}$ and $\phi(\Phi) = \hat{\theta} f$,
where $f \in \hat{N}$.  Let $\hat{w}$ be the image of $f$ in
$\hat{W}_o$, and let $w$ be the element of $W_o$ corresponding to $\hat{w}$.

Let $\phi$ be a TRSELP with associated $w$ and set $\sigma = w \theta$.
$\sigma$ is an automorphism of $X_*(T)$.  Let $\hat{\sigma}$ be the
automorphism dual to $\sigma$, and let $n$ be the order of $\sigma$.
We set $\hat{G}_{ab} = \hat{G} / \hat{G}'$, where $\hat{G}'$ denotes
the derived group of $\hat{G}$. Let
${}^L T_{\sigma} = \langle \hat{\sigma} \rangle \ltimes \hat{T}$.
Associated to $\phi$, DeBacker--Reeder (see \cite[Chapter
4]{debackerreeder}) define a $\hat{T}$-conjugacy class of Langlands
parameters

\begin{equation}
\phi_T : W_F \rightarrow {}^L T_{\sigma} \ \label{phiT}
\end{equation}

\noindent as follows.  Set $\phi_T = \phi$ on $I_F$, and
$\phi_T(\Phi) = \hat{\sigma} \ltimes \tau$ where $\tau \in \hat{T}$
is any element whose class in $\hat{T} / (1 - \hat{\sigma}) \hat{T}$
corresponds to the image of $f$ in $\hat{G}_{ab} / (1 - \hat{\theta}) \hat{G}_{ab}$
under the bijection

\begin{equation}
\hat{T} / (1 - \hat{\sigma}) \hat{T} \stackrel{\sim}{\rightarrow} \hat{G}_{ab} / (1 - \hat{\theta}) \hat{G}_{ab} \ \label{bijectionfortau}
\end{equation}

In \cite[Chapter 4]{debackerreeder}, DeBacker and Reeder construct a
canonical bijection between $\hat{T}$-conjugacy classes of admissible
homomorphisms $\phi : W_t \rightarrow {}^L T_{\sigma}$ and depth-zero
characers of $T^{\Phi_{\sigma}}$ where
$\Phi_{\sigma} := \sigma \otimes \Phi^{-1}$.  We briefly summarize
this construction. Let $\mathbb{T} = X_*(T) \otimes \mathfrak{F}^\times$.
Given automorphisms $\alpha, \beta$ of abelian groups $A,B$,
respectively, let $\Hom_{\alpha, \beta}(A,B)$ denote the set of
homomorphisms $f : A \rightarrow B$ such that $f \circ \alpha = \beta \circ f$.
The twisted norm map
$$\Nm_{\sigma} : \mathbb{T}^{\Phi_{\sigma}^n} \rightarrow \mathbb{T}^{\Phi_{\sigma}}$$
given by
$\Nm_{\sigma}(t) = t \Phi_{\sigma}(t) \Phi_{\sigma}^2(t) \cdots \Phi_{\sigma}^{n-1}(t)$
induces isomorphisms
$$\Hom(\mathbb{T}^{\Phi_{\sigma}},
  \mathbb{C}^\times) \stackrel{\sim}{\rightarrow}
  \Hom_{\Phi_{\sigma}, 1}(\mathbb{T}^{\Phi_{\sigma}^n},
  \mathbb{C}^\times) \stackrel{\sim}{\rightarrow}
  \Hom_{\Phi_{\sigma}, 1}(X_*(T) \otimes k_n^\times, \mathbb{C}^\times)$$
Moreover, the map $s \mapsto \chi_s$ gives an isomorphism
$$\Hom_{\Phi, \hat{\sigma}}(k_n^\times, \hat{T})
  \stackrel{\sim}{\rightarrow} \Hom_{\Phi_{\sigma},
  1}(X_*(T) \otimes k_n^\times, \mathbb{C}^\times)$$
where $\chi_s(\lambda \otimes a) := \lambda(s(a))$.  The canonical
projection $I_t \rightarrow k_m^\times$ induces an isomorphism
as $\Phi$-modules
$I_t / (1 - \Ad (\Phi)^m)I_t \stackrel{\sim}{\rightarrow} k_m^\times$
where $\Ad$ denotes the adjoint action.  Since $\hat{\sigma}$ has order
$n$, we have $\Hom_{\Phi, \hat{\sigma}}(k_n^\times,
\hat{T}) \cong \Hom_{\Ad(\Phi), \hat{\sigma}}(I_t, \hat{T})$.
Therefore, the map $s \mapsto \chi_s$ is a canonical bijection
$$\Hom_{\Ad(\Phi), \hat{\sigma}}(I_t, \hat{T})
  \stackrel{\sim}{\rightarrow} \Hom(\mathbb{T}^{\Phi_{\sigma}},
  \mathbb{C}^\times)$$
Moreover, we have an isomorphism
$${}^0T^{\Phi_{\sigma}} \times X_*(T)^{\sigma} \stackrel{\sim}{\rightarrow}
  T^{\Phi_{\sigma}}$$ $$(\gamma, \lambda) \mapsto \gamma \lambda(\varpi)$$
where ${}^0 T$ is the group of $\mathcal{O}_{F^u}$-points of $\mathbf{T}$.

Finally, note that $\hat{T} / (1 - \hat{\sigma}) \hat{T}$ is the
character group of $X_*(T)^{\sigma}$, whereby
$\tau \in \hat{T} / (1 - \hat{\sigma}) \hat{T}$ corresponds to
$\chi_{\tau} \in \Hom(X_*(T)^{\sigma}, \mathbb{C}^\times),$ where
$\chi_{\tau}(\lambda) := \lambda(\tau)$.  Therefore, we have a
canonical bijection between $\hat{T}$-conjugacy classes of admissible
homomorphisms $\phi : W_t \rightarrow {}^L T_{\sigma}$ and depth-zero
characters

\begin{equation}
\chi_{\phi} := \chi_s \otimes \chi_{\tau} \in \mathrm{Irr}(T^{\Phi_{\sigma}}) \ \ \label{chitau}
\end{equation}

\noindent where $s := \phi|_{I_t}$, $\phi(\Phi) = \hat{\sigma} \ltimes \tau$,
and where we have inflated $\chi_s$ to ${}^0 T^{\Phi_{\sigma}}$.  To
$\chi_{\phi}$, DeBacker and Reeder construct an $L$-packet of
supercuspidal representations, which we denote $L(\chi_{\phi})$.

\section{Groups of type L}\label{groupsoftypeL}
We now review the theory of ``groups of type L'' due to Benedict
Gross.  Let $F$ be a field, $F^{\mathrm s}$ a separable closure, and
$T$ a torus defined over $F$ that splits over an extension $E \subset
F^s$.  Let $\Gamma = \Gal(E/F)$.  Let $X^*(T)$ be the character module
of $T$ and $X_*(T)$ the cocharacter module of $T$.  Define
$\hat{T} = X^*(T) \otimes \mathbb{C}^\times$.  The group $\Gamma$ acts on
$\hat{T}$ via its action on $X^*(T)$.

\begin{definition}
A \emph{group of type L} is a group extension of $\Gamma$ by $\hat{T}$.
\end{definition}

Let $D$ be such a group.  Then we have an exact sequence
$$1 \rightarrow \hat{T} \rightarrow D \rightarrow \Gamma \rightarrow 1$$

We now describe how, given a Langlands parameter
$$\phi : W_F \rightarrow D,$$
where $D$ is a group of type L, we can naturally attach a character of
$T(E)_{\Gamma} := T(E) / I_{\Gamma}(T(E))$, where
$I_{\Gamma}(T(E)) = \{(1 - \sigma)t \ : t \in T(E), \sigma \in \Gamma \}$.
Restricting $\phi$ to $W_E$ we get a homomorphism
$$\phi|_{W_E} : W_E \rightarrow \hat{T}.$$
By the Langlands correspondence for tori, this gives us a character
$\chi : T(E) \rightarrow \mathbb{C}^\times$.  Since $\phi|_{W_E}$ extends
to $\phi$, one can see that
$$\chi(t^{\sigma}) = \chi(t)\ \mbox{for all $\sigma \in \Gamma$.}$$
Therefore, $\chi(t^{\sigma - 1}) = 1$ for all $\sigma \in \Gamma$.
Thus, $\chi$ is trivial on the augmentation ideal $I_{\Gamma}(T(E))$
and gives $$\chi : T(E)_\Gamma \rightarrow \mathbb{C}^\times$$ Invariants
and coinvariants are related by the norm map
$$\Nm : T(E) \rightarrow T(F)$$ $$t \mapsto \displaystyle\prod_{\sigma \in \Gamma} \sigma(t)$$
in the Tate cohomology sequence
$$1 \rightarrow \HT{-1}(\Gamma,T(E)) \rightarrow T(E)_{\Gamma} \xrightarrow{\Nm} T(F)
  = T(E)^{\Gamma} \rightarrow \HT{0}(\Gamma,T(E)) \rightarrow 1$$
(note that the norm map $\Nm$ factors to $T(E)_{\Gamma}$).
We have thus constructed a character $\chi$ of $T(E)_{\Gamma}$ from a
Langlands parameter $\phi$. We note that $T(E)_{\Gamma}$ is a cover of
$\Nm(T(E)_{\Gamma})$, which is a subgroup of $T(F)$.  It is sometimes
the case that $\Nm$ is surjective, in which case $\chi$ is then a
character of $T(E)_{\Gamma}$, which is a cover of $T(F)$.

We now give meaning to the definition of \emph{admissible pair} in the introduction.

\begin{lemma}\label{weylgroups}
Let $G$ be a connected reductive $F$-group and let $T$ be a maximal
$F$-torus of $G$.  Let $E$ be the splitting field of $T$, and set
$\Gamma = \Gal(E/F)$.
\begin{enumerate}
\item $N(G(E), T(F)) / T(E) \cong (N(G,T)/T)(F)$.
\item The standard action of $N(G(E),T(E)) / T(E)$ on $T(E)$ determines
well-defined actions of $N(G(E), T(E))^{\Gamma} / T(F)$ and $(N(G,T)/T)(F)$
on $T(E)$, which factor naturally to actions on $T(E)_{\Gamma}$.
\end{enumerate}
\end{lemma}

\proof
We first prove (1).  Let $nT(E)\in N(G(E), T(F)) / T(E)$.  Let $t\in T(F)$.
Then $ntn^{-1}\in T(F)$, so for $\gamma\in\Gamma$,
$\gamma(n)t\gamma(n)^{-1} =  \gamma (ntn^{-1}) = ntn^{-1}$.
Thus $n^{-1}\gamma(n)$ centralizes $T(F)$, hence centralizes the centralizer
of $T(F)$ in $G(E)$, which is $T(E)$. Thus $n^{-1}\gamma(n)\in T(E)$ so
$\gamma(n T(E)) = \gamma (n) T(E) = n T(E)$. Thus the elements of
$N(G(E), T(F)) / T(E)$ are $\Gamma$-fixed, so
$N(G(E), T(F)) / T(E) \subset (N(G,T)/T)(F)$.

Conversely, since $T$ splits over $E$, any coset in $(N(G,T)/T)(F)$
has a representative in the normalizer $N(G(E), T(E))$. Then $\gamma(n) = nt'$
for some $t'\in T(E)$ since $nT$ is $F$-rational.  Let $t\in T(F)$.
Then for $\gamma\in\Gamma$,
$\gamma(ntn^{-1}) = \gamma(n) t \gamma(n)^{-1} = (nt')t(nt')^{-1} = ntn^{-1}$.
Thus $ntn^{-1}\in T(F)$, so $n\in N(G(E),T(F))$, as desired.

We now prove (2).
The group $N(G(E), T(E))^{\Gamma} / T(F)$ embeds naturally in $(N(G,T)/T)(F)$.
Thus it suffices to to prove the statement for
$(N(G,T)/T)(F) = N(G(E), T(F)) / T(E)$.  Let $n \in N(G(E),T(F))$.  Then since $n$
normalizes $T(F)$, $n$ also normalizes $T(E)$.  Therefore, we have that
$N(G(E), T(F)) / T(E)$ acts on $T(E)$.

We now have to show that $N(G(E), T(F)) / T(E)$
sends the augmentation ideal
$\{ t\sigma(t)^{-1} : t \in T(E), \sigma \in \Gamma \}$ to itself.
Let $t \in T(E), \sigma \in \Gamma$ and let $w \in  (N(G,T)/T)(F)$.  Then $$w\left(\frac{t}{\sigma(t)}\right)
= \frac{wt}{w \sigma(t)} = \frac{w(t)}{\sigma (\sigma^{-1}(w(\sigma(t))))} =
\frac{wt}{\sigma ({}^{\sigma^{-1}} w) (t)} = \frac{wt}{\sigma(wt)},$$
the last equality coming from the fact that $w \in (N(G,T)/T)(F)$, so $w$
is fixed by $\Gamma$.
\qed

We will need the following structural result about Langlands
parameters mapping to groups of type $L$ later.  First let $E,F,T,\hat{T}$
be as above, with $E/F$ unramified.  Let $\varphi_1 : W_F \rightarrow D$
be a Langlands parameter mapping to a group of type $L$.  $\varphi_1$
gives rise to a character $\chi_1$ of $T(E)_{\Gamma}$.  Let $\varphi_2
: W_F \rightarrow D$ be defined with the only condition that
$\varphi_2(\Phi_F) = t \varphi_1(\Phi_F)$ for some $t \in \hat{T}$.
$\varphi_2$ gives rise to a character $\chi_2$ of $T(E)_{\Gamma}$.

\begin{lemma}\label{toralmodification}
$\chi_1$ and $\chi_2$ have the same restriction to $\HT{-1}(\Gamma, T(E))$.
\end{lemma}

\proof
I want to explicitly describe the character $\chi$ restricted to
$\HT{-1}(\Gamma, T(E))$ that is attached to a Langlands parameter
$\phi$.  Here, $T$ is defined over $F$, splits over unramified $E$,
where the degree of $E/F$ is $r$.  Let $\Gamma = \Gal(E/F)$, generated
by an element $\sigma$.

We restrict $\phi$ to $W_E$.  Then $\phi(Frob_E) = \phi(Frob_F)^r$.
Suppose $\phi(Frob_F) = n$, where $n$ is an element of the normalizer
of the dual torus.  Then $n^r$ is an element in the dual torus
$\hat{T}$.

What is $\chi$ restricted to $\HT{-1}(\Gamma, T(E))$?  Well, by
stuff that Gordan and I did, and stuff that you have probably done as
well, we know that $\HT{-1}(\Gamma, T(E)) = \HT{-1}(\Gamma, X_*(T))$,
since $E/F$ is unramified.  This latter group is some quotient of
$\ker(\Nm : X_*(T) \rightarrow X_*(T))$.

Let's describe $\chi$ on $\HT{-1}$.  Well, let
$\lambda \in \ker(\Nm : X_*(T) \rightarrow X_*(T))$.  Let us think of
$\lambda$ as living in $X^*(\hat{T})$ instead.

Write $n^r = \mu \otimes z$, for some $\mu \in X_*(\hat{T})$ and
$z \in \mathbb{C}^\times$.  What is $\chi(\lambda)$?  Most likely it is
just $z^{<\mu, \lambda>}$ (where $< , >$ denotes the canonical pairing
on $X_*(\hat{T}) \times X^*(\hat{T})$ ).

Now let's modify our Langlands parameter.  Suppose
$\phi'(Frob_F) = tn$, for some element $t \in \hat{T}$, and suppose
$\phi'$ has the same restriction to inertia as $\phi$.  Then we get an
associated $\chi'$.  What is $\chi'$ restricted to
$\HT{-1}(\Gamma, T(E)) = \HT{-1}(\Gamma, X_*(T))$?

Well, let $\lambda \in \ker(\Nm : X_*(T) \rightarrow X_*(T))$.  Let us
think of $\lambda$ as living in $X^*(\hat{T})$ instead.  We need to
compute $(tn)^r$.  This is $t ntn^{-1} n^2 t n^{-2} \cdots n^{r-1} t n^{1-r} n^r$.

But $t ntn^{-1} n^2 t n^{-2} \cdots n^{r-1} t n^{1-r} = t w(t) w^2(t) \cdots w^{r-1}(t)$,
where $w$ is the Weyl group element associated to $n$.  If we write
$t = \mu' \otimes z'$, then $t w(t) w^2(t) \cdots w^{r-1}(t) = \Nm(\mu' ) \otimes z'$,
since $w$ acts as Galois on the character lattices, and since Galois
acts holomorphically on the dual group rather than antiholomorphically
(i.e. Galois acts trivially on $\mathbb{C}^\times$).  But now one checks:
We have $\chi'(\lambda) = z^{<\mu, \lambda>} (z' )^{<\Nm(\mu'), \lambda>}$.  But

$$<\Nm(\mu'), \lambda> = < \mu' + \sigma(\mu') + \sigma^2(\mu') + ... + \sigma^{r-1}(\mu'), \lambda>
  = \displaystyle\sum_{i=0}^{r-1} < \sigma^i(\mu'), \lambda >$$

Since $< , >$ is Galois equivariant, $<\sigma^i(\mu'), \lambda> =
<\mu', \sigma^{-i}(\lambda)>$.  Thus, the sum
becomes
$$\displaystyle\sum_{i=0}^{r-1} <\mu', \sigma^{-i}(\lambda)> = <\mu', \Nm(\lambda)>.$$

But $\lambda$ is assumed to be in the kernel of the norm map!  So we
get $<\mu', \Nm(\lambda)> = <\mu', 0> = 0$.  So
$\chi'(\lambda) = z^{<\mu, \lambda>} = \chi(\lambda)$.  Done.
\qed

\section{The relationship between the Gross construction and the DeBacker--Reeder construction}\label{grossdebackerreeder}

We consider here a TRSELP $\phi$ for any unramified connected
reductive group G.  Let $T$ be as in \S\ref{groupsoftypeL}.  Let $E$,
$\chi$, etc.~be as in~\S\ref{groupsoftypeL}.  Let $w$ be the Weyl
group element associated to $\phi$, and set $\sigma = w \theta\in {\rm Aut}(T)$.
Let $\chi_{\phi}$ be the character of $T(F) = T^{\Phi_{\sigma}}$ that
DeBacker and Reeder attach to $\phi$ (see section
\ref{preliminaries}).

We have the exact sequence

$$1 \rightarrow \HT{-1}(\Gamma, T(E)) \rightarrow T(E)_{\Gamma} \rightarrow T(F)
  \rightarrow \HT{0}(\Gamma, T(E)) \rightarrow 1$$

where $\Gamma = \Gal(E/F)$.  Recall that $\phi$ canonically gives
rise to a character $\chi$ of $T(E)_{\Gamma}$, by local Langlands
for tori (see section \ref{groupsoftypeL}).  Note that the above
exact sequence restricts to an exact sequence

$$1 \rightarrow \HT{-1}(\Gamma, T(\mathcal{O}_E)) \rightarrow T(\mathcal{O}_E)_{\Gamma}
  \rightarrow T(\mathcal{O}_F) \rightarrow \HT{0}(\Gamma, T(\mathcal{O}_E)) \rightarrow 1$$

Moreover, one can show using a profinite version of Lang's theorem and
various results about tori over finite fields, that since $T$ is unramified,
$\HT{-1}(\Gamma, T(\mathcal{O}_E)) = \HT{0}(\Gamma, T(\mathcal{O}_E)) = 1$.
Therefore, the map
$$T(\mathcal{O}_E)_{\Gamma} \xrightarrow{\Nm} T(\mathcal{O}_F)$$
is an isomorphism, and we may view
$\chi|_{T(\mathcal{O}_E)_{\Gamma}}$ as a character of
$T(\mathcal{O}_F)$ via this isomorphism.  Now, since $\chi$ was
obtained from $\phi|_{W_E}$ via the local Langlands correspondence, we
get that $\chi_{\phi} \circ \Nm = \chi$ on $T(\mathcal{O}_E)_{\Gamma}$.
We have therefore proven the following lemma.

\begin{lemma}\label{grossanddebackerreedercompatibility}
  The restriction to $T(\mathcal{O}_E)_{\Gamma} \xrightarrow{\sim} T(\mathcal{O}_F)$
  of the genuine character arising from the Gross construction
  coincides with the character of $T(\mathcal{O}_F)$ that is
  constructed from $\phi$ via the construction of DeBacker--Reeder
  construction.
\end{lemma}

\section{Rectifiers and DeBacker/Reeder}\label{mainresults}

Recall that we have fixed a splitting $(\hat{T}, \hat{B}, \{x_{\alpha} \})$
for the dual group $\hat{G}$ and that $\hat{\Nm} = \Nm_{\hat{G}}(\hat{T})$.
For each simple root $\alpha$, let $\phi_{\alpha} : SL(2) \rightarrow \hat{G}$
be defined by $\phi_{\alpha}(\mathrm{diag}(z,1/z)) = \alpha^{\vee}(z)$
and $\phi_{\alpha}\mat{1}{1}{0}{1} = x_{\alpha}$. Let
$\sigma_{\alpha} = \phi_{\alpha}\mat{0}{1}{-1}{0}$.

\begin{definition}
  The Tits group $\widetilde{W}$ is the subgroup of $\hat{N}$
  generated by $\{\sigma_{\alpha} \}$ for $\alpha$ simple.
\end{definition}

For each simple root $\alpha$, let $m_{\alpha} = \sigma_{\alpha}^2 = \alpha^{\vee}(-1)$.
Let $\hat{T}_2$ be the subgroup of $\hat{T}$ generated by the $m_{\alpha}$.

\begin{theorem}{(Tits \cite{tits})}
\begin{enumerate}

\item The kernel of the natural map $\widetilde{W_o} \rightarrow \hat{W}_o$
  is $\hat{T}_2$,
\item The elements $\sigma_{\alpha}$ satisfy the braid relations,
\item There is a canonical lifting of $\hat{W}_o$ to a subset of
  $\widetilde{W}$: take a reduced expression $w = s_{\alpha_1} \cdots s_{\alpha_n}$,
  and let $\tilde{w} = \sigma_{\alpha_1} ... \sigma_{\alpha_n}$.
\end{enumerate}
\end{theorem}

Let $\phi$ be a TRSELP.  Recall that $\phi(I_F) \subset \hat{T}$ and
$\phi(\Phi) = \hat{\theta} f$, where $f \in \hat{N}$.  Let $\hat{w}$
be the image of $f$ in $\hat{W}_o$, and let $w$ be the element of
$W_o$ corresponding to $\hat{w}$.  Write $\hat{w} = s_{\alpha_1} \cdots s_{\alpha_n}$
as a product of simple reflections, and let
$\tilde{w} = \sigma_{\alpha_1} \cdots \sigma_{\alpha_n}$ be the
canonical lift of $\hat{w}$ to $\widetilde{W}$.

\begin{definition}
Given $\phi$, we define a homomorphism $\phi_o : W_F \rightarrow {}^L G$ by
\begin{enumerate}
\item $\phi_o|_{I_F} \equiv 1$
\item $\phi_o(\Phi) = \hat{\theta} \tilde{w}$
\end{enumerate}
\end{definition}

By the theory in section \ref{preliminaries}, $\phi$ and $\phi_o$ give
rise to a character $\chi_{\phi}$ and $\chi_{\phi_o}$ of
$T(F) = T^{\Phi_{\sigma}}$, respectively.  By the theory in section
\ref{groupsoftypeL}, $\phi$ and $\phi_o$ gives rise to characters
$\chi$ and $\chi_o$ of $T(E)_{\Gamma}$, respectively.

\begin{proposition}
If $G$ is semisimple, then $\chi \otimes \chi_o^{-1} = \chi_{\phi}$.
\end{proposition}

\proof
Since $G$ is semisimple, $T(F)$ is compact.  In particular,
$\HT{0}(\Gamma, T(E)) = 0$, so we have the following exact
sequence:
$$1 \rightarrow \HT{-1}(\Gamma, T(E)) \rightarrow T(E)_{\Gamma} \rightarrow T(F) \rightarrow 1$$
Note that $T(F) = T(\mathcal{O}_F)$, and so in particular
$T(\mathcal{O}_E)_{\Gamma}$ surjects onto $T(F)$ via the norm map
$\Nm$.  This, together with the fact that $\HT{-1}(\Gamma, T(E)) \neq 0$,
gives us that $\HT{-1}(\Gamma,T(E))$ and
$T(\mathcal{O}_E)_{\Gamma}$ together generate $T(E)_{\Gamma}$.

Since $\phi_o|_{I_F} \equiv 1$, $\chi_o$ is trivial on
$T(\mathcal{O}_E)_{\Gamma}$.  By lemma
\ref{grossanddebackerreedercompatibility}, we have that
$(\chi \otimes \chi_o^{-1})|_{T(\mathcal{O}_E)_{\Gamma}} = \chi_{\phi}|_{T(\mathcal{O}_F)}$.
Therefore, $\chi \otimes \chi_o^{-1}$ is correct on
$T(\mathcal{O}_E)_{\Gamma}$.  Moreover, by Lemma
\ref{toralmodification}, we also have that $\chi$ and $\chi_o$ have
the same restriction to $\HT{-1}(\Gamma, T(E))$.  Therefore,
$\chi \otimes \chi_o^{-1}$ is a character of $T(F)$.  Since
$\chi \otimes \chi_o^{-1}$ is correct on $T(\mathcal{O}_E)_{\Gamma}$, and
since $T(\mathcal{O}_E)_{\Gamma} \xrightarrow{\Nm} T(F)$ is surjective,
we are done.  (Do we need at some point that
$\HT{-1}(\Gamma, T(E)) \bigcap T(\mathcal{O}_E)_{\Gamma} = 1$?
It's true, because otherwise, there would be a nontrivial element in
$T(\mathcal{O}_E)_{\Gamma}$ that mapped to the trivial element in
$T(F)$ via the norm map.  But this can't be, because
$T(\mathcal{O}_E)_{\Gamma}$ is isomorphic to $T(\mathcal{O}_F)$ via
the norm map.)
\qed

In the case that $G$ is semisimple, there was nothing special about
$\phi_o$.  In fact, by the same arguments, the rectifier is obtained
from any Langlands parameter $\phi'$ such that $\phi'|_{I_F} \equiv 1$
and $\phi'(\Phi) = w'$, where $w'$ was any lift of $\hat{w}$ to $N$.
However, we will see that the Tits group lift is forced on us when we
consider groups that are not necessarily semisimple, such as $GL(n,F)$.

\section{Rectifiers for $GL(n,F)$}\label{GL(n)}

We first recall the structure of the rectifiers for $GL(n,F)$ in our
setting (see \cite{bushnellhenniart}).

\begin{lemma}
  Suppose that $(E/F, \xi)$ is an admissible pair such that $E/F$ is
  unramified of degree $n$ and $\xi|_{\mathcal{O}_E} \equiv 1$.  Let
  ${}_F \nu_{\xi}$ be the associated rectifier, as defined in
  \cite{bushnellhenniart}.  Then ${}_F \nu_{\xi}$ is unramified and
  ${}_F \nu_{\xi}(\varpi) = (-1)^{n-1}$.
\end{lemma}

\proof
See \cite[Proposition 21]{bushnellhenniart}.
\qed

\MAxxx{In this section somewhere, should we calculate $\HT{0}$ and
  $\HT{-1}$ for unramified tori of $GL(n)$? We have done this
  already, and it's easy.  Maybe we can use it for something, or
  explain something with it (like Gross's construction, together with
  our rectifiers, for $GL(n)$.  That would be nice)}

\begin{thebibliography}{9}

\bibitem{adrian}
  Moshe Adrian,
  \emph{A new realization of the Langlands correspondence for $PGL(2,F)$}, Journal of Number Theory 133 (2013) 446-–474.

\bibitem{amano}
  Kazuo Amano,
  \emph{A note on the Galois cohomology groups of algebraic tori}, Nagoya Math. J. Volume 34 (1969), 121-127.

\bibitem{bushnellhenniart}
  Colin Bushnell, Guy Henniart,
  \emph{The essentially tame local Langlands correspondence, III: the general case}, Proc. Lond. Math. Soc. (3) 101 (2010), no. 2, 497–553.

\bibitem{debackerreeder}
  Stephen DeBacker and Mark Reeder,
  \emph{Depth-zero supercuspidal $L$-packets and their stability.}
  Ann. of Math. (2) 169 (2009), no. 3, 795--901.

\bibitem{geo}
  Geo Kam-Fai Tam,
  \emph{Transfer relations in essentially tame local Langlands correspondence}, Ph.D. thesis, University of Toronto, 2012.

\bibitem{howe}
  Roger Howe,
  \emph{Tamely ramified supercuspidal representations of $GL_n(F)$},
   Pacific Journal of Math.  73  (1977),  437--460.

\bibitem{kaletha}
  Tasho Kaletha, \emph{Simple Wild L-packets}, J. Inst. Math. Jussieu (2013) 12(1), 43-75.

\bibitem{moyprasad1}
  Allen Moy, Gopal Prasad,
  \emph{Jacquet functors and unrefined minimal $K$-types},
   Comment. Math. Helv. 71 (1996), no. 1, 98--121.

\bibitem{reeder}
  Mark Reeder,
  \emph{Supercuspidal $L$-packets of positive depth and twisted Coxeter elements},
  J. Reine Angew. Math. 620 (2008), 1-33.

\bibitem{serre1}
  Jean-Pierre Serre,
  \emph{Local Fields}, Graduate Texts in Mathematics, 67. Springer-Verlag, New York-Berlin, 1979.

\end{thebibliography}

\end{document}
