\documentclass[11pt]{amsart}
\usepackage{amsmath,amscd,amssymb,latexsym, amsfonts}
\usepackage{mathtools}

\theoremstyle{plain}
\newtheorem{iassumption}{Assumption}
\newtheorem{theorem}{Theorem}
\newtheorem{conjecture}[theorem]{Conjecture}
\newtheorem{proposition}[theorem]{Proposition}
\newtheorem*{corollary}{Corollary}
\newtheorem{hypothesis}[theorem]{Hypothesis}
\newtheorem{assumption}[theorem]{Assumption}
\newtheorem{lemma}[theorem]{Lemma}
\newtheorem{question}[theorem]{Question}
\newtheorem{exercise}[theorem]{Exercise}
\newtheorem{statement}[theorem]{Statement}
\newtheorem{example}[theorem]{Example}
\newtheorem*{problem}{Problem}

\newcommand{\MAxxx}[1]{$\clubsuit$\footnote{#1}}
\newcommand{\DRxxx}[1]{$\spadesuit$\footnote{#1}}
\newcommand{\HT}[1]{\hat{\HH}{}^{#1}}

\theoremstyle{definition}
\newtheorem{definition}[theorem]{Definition}
\newtheorem{remark}[theorem]{Remark}

\setlength{\oddsidemargin}{0.2in}
\setlength{\evensidemargin}{0.2in}
\setlength{\textwidth}{6.1in}

\DeclareMathOperator{\Gal}{Gal}
\DeclareMathOperator{\val}{val}
\DeclareMathOperator{\HH}{H}
\DeclareMathOperator{\Ad}{Ad}
\DeclareMathOperator{\Nm}{Nm}
\DeclareMathOperator{\Hom}{Hom}
\DeclareMathOperator{\Spec}{Spec}
\DeclareMathOperator{\Res}{Res}
\DeclareMathOperator{\Fr}{Fr}
\DeclareMathOperator{\Ind}{Ind}
\DeclareMathOperator{\gen}{gen}
\DeclareMathOperator{\Z}{Z}


\DeclareMathOperator{\GL}{GL}
\DeclareMathOperator{\PGL}{PGL}
\DeclareMathOperator{\SL}{SL}
\DeclareMathOperator{\SU}{SU}
\DeclareMathOperator{\PU}{PU}
\DeclareMathOperator{\SO}{SO}
\DeclareMathOperator{\Sp}{Sp}

\newcommand{\st}{\ensuremath{\ \ \ \vert\ }}
\newcommand{\TT}{\mathcal{T}}
\newcommand{\C}{\mathcal{C}}
\newcommand{\CC}{\mathbb{C}}
\newcommand{\CCx}{\mathbb{C}^\times}
\newcommand{\OK}{\mathcal{O}_K}
\newcommand{\OKn}{\mathcal{O}_{K_n}}
\newcommand{\PK}{\mathcal{P}_K}
\newcommand{\PL}{\mathcal{P}_L}
\newcommand{\OL}{\mathcal{O}_L}
\newcommand{\ZZ}{\mathbb{Z}}
\newcommand{\QQ}{\mathbb{Q}}
\newcommand{\Gm}{\mathbb{G}_m}
\newcommand{\Lx}{L^\times}

\newcommand{\T}{\mathbf{T}}
\newcommand{\G}{\mathbf{G}}
\newcommand{\N}{\mathbf{N}}
\newcommand{\W}{\mathbf{W}}
\newcommand{\Galk}{\Gamma}
\newcommand{\hatT}{\hat{T}}

\newcommand{\Treg}{T^{\operatorname{reg}}}
\newcommand{\Threg}{\hatT^{\operatorname{reg}}}
\newcommand{\Tirr}{T^{\operatorname{irr}}}
\newcommand{\Thirr}{\hatT^{\operatorname{irr}}}

\newcommand{\Weil}{\mathcal{W}}
\newcommand{\WD}{\mathcal{W}'}
\newcommand{\Lpack}{\mathcal{L}}
\newcommand{\Pgen}{P_G^{\gen}}
\newcommand{\bmu}{\boldsymbol\mu}
\newcommand{\mugen}{\bmu^{\gen}}

\newcommand{\la}{\langle}
\newcommand{\ra}{\rangle}

\newcommand{\invlim}[1]{\varprojlim_{#1}}
\newcommand{\Normalizer}[2]{\operatorname{N}_{#2}(#1)}

\begin{document}
\title{The Lemma}

Suppose $k$ is a finite field, $\G$ is a connected reductive group over $k$ and $\T$ is a maximal torus in $G$.  Let $\N$ be the normalizer of $\T$ and $\W$ the absolute Weyl group $\N / \T$.  The absolute Galois group $\Galk = \Gal(\bar{k}/k)$ acts on everything; write $G = \G^\Galk$, $T = \T^\Galk$ and $W = \W^\Galk$.  Set $\hatT = \Hom(T, \CC^\times)$.  We have that $W$ acts on $T$ and $\hat{T}$; we call an element or character \emph{regular} if it is not fixed by any nontrivial element of $W$ (this condition is sometimes referred to as strongly regular).  Write $\Treg$ for the set of regular elements in $T$, $\Threg$ for the set of regular characters, $\Tirr$ for the set of irregular elements and $\Thirr$ for the set of irregular characters.  Note that in general neither the regular nor the irregular elements of $T$ or $\hatT$ form a subgroup.

\begin{problem}
Identify the set $Q_{\T}$ of $\alpha \in \hatT$ with the following property:
\begin{itemize}
\item For every $\chi \in \Threg$ there is a $w \in W$ with $\frac{\chi}{w(\chi)} = \alpha$.
\end{itemize}
\end{problem}

\section{Facts}

\begin{proposition} \label{pigeonhole}
Suppose $\#\Threg > \# W \cdot \# \Thirr$.  Then $Q_{\T} = \{ 1 \}$.
\end{proposition}
\begin{proof}
For $w \in W$, set
$$S_w = \{\chi \in \Threg \st  \frac{\chi}{w(\chi)} = \alpha\}.$$
By the pigeonhole principle, there is a $w \in W$ with $\# S_w > \# \Thirr$.  Pick $\chi \in S_w$; since $\# S_w > \#\Thirr$ there is some $\chi' \in S_w$ with $\frac{\chi}{\chi'}$ regular.  We now have
$$\frac{\chi}{w(\chi)} = \alpha = \frac{\chi'}{w(\chi')}$$
and therefore $\frac{\chi}{\chi'}$ is fixed by $w$.  Since $\frac{\chi}{\chi'}$ is regular, we must have $w = 1$ and thus 
$$\alpha = \frac{\chi}{\chi} = 1.$$
\end{proof}

Recall that Frobenius acts on $X^*(\T)$ via an endomorphism $F = qF_0$, where $F_0$ is an automorphism of finite order.  So it makes sense to vary $q$: we fix $F_0$ and consider the tori dual to the $\Galk$-modules with Frobenius acting through $qF_0$.

\begin{corollary}[{c.f. \cite[Lemma 8.4.2]{carter}}]
Consider the family of tori $\T_q$ with the same $F_0$.  Then for sufficiently large $q$, $Q_{\T_q} = \{ 1 \}$ (regardless of the $\G$ in which $\T_q$ is embedded).
\end{corollary}
\begin{proof}
We will write $\T$ for a general torus in the family and $r$ for the common dimension.  For $w \in W$ with $w \ne 1$ the centralizer $\Z_{\T}(w)$ is a proper $F$-stable subgroup of $\T$, and thus $\dim(\Z_{\T}(w)) \le r - 1$.  By \cite[3.3.5]{carter} $\# T$ is a polynomial in $q$ of degree $r$ and $\# \Z_{\T}(w)^\Galk = \# \Z_T(w)$ is a polynomial in $q$ of degree at most $r-1$.  Thus the ratio
$$\frac{\# \Treg}{\# \Tirr} = \frac{\# T - \sum_{1 \ne w \in W} \# \Z_T(w)}{\sum_{1 \ne w \in W} \# \Z_T(w)}$$
grows without bound as $q$ does.  Applying this result to the dual torus $\hatT$ yields the corollary.
\end{proof}

\begin{proposition}
The set $Q_{\T}$ is a subgroup of $\hatT$, contained within $\Thirr$ and stable under the action of $W$.
\end{proposition}
\begin{proof}
If $\alpha \in \Threg \cap Q_{\T}$ then there is some $w \in W$ with $\frac{\alpha}{w(\alpha)} = \alpha$, so $\alpha = 1$ which is not regular.

We now show that $Q_{\T}$ is a group.  Certainly $1 \in Q_{\T}$.  Suppose $\alpha, \alpha' \in Q_{\T}$ and $\chi \in \Threg$.  Then there are $w, w' \in W$ with
\begin{align*}
\frac{\chi}{w(\chi)} &= \alpha, \\
\frac{w(\chi)}{w'(w(\chi))} &= \alpha'.
\end{align*}
Multiplying the two relations yields $\frac{\chi}{w'w(\chi)} = \alpha\alpha'$, so $\alpha\alpha' \in Q_{\T}$.  Since $Q_{\T}$ is finite and contained within $\hatT$ it is thus closed under inversion.

Finally, suppose $\tau \in W$.  Given $\chi \in \Threg$ with $\alpha = \frac{\chi}{w(\chi)}$ we have
$$\tau(\alpha) = \frac{\tau(\chi)}{\tau w(\chi)} = \frac{\tau(\chi)}{w' \tau(\chi)}$$
for some $w' \in W$.  Since $\tau$ permutes the regular characters we get that $\tau(\alpha) \in Q_{\T}$.
\end{proof}

\begin{proposition} \label{orderdiv}
If $\alpha \in Q_{\T}$ has order $m$ and $\chi \in \Threg$ has order $k$ then $m$ divides $k$.
\end{proposition}
\begin{proof}
There is a $w \in W$ with
$$\frac{\chi}{w(\chi)} = \alpha.$$
Since $w(\chi)$ also has order $k$, raising both sides to the $k$th power  yields $\alpha^k = 1$.
\end{proof}

\section{Rank 1 and 2 tori}

Write $k_r$ for the degree $r$ extension of $k$, $\T_r$ for the restriction of scalars torus $\Res_{k_r/k} \Gm$ and $\T_r^s$ for $\{x \in \Res_{k_r/k} \Gm \st \Nm_{k_r/k_s} x = 1\}$.

In rank one there are two tori:
\begin{itemize}
\item $\Gm$
\item $\T_2^1$
\end{itemize}

In rank two there are 
\begin{itemize}
\item $\T_1 \times \T_1$
\item $\T_1 \times \T_2^1$
\item $\T_2^1 \times \T_2^1$
\item $\T_2$
\item $\T_3^1$
\item $\T_4^2$
\item $\T_6^3 \cap \T_6^2 \simeq \T_6^3 / \T_2^1$
\end{itemize}

There are two rank 1 semisimple groups over $k$: $\SL_2$ and $\PGL_2$.  In both cases, for both the split and anisotropic tori, the Weyl group acts as inversion, so the irregular characters are precisely those of order $1$ or $2$.  When $q = 2^k$ neither torus has any nontrivial irregular characters so $Q_{\T} = \{1\}$.  When $q = 3$ the split torus has no regular characters and thus $Q_{\T} = \Thirr$ has order $2$.  The anisotropic torus has $4$ elements, $2$ regular and $2$ irregular characters.  Again $Q_{\T}$ has order $2$ since the Weyl group exchanges the two regular characters and their quotient is the nontrivial irregular character.  For $q = 5$ the split torus has the same behavior as the anisotropic torus for $q = 3$: $Q_{\T}$ has order $2$.  The anisotropic torus has regular characters of order $3$ and $6$ so $Q_{\T} = \{1\}$ by Proposition \ref{orderdiv}; the same holds for the split torus when $q = 7$.  For the anisotropic torus ($q = 7$) and all larger $q$, $Q_{\T}$ is trivial by Proposition \ref{pigeonhole}.

\begin{tabular}{l|l|l}
Group & Torus & Weyl Group Action \\
\hline && \\
$\SL_3,$ & $\T_1 \times \T_1$ & Full $S_3$ action \\
$\PGL_3$ & $\T_2$ & $\ZZ/2$: Galois action \\
& $\T_3^1$ & $\ZZ/3$: Galois action \\
\hline && \\
$\SU_3$ & $\T_6^3 \cap \T_6^2$ & trivial Weyl group \\
& $\T_2$ & $\ZZ/2$: Galois action \\
$\PU_3$ & $\T_6^3 / \T_2^1$ & trivial Weyl group \\
& $\T_2$ & $\ZZ/2$: Galois action \\
\hline && \\
$\SO_5,$ & $\T_1 \times \T_1$ & Full $D_8$ action \\
$\Sp_4$ & $\T_1 \times T_2^1$ & $\ZZ/2 \times \ZZ/2$: inversion on each component \\
& $\T_4^2$ & $\ZZ/4$: Galois action \\
\hline && \\
$\SO_4$ & $\T_1 \times \T_1$ & $\ZZ / 2 \times \ZZ / 2$: inversion on each component \\
& $\T_1 \times \T_2^1$ & $\ZZ / 2 \times \ZZ / 2$: inversion on each component \\
& $\T_2^1 \times \T_2^1$ & $\ZZ / 2 \times \ZZ / 2$: inversion on each component \\
& $\T_4^2$ & $\ZZ / 2$: subgroup of Galois \\
\hline && \\
$\SL_2 \times \SL_2,$ & $\T_1 \times \T_1$ & $\ZZ / 2 \times \ZZ / 2$: inversion on each component \\
$\SL_2 \times \PGL_2,$ & $\T_1 \times \T_2^1$ & $\ZZ / 2 \times \ZZ / 2$: inversion on each component \\
$\PGL_2 \times \PGL_2$ & $\T_2^1 \times \T_2^1$& $\ZZ / 2 \times \ZZ / 2$: inversion on each component
\end{tabular}

\begin{thebibliography}{9}
\bibitem{carter}
  Roger Carter,
  \emph{Finite Groups of Lie Type: Conjugacy Classes and Complex Characters}.  Wiley \& Sons, Chichester, 1985.
\end{thebibliography}

\end{document}